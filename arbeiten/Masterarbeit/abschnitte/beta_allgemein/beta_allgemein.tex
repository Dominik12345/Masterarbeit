\clearpage
\section{Allgemeine Untersuchung einer $\beta$-Funktion}

	In Abschnitt \ref{selbstenergie} wurde die $\beta$-Funktion einer skalaren 
	$SU(N)$ berechnet, bei der die Energieabhängigkeit der renormierten 
	Kopplungskonstanten durch eine gewöhnliche 
	\textit{Differenzialgleichung (DGL)} erster 
	Ordnung beschrieben wird. Für den allgemeinen Fall ergibt sich ein 
	System aus gekoppelten DGLs.
   Dies lässt sich mit $g:=(g_1,\ldots,g_N)^\text{T}$ und 
  $\beta:=(\beta_1,\ldots,\beta_N)^\text{T}$ 
  auch als
  \begin{equation}
   \mu \frac{\d}{\d \mu} g(\mu)=\beta(g) \label{eq:beta_allgemein:dgl}
  \end{equation}
  schreiben. Der $\mathbb{R}^N$ ist dann der \textit{Phasenraum} 
  der $\beta$-Funktion. Wie an Gleichung \eqref{eq:renormierung:beta-funktion} 
  zu erkennen ist, sind die Beiträge zur $\beta$-Funktionen bei 
  dimensionslosen Kopplungen mindestens 
  $\mathcal{O}(g^3)$.
  
    Lösungen der Gleichung \eqref{eq:beta_allgemein:dgl} 
    heißen \textit{Trajektorien}, ein Punkt 
    $g^*$ im Phasenraum mit 
    $\beta(g^*)=0$ heißt \textit{Fixpunkt}.
  
  Eine physikalisch sinnvolle QFT sollte für beliebige Energieskalen, 
  insbesondere für beliebig hohe Energieskalen,
  Vorhersagen machen können, sofern sie nicht von Beginn an als effektive 
  Theorie angesetzt ist. Für die Kopplungskonstanten heißt das, dass 
  eine Trajektorie für 
  $\mu \to \infty$ die Grenzwerte $g(\mu) \to g^*$ und 
  $\beta(g(\mu)) \to 0$ haben muss, um ein stabiles 
  Hochenergieverhalten zu gewährleisten. Demnach sind gerade die Fixpunkte der 
  $\beta$-Funktion die möglichen Grenzwerte der Kopplungskonstanten.
  
  Für die Untersuchung von $\beta$-Funktionen haben sich die folgenden 
  Bezeichnungen entwickelt.
  \begin{description}
   \item[Gaußscher Fixpunkt: ] Der Fixpunkt $g^*=0$ heißt Gaußscher Fixpunkt.
   \item[Banks-Zaks Fixpunkt: ] Ein Fixpunkt $g^*\neq 0$, der physikalsich 
      sinnvoll und für $\mu\to 0$ erreicht wird, heißt Banks-Zaks oder 
      Caswell-Banks-Zaks Fixpunkt.
   \item[Landau Pol: ] Besitzt die Lösung des Problems 
      \eqref{eq:beta_allgemein:dgl} mit 
      Anfangswert $g(\mu_0)=g_0$ eine Polstelle $\mu_\text{Pol}<\infty$, sodass 
      $g(\mu)\overset{\mu\to\mu_\text{Pol}}{\longrightarrow}\infty$, dann 
      spricht man von einem Landau-Pol (vgl. \eqref{eq:beta_im_SM:landau_pol}).
  \end{description}
  
  \subsection{Vereinfachung des mathematischen Problems}
      Die Berechnung einer Trajetkorie als Lösung zum Anfangswertproblem 
      \eqref{eq:beta_allgemein:dgl} mit Anfangswert $g(\mu_0)=g_0$ 
      ist in der Regel analytisch nicht möglich. Durch einige einfache 
      Schritte lässt sich das Problem jedoch zunächst in die einfacher 
      zuhandhabende Form eines autonomen DGL-Systems überführen und sich das 
      Verhalten in der Nähe eines Fixpunktes bestimmen.
      
      In \cite{GR_Weinberg} schlägt S. Weinberg die Einführung der 
      dimensionslosen Kopplungskonstanten\begin{equation}
       \bar{g}_i(\mu):= \mu^{-d_i} g_i(\mu)
      \end{equation}
      vor, wobei $d_i$ die Massendimension der Kopplungskonstanten $g_i$ ist. 
      Es ist oft hilfreich die quadratischen Kopplungskonstanten 
      \begin{equation}
       \alpha_i(\mu):= \mathcal{N} \left(\bar{g}_i(\mu)\right)^2
      \end{equation}
      zu definieren um den Grad der $\beta$-Funktion zu verringern 
      (vgl. \cite{Scale_of_dark_QCD}, \cite{Asymptotic_safety_guaranteed}). 
      Dabei dient $\mathcal{N}$ als Normierungskonstante, die insbesondere von 
      dimensionslosen Größen wie Teilchenzahlen oder Größen der Symmetriegruppe 
      abhängen kann.
      \begin{beispiel}
	  Für ein eindimensionales Problem
	  \begin{equation}
	  \mu \frac{\d}{\d \mu} g(\mu) = X^g g(\mu)^3 + Y^g g(\mu)^5
	  \end{equation}
	  und für den einfachen Fall $[g]=0$ kann 
	  \begin{equation}
	  \alpha(\mu) := \mathcal{N} g(\mu)^2 \quad \Rightarrow \quad  
	  \frac{\d g}{\d \mu} g=\frac{1}{2\mathcal{N}} 
	  \frac{\d \alpha}{\d \mu} 
	\end{equation}
	definiert werden. 
	So erhält man eine einfachere DGL mit den 
	Koeffizienten \\$X=2X^g \mathcal{N}^{-1}$ und 
	$Y=2Y^g\mathcal{N}^{-2}$,
	 \begin{equation}
	  \mu \frac{\d}{\d \mu} \alpha(\mu) =X\alpha(\mu)^2+Y
	  \alpha(\mu)^3 \quad .
	\end{equation}
	
    \end{beispiel}
    Naheliegend wird wieder $\alpha=(\alpha_1,\ldots,\alpha_N)^\text{T}$ und 
    $\beta(\alpha)=\beta(g\circ\alpha)$ geschrieben.
    
    Der physikalisch sinnvolle Wertebereich für die Energieskala $\mu$ ist 
    $(0,\infty)$. Mit der Renormierungsgruppenzeit (RG-Zeit) $t$, definiert als
    \begin{equation}
     t(\mu):=\ln\left(\frac{\mu}{\Lambda}\right) \quad
     \Leftrightarrow \quad \mu(t)=e^t \quad , \label{eq:beta_allgemein:RG-Zeit}
    \end{equation}
    gibt es eine Bijektion $(0,\infty)\overset{t}{\longleftrightarrow}
    (-\infty,\infty)$, die es erlaubt die Kopplungskonstante als 
    \begin{equation}
    \tilde{\alpha}(t):=\alpha\left(e^t\right)=\alpha(\mu)
    \end{equation}
    zu schreiben. Der Parameter $\Lambda$ ist beliebig und hat keine 
    physikalische Bedeutung, er wird später lediglich die Extrapolation der 
    Fixpunkte übersichtlicher gestalten. Es folgt
    \begin{equation}
     \mu\frac{\d}{\d\mu}\alpha(\mu) = \mu \underbrace{\frac{\d t}{\d \mu}}_{=\mu^{-1}}
     \frac{\d}{\d t} \tilde{\alpha}(t)
     =\frac{\d}{\d t} \tilde{\alpha}(t) \quad .
    \end{equation}
    Damit ist Gleichung \eqref{eq:beta_allgemein:dgl} äquivalent zu dem 
    autonomen Differentialgleichungssystem\begin{equation}
     \frac{\d}{\d t} \alpha(t)=\beta(\alpha) \quad , 
     \label{eq:beta_allgemein:dgl_alpha}
    \end{equation}
    wobei $\tilde{\alpha}$ wieder zu $\alpha$ umbenannt wurde.
    
  \subsection{Verhalten in einer Umgebung eines Fixpunktes}\label{beta_allgemein:Verhalten}
    Um das Verhalten der Kopplungskonstanten $\alpha(t)$ in der Nähe eines 
    Fixpunktes zu untersuchen wird die \textit{Stabilitätsmatrix} 
    \begin{equation}
     \frac{\partial \beta}{\partial \alpha}:= 
     \left( \frac{\partial \beta_i}{\partial \alpha_j} \right)_{1\leq i,j 
     \leq N}
    \end{equation}
    eingeführt \cite{GR_Weinberg}. 
    Außgewertet am Punkt $\alpha^*$ 
    ist die Schreibweise $\dbdafix$. 
    Ein Fixpunkt $\alpha^*$ heißt \textit{hyperbolisch}, wenn alle Eigenwerte 
    von $\dbdafix$ einen von Null verschiedenen Realteil besitzen 
    \cite{Bronstein4}, sonst heißt er \textit{nicht-hyperbolisch}.
    
    
    Der Zusammenhang zu der Stabilität des Fixpunktes ist folgendermaßen zu 
    erkennen. 
    
    In der Nähe eines hyperbolischen Fixpunktes $\alpha^*$ kann Gleichung 
    \eqref{eq:beta_allgemein:dgl_alpha} durch ihre Linearisierung
     beschrieben werden. Da bei einem hyperbolischen Fixpunkt die 
     Eigenvektoren $\{e_i\}$ der Stabilitätsmatrix eine Basis sind, kann 
     $(\alpha(t)-\alpha^*)$ in Eigenvektoren zerlegt werden,
     \begin{equation} \ddt \alpha(t) \simeq \dbdafix \left(\alpha(t)-\alpha^*\right) 
     = \dbdafix \sum_{i=1}^N K_i(t) e_i  \quad . 
     \label{eq:beta_allgemein:beta_linear}
     \end{equation}
     Für die Koeffizienten $\{K_i\}$ in der Basis der Eigenvektoren ergibts 
     sich das entkoppelte DGL-System
    \begin{align}
     &&\ddt \left(\sum\limits_{i=1}^N K_i(t)e_i + \alpha^* \right) &=
     \dbdafix \sum\limits_{i=1}^N K_i(t) e_i 		&&&&&& \\
     &\Leftrightarrow &  \ddt K_i(t) &= K_i(t) \lambda_i  
     \label{eq:beta_allgemein:K} 			\\
     &\Rightarrow &
     K_i(t)&=\e^{\lambda_i t} K_i(0)      	
     \label{eq:beta_allgemein:K_loesung} 	\quad ,
    \end{align}
    wobei $\lambda_i$ der Eigenwert zu $e_i$ ist. Damit kann das Verhalten der 
    Kopplungskonstanten durch
    \begin{equation}
     \alpha(t)=\sum\limits_{i=1}^N \e^{\lambda_i t} K_i(0) e_i +\alpha^* 
     \label{eq:beta_allgemein:loesung}
    \end{equation}
    beschrieben werden.
    Dieses Ergebnis ist unter anderem in \cite{Weinberg:1976}, 
    \cite{GR_Weinberg} und \cite{Asymptotic_safety_guaranteed} zu sehen.
    Aus Gleichung \eqref{eq:beta_allgemein:loesung} wird dann klar, dass 
    der Untervektorraum, der durch die Eigenvektoren $\{e_i\}$ mit 
    $\Re\lambda_i < 0$ aufgespannt wird, die Punkte in der Fixpunktumgebung 
    enthält, die für 
    $t \to \infty$ in den Fixpunkt hineinlaufen, entsprechend enthält der 
    Untervektorraum zu Eigenvektoren mit $\Re\lambda_i > 0$ alle Punkte, die 
    den Fixpunkt für $t\to -\infty$ erreichen beziehungsweise aus ihm 
    herauslaufen. Alle weiteren Punkte der Fixpunktumgebung liegen auf 
    Trajektorien, die den Fixpunkt nicht enthalten.
    
    Ein Eigenvektor mit positivem Eigenwert wird oft auch als IR-attraktiv, mit 
    einem negativen Eigenwert als IR-repulsiv bezeichnet \cite{Weinberg:1976}. 
    Da in dieser Arbeit jedoch das UV-Verhalten von Interesse ist, werden die 
    folgenden Bezeichnungen verwendet.
    \begin{enumerate}
     \item Ein Fixpunkt heißt \textit{attraktiver (UV-)Fixpunkt}, wenn 
     	$\Re \lambda_i<0 \forall i$.
     \item Ein Fixpunkt heißt \textit{repulsiver (UV-)Fixpunkt}, wenn 
	$\Re \lambda_i>0 \forall i$.
     \item Falls ein Fixpunkt weder attraktiv noch repulsiv ist, d.h. wenn es 
     sowohl Trajektorien gibt, die in ihn hinein-, als auch welche die 
     hinauslaufen, wird er \textit{Sattelpunkt} genannt.
     \item
      Die Menge der in den Fixpunkt hineinlaufenden Kurven heißt 
      \textit{kritische Hy\-per\-flä\-che} $\Mc$ (critical manifold) oder 
      \textit{UV-Hyperfläche} des 
      Fixpunktes.
    \end{enumerate}


    Bei allgemeinen Betrachtungen (vgl. \cite{Weinberg:1976}) 
    wird der nicht-hyperbolische Fall $\lambda_i = 0$ oft als 
    Sonderfall nicht weiter betrachtet. Bei der Untersuchung einer konkreten 
    $\beta$-Funktion kommt dieser Sonderfall aber auf natürliche Weise  
    zu Stande, sobald ein Fixpunkt einen Wert $\alpha^*_i = 0$ besitzt. In diesem 
    Fall ist es schwierig allgemeine Aussagen zu treffen.
    

%     Sein nun also $\alpha^*$ ein Fixpunkt, $\{e_i\}$ die Basis aus Eigenvektoren der 
%     Stabilitätsmatrix und der Eigenwert $\lambda_k=0$, alle anderen $\lambda_i \neq 0$ 
%     für $i\neq k$. In zweiter Ordnung gilt
%     \begin{equation}
%      \beta_i(\alpha) \simeq \sum\limits_{m=1}^N \frac{\partial \beta_i(\alpha)}{\partial
%      \alpha_m} \left(\alpha_m-\alpha^*_m\right) + \frac12 \sum\limits_{m,n=1}^N 
%       \left(\alpha_m-\alpha^*_m\right)
%      \frac{\partial^2 \beta_i(\alpha)}{\partial\alpha_m \partial\alpha_n}
%      \left(\alpha_n-\alpha^*_n\right) \quad ,
%     \end{equation}
%     sodass für die Koeffizienten $\{K_i\}$ in der $\{e_i\}$-Basis folgt
%     \begin{equation}
%      \ddt K_i(t) = \lambda_i K_i(t) + \frac12
%      \sum\limits_{j=1}^N \left( e_j\cdot \nabla \lambda_i \right) K_i(t)K_j(t) \quad .
%     \end{equation}
%     Für alle $\lambda_i \neq 0 $ reicht es, die Gleichungen in Ordnung $\mathcal{O}(K_i)$, 
%     also \eqref{eq:beta_allgemein:K} zu lösen, für $\lambda_k=0$ verschwindet jedoch die 
%     erste Ordnung, sodass sich als DGL
%     \begin{equation}
%      \ddt K_k(t) = \frac12 \sum\limits_{j=1}^N \left( e_j\cdot \nabla \lambda_k 
%      \right) K_k(t)K_j(t) \quad . \label{eq:beta_allgemein:K_gekoppelt}
%     \end{equation}
%     ergibt. Auf $\Mc$ sind $K_i(t)\equiv 0$ für alle $i$ mit $\lambda_i > 0$, sodass die 
%     DGL weiter vereinfacht werden kann, 
%     \begin{equation}
%      \ddt K_k(t) \overset{t\to\infty}{\simeq} \frac12 \left( e_k \cdot 
%      \nabla\lambda_k \right) K_k(t)^2 \quad .
%     \end{equation}
%     Die Lösung ist
%     \begin{equation}
%      K_k(t) = \frac{1}{K_k(0)^{-1} - \frac12 \left( e_k\cdot \nabla \lambda_k \right) t } \quad ,
%     \end{equation}
%     wird $e_k$ so gewählt, dass $K_k(t)\geq 0$, dann ergibt sich die Bedingung
%     \begin{equation}
%      e_k \cdot \nabla\lambda_k < 0 \quad , \label{eq:beta_allgemein:lambda0_bedingung} 
%     \end{equation}
%     damit $K_k(t)\overset{t\to\infty}{\longrightarrow}0$ ohne einen Pol zu passieren, der die 
%     Reihenentwicklung der $\beta$-Funktion unzulässig machen würde. Falls mehr als ein 
%     Eigenwert gleich Null ist, ergibt sich in \eqref{eq:beta_allgemein:K_gekoppelt} ein 
%     System gekoppelter DGLs.

%  \subsection{Experimentelle Daten und kritische Hyperfläche}


%     Während die $\beta$-Funktion als DGL-System aus dem erzeugenden Funktional 
%     und somit letztlich aus der postulierten Lagrangedichte hervorgeht, ist die 
%     Bestimmung der Trajektorie, die 
%     den Kopplungskonstanten "`unserer Welt"' entspricht eine rein 
%     experimentelle Aufgabe.  
  
 
    

  
    



      


  
  
  
  
