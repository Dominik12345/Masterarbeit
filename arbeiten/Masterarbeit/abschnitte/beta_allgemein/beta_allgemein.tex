\clearpage
\section{Untersuchung einer $\beta$-Funktion}

% \ref{eq:running_coupling} --- beta-funktion aus der gleichung zur quantenwirkung

  An Gleichung \ref{eq:running_coupling} erkennt man, dass die $\beta$-Funktion einer 
  QFT ein System $N$ gekoppelter, gewöhnlicher, nichtlinearer 
  Differentialgleichungen der Form 
  \begin{equation}
   \mu \frac{\d}{\d \mu} g_k (\mu)  = P_k^{M_L}(g_1,\ldots, g_N)=:\beta_k(g_1,\ldots,g_N) 
   , \quad k=1,\ldots,N \label{eq:beta_allgemein:dgl_system}
  \end{equation}
  ist. Dabei stellt jedes $P_k$ ein Polynom maximal $M_L$-ten Grades in den Kopplungskonstanten 
  dar. Der Grad des Polynoms hängt nur von der Ordnung der Störungstheorie ab, im Bild von 
  Feynmangraphen entspricht dies der maximalen Anszahl $L$ von Quantenschleifen, die in 
  der Berechnung berücksichtigt werden. Hier ist es naheliegend, das DGL-System als Problem 
  im $\mathbb{R}^N$ zu betrachtet. Mit $g:=(g_1,\ldots,g_N)^\text{T}$ und 
  $\beta:=(\beta_1,\ldots,\beta_N)^\text{T}$ lässt sich \eqref{eq:beta_allgemein:dgl_system} 
  auch als
  \begin{equation}
   \mu \frac{\d}{\d \mu} g(\mu)=\beta(g) \label{eq:beta_allgemein:dgl}
  \end{equation}
  schreiben. Dann heißt $\mathbb{R}^N$ auch der Phasenraum der $\beta$-Funktion.
  \begin{definition}
  Eine Trajektorie im Phasenraum ist eine Funktion $g:(0,\infty)\rightarrow \mathbb{R}^N$, 
  die die Gleichung \eqref{eq:beta_allgemein:dgl} löst.\\
  Ein Fixpunkt der $\beta$-Funktion ist ein Punkt $g^*\in \mathbb{R}^N$, für den
   $\beta(g^*)=0$ gilt.
  \end{definition}
  Damit eine QFT physikalisch sinnvoll ist, muss sie Vorhersagen für alle Energieskalen machen 
  können. Um eine auch für hohe Energieskalen $\mu$ gültige Theorie zu beschreiben muss demnach 
  der Wert $\lim\limits_{\mu\to\infty} g(\mu)$ existieren, ebenso muss der Wert $\lim\limits_{
  \mu\to 0}g(\mu)$ existieren, wenn die Theorie auch für niedrige Energieskalen gültige sein soll. 
  Damit die Grenzwerte existieren muss $\lim\nicefrac{\d }{\d \mu} g(\mu)
  =0$ sein, es sind also gerade die Fixpunkte der $\beta$-Funktion, die als Grenzwerte 
  in Frage kommen.
  
  Im Laufe der Untersuchung der $\beta$-Funktion haben sich die folgenden Bezeichnungen 
  entwickelt.
  \begin{description}
   \item[Gaußscher Fixpunkt: ] Ist der Punkt $g^*=0$ ein Fixpunkt der $\beta$-Funktion, so 
			      spricht man von einem Gaußschen Fixpunkt.
   \item[Banks-Zaks Fixpunkt: ] Ein Fixpunkt $g^*\neq 0$, der physikalsich sinnvoll und 
			      perturbativ ist heißt Banks-Zacks oder 
			      Caswell-Banks-Zaks Fixpunkt.
   \item[Landau Pol: ] Besitzt die Lösung des Problems \eqref{eq:beta_allgemein:dgl} mit 
		      Anfangswert $g(\mu_0)=g_0$ eine Polstelle $\mu_\text{Pol}<\infty$, sodass 
		      $g(\mu)\overset{\mu\to\mu_\text{Pol}}{\longrightarrow}\infty$, dann nennt 
		      man diesen Pol Landau-Pol.
  \end{description}
  
  \subsection{Vereinfachung des mathematischen Problems}
      Die Berechnung einer Trajetkorie als Lösung zum Anfangswertproblem 
      \eqref{eq:beta_allgemein:dgl} mit Anfangswert $g(\mu_0)=g_0$ 
      ist in der Regel analytisch nicht möglich. Durch einige einfache Schritte lässt sich das 
      Problem jedoch zunächst in eine einfacher zuhandhabende Form überführen und sich das 
      Verhalten in der Nähe eines Fixpunktes vorhersagen.
      
      In \cite{General_relativity} schlägt S. Weinberg die Einführung der dimensionslosen 
      Kopplungskonstanten
      \begin{equation}
       \bar{g}_i(\mu):= \mu^{-d_i} g_i(\mu)
      \end{equation}
      vor, wobei $d_i$ die Massendimension der Kopplungskonstanten $g_i$ ist. Bei der 
      Untersuchung der \QCDxdQCD-$\beta$-Funktion wird klar, dass die Erweiterung 
      \begin{equation}
       \alpha(\mu):= \left(\mu^{-d_i} g_i(\mu)\right)^n
      \end{equation}
      den Grad $M_L$ der $\beta$-Funktion verringern kann und somit das Problem weiter 
      vereinfacht (vgl. \cite{arXiv:1406.2337}, \cite{PhysRevD.89.063522}). 
      \begin{beispiel}
	Für ein eindimensionales Problem
	\begin{equation}
	  \mu \frac{\d}{\d \mu} g(\mu) = X g(\mu)^3 + Y g(\mu)^5
	\end{equation}
	definiere 
	\begin{equation}
	 \alpha(\mu)
	\end{equation}

	

      \end{beispiel}

      


  
  
  
  
