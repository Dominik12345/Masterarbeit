\clearpage
\section{Untersuchung einer $\beta$-Funktion}

% \ref{eq:running_coupling} --- beta-funktion aus der gleichung zur quantenwirkung

  An Gleichung \ref{eq:running_coupling} erkennt man, dass die $\beta$-Funktion einer 
  QFT eine Seite eines Systems $N$ gekoppelter, gewöhnlicher, nichtlinearer 
  Differentialgleichungen der Form 
  \begin{equation}
   \mu \frac{\d}{\d \mu} g_k (\mu)  = P_k^{M_L}(g_1,\ldots, g_N)=:\beta_k(g_1,\ldots,g_N) 
   , \quad k=1,\ldots,N \label{eq:beta_allgemein:dgl_system}
  \end{equation}
  ist. Dabei stellt jedes $P_k$ ein Polynom maximal $M_L$-ten Grades in den Kopplungskonstanten 
  dar. Der Grad des Polynoms hängt nur von der Ordnung der Störungstheorie ab, im Bild von 
  Feynmangraphen entspricht dies der maximalen Anszahl $L$ von Quantenschleifen, die in 
  der Berechnung berücksichtigt werden. Hier ist es naheliegend, das DGL-System als Problem 
  im $\mathbb{R}^N$ zu betrachtet. Mit $g:=(g_1,\ldots,g_N)^\text{T}$ und 
  $\beta:=(\beta_1,\ldots,\beta_N)^\text{T}$ lässt sich \eqref{eq:beta_allgemein:dgl_system} 
  auch als
  \begin{equation}
   \mu \frac{\d}{\d \mu} g(\mu)=\beta(g) \label{eq:beta_allgemein:dgl}
  \end{equation}
  schreiben. Dann heißt $\mathbb{R}^N$ auch der Phasenraum der $\beta$-Funktion.
  \begin{definition}
  Eine \textit{Trajektorie im Phasenraum} ist eine Funktion $g:(0,\infty)\rightarrow \mathbb{R}^N$, 
  die die Gleichung \eqref{eq:beta_allgemein:dgl} löst.\\
  Ein \textit{Fixpunkt der $\beta$-Funktion} ist ein Punkt $g^*\in \mathbb{R}^N$, für den
   $\beta(g^*)=0$ gilt.
  \end{definition}
  Damit eine QFT physikalisch sinnvoll ist, muss sie Vorhersagen für alle Energieskalen machen 
  können. Um eine auch für hohe Energieskalen $\mu$ gültige Theorie zu beschreiben muss demnach 
  der Wert $\lim\limits_{\mu\to\infty} g(\mu)$ existieren, ebenso muss der Wert $\lim\limits_{
  \mu\to 0}g(\mu)$ existieren, wenn die Theorie auch für niedrige Energieskalen gültig sein soll. 
  Damit die Grenzwerte existieren muss $\lim\nicefrac{\d }{\d \mu} g(\mu)
  =0$ sein, es sind also gerade die Fixpunkte der $\beta$-Funktion, die als Grenzwerte 
  in Frage kommen.
  
  Im Laufe der Untersuchung der $\beta$-Funktion haben sich die folgenden Bezeichnungen 
  entwickelt.
  \begin{description}
   \item[Gaußscher Fixpunkt: ] Ist der Punkt $g^*=0$ ein Fixpunkt der $\beta$-Funktion, so 
			      spricht man von einem Gaußschen Fixpunkt.
   \item[Banks-Zaks Fixpunkt: ] Ein Fixpunkt $g^*\neq 0$, der physikalsich sinnvoll und 
			      perturbativ ist, heißt Banks-Zaks oder 
			      Caswell-Banks-Zaks Fixpunkt.
   \item[Landau Pol: ] Besitzt die Lösung des Problems \eqref{eq:beta_allgemein:dgl} mit 
		      Anfangswert $g(\mu_0)=g_0$ eine Polstelle $\mu_\text{Pol}<\infty$, sodass 
		      $g(\mu)\overset{\mu\to\mu_\text{Pol}}{\longrightarrow}\infty$, dann spricht 
		      man von einem Landau-Pol.
  \end{description}
  
  \subsection{Vereinfachung des mathematischen Problems}
      Die Berechnung einer Trajetkorie als Lösung zum Anfangswertproblem 
      \eqref{eq:beta_allgemein:dgl} mit Anfangswert $g(\mu_0)=g_0$ 
      ist in der Regel analytisch nicht möglich. Durch einige einfache Schritte lässt sich das 
      Problem jedoch zunächst in eine einfacher zuhandhabende Form überführen und sich das 
      Verhalten in der Nähe eines Fixpunktes vorhersagen.
      
      In \cite{General_relativity} schlägt S. Weinberg die Einführung der dimensionslosen 
      Kopplungskonstanten
      \begin{equation}
       \bar{g}_i(\mu):= \mu^{-d_i} g_i(\mu)
      \end{equation}
      vor, wobei $d_i$ die Massendimension der Kopplungskonstanten $g_i$ ist. Bei der 
      Untersuchung der \QCDxdQCD-$\beta$-Funktion wird klar, dass die Erweiterung 
      \begin{equation}
       \alpha(\mu)_i:= \mathcal{N} \left(\mu^{-d_i} g_i(\mu)\right)^n
      \end{equation}
      den Grad $M_L$ der $\beta$-Funktion verringern kann und somit das Problem weiter 
      vereinfacht (vgl. \cite{Scale_of_dark_QCD}, \cite{Asymptotic_safety_guaranteed}). Dabei 
      dient $\mathcal{N}$ als Normierungskonstante, die insbesondere von dimensionslosen 
      Größen wie Teilchenzahlen oder Größen der Symmetriegruppe abhängen kann.
      \begin{beispiel}
	  Für ein eindimensionales Problem
	  \begin{equation}
	  \mu \frac{\d}{\d \mu} g(\mu) = X g(\mu)^3 + Y g(\mu)^5
	  \end{equation}
	  und für den einfachen Fall $[g]=0$ definiere 
	  \begin{equation}
	  \alpha(\mu) =\mathcal{N} g(\mu)^2 \quad .
	  \end{equation}
	  Es folgt
	\begin{equation}
	\frac{\d \alpha}{\d \mu}=\mathcal{N} 2 g \frac{\d g}{\d \mu} 
	\,\Rightarrow \, \frac{\d g}{\d \mu} g=\frac{1}{2\mathcal{N}} 
	\frac{\d \alpha}{\d \mu}
	\end{equation}
	und damit 
	\begin{align}
	 &&\frac{1}{2\mathcal{N}} \mu \frac{\d}{\d \mu} \alpha(\mu) &= 
	 X g(\mu)^4 +Yg(\mu)^6 &\\
	 &\Rightarrow&\frac{1}{2\mathcal{N}} \mu \frac{\d}{\d \mu} \alpha(\mu) &= 
	 \frac{X}{\mathcal{N}^2} \alpha(\mu)^2 + \frac{Y}{\mathcal{N}^3} \alpha(\mu)^3 &\\
	 &\Rightarrow& \mu \frac{\d}{\d \mu} \alpha(\mu) &=\tilde{X}\alpha(\mu)^2+\tilde{Y}
	 \alpha(\mu)^3 &
	\end{align}
	mit $\tilde{X}=2X\mathcal{N}^{-1}$ und $\tilde{Y}=2Y\mathcal{N}^{-2}$.
    \end{beispiel}
    Naheliegend wird wieder $\alpha=(\alpha_1,\ldots,\alpha_N)^\text{T}$ und 
    $\beta(\alpha)=\beta(g\circ\alpha)$ geschrieben.
    
    Der physikalisch sinnvolle Wertebereich für die Energieskala $\mu$ ist $(0,\infty)$. 
    Mit der Renormierungsgruppenzeit (RG-Zeit) $t$, definiert als
    \begin{equation}
     t(\mu):=\ln\left(\frac{\mu}{\Lambda}\right) \,
     \Leftrightarrow \, \mu(t):=e^t \quad ,
    \end{equation}
    gibt es eine Bijektion $(0,\infty)\overset{t}{\longrightarrow}(-\infty,\infty)$, die 
    es erlaubt die Kopplungskonstante als 
    \begin{equation}
    \tilde{\alpha}(t):=\alpha(e^t)=\alpha(\mu)
    \end{equation}
    zu schreiben. Der Parameter $\Lambda$ ist beliebig und hat keine physikalische Bedeutung, 
    er wird später lediglich die Extrapolation der Fixpunkte übersichtlicher gestalten. Es folgt
    \begin{equation}
     \mu\frac{\d}{\d\mu}\alpha(\mu) = \mu \underbrace{\frac{\d t}{\d \mu}}_{=\mu^{-1}}
     \frac{\d}{\d t} \tilde{\alpha}(t)
     =\frac{\d}{\d t} \tilde{\alpha}(t) \quad .
    \end{equation}
    Damit ist Gleichung \eqref{eq:beta_allgemein:dgl} äquivalent zu der autonomen 
    Differentialgleichung
    \begin{equation}
     \frac{\d}{\d t} \alpha(t)=\beta(\alpha) \quad , \label{eq:beta_allgemein:dgl_alpha}
    \end{equation}
    wobei $\tilde{\alpha}$ wieder zu $\alpha$ umbenannt wurde.
    
  \subsection{Verhalten in einer Umgebung eines Fixpunktes}
    Um das Verhalten der Kopplungskonstanten $\alpha(t)$ in der Nähe eines Fixpunktes zu 
    untersuchen wird die Stabilitätsmatrix wie folgt eingeführt.
    \begin{definition}
    Sei $\alpha^*$ ein Fixpunkt der $\beta$-Funktion im $\mathbb{R}^N$ und sei $\beta$ in 
    $\alpha^*$ zweimal stetig differenzierbar. Die Matrix
    \begin{equation}
     \frac{\partial \beta}{\partial \alpha}:= 
     \left( \frac{\partial \beta_i}{\partial \alpha_j} \right)_{1\leq i,j \leq N}
    \end{equation}
    heißt \textit{Stabilitätsmatrix der $\beta$-Funktion} \cite{General_relativity}. 
    Außgewertet am Punkt $\alpha^*$ 
    ist die Schreibweise $\left.\frac{\partial \beta}{\partial \alpha}\right|_{\alpha^*}$ oder 
    kurz $\dbdafix$.\\
    Ein Fixpunkt $\alpha^*$ heißt \textit{hyperbolisch}, wenn alle Eigenwerte von $\dbdafix$ 
    einen von Null verschiedenen Realteil besitzen \cite{Bronstein}.
    \end{definition}
    Der Zusammenhang zu der Stabilität des Fixpunktes ist folgendermaßen zu erkennen. 
    
    In der Nähe eines hyperbolischen Fixpunktes $\alpha^*$ kann Gleichung 
    \eqref{eq:beta_allgemein:dgl_alpha} durch ihre Linearisierung
    \begin{equation}
     \ddt \alpha(t) \simeq \dbdafix (\alpha(t)-\alpha^*) \label{eq:beta_allgemein:beta_linear}
    \end{equation}
     beschrieben werden. Mit den Eigenvektoren\footnote{Die Eigenvektoren 
     müssen gegebenenfalls zu einer Basis ergänzt werden.} $\{e_i\}$ zu den Eigenwerten 
     $\{\lambda_i\}$ und 
     \begin{equation}
      \alpha(t)-\alpha^*=:\sum_{i=1}^N K_i(t) e_i
     \end{equation}
     folgt  
    \begin{align}
     &&\ddt \left(\sum\limits_{i=1}^N K_i(t)e_i + \alpha^* \right) &=
     \dbdafix \sum\limits_{i=1}^N K_i(t) e_i = \sum\limits_{i=1}^N K_i(t) \lambda_i e_i &\\
     &\Rightarrow&  \ddt K_i(t) &= K_i(t) \lambda_i &
    \end{align}
    mit der Lösung
    \begin{equation}
     K_i(t)=\e^{\lambda_i t} K_i(0) \label{eq:beta_allgemein:K}
    \end{equation}
    für $K_i(0) \in \mathbb{R}$ klein. Die Lösung für $\alpha(t)$ lässt sich dann schreiben als 
    \begin{equation}
     \alpha(t)=\sum\limits_{i=1}^N \e^{\lambda_i t} K_i(0) e_i +\alpha^* \label{eq:beta_allgemein:loesung}
    \end{equation}
    Dieses Ergebnis ist unter anderem in \cite{Weinberg:1976}, \cite{General_relativity} und 
    \cite{Asymptotic_safety_guaranteed} zu sehen.
    Im linearisierten Problem werden nun drei eindeutige, lineare Unterräume unterschieden 
    \cite{Bronstein}.
    \begin{definition}
     \begin{enumerate}
      \item Die \textit{stabile Mannigfaltigkeit} besteht aus allen Punkten im Phasenraum, 
      die in den Fixpunkt hinein laufen.
      \item Die \textit{instabile Mannigfaltigkeit} besteht aus allen Punkten im Phasenraum, 
      die aus dem Fixpunkt heraus laufen.
      \item Die \textit{Zentrumsmannigfaltigkeit} $M_\text{z}$ besteht aus den restlichen 
      Punkten in der Fix\-punkt\-um\-ge\-bung.
      \end{enumerate}
    \end{definition}
    Aus Gleichung \eqref{eq:beta_allgemein:loesung} wird dann klar, dass eine Trajektorie in 
    $\Ms$ für $t\to\infty$ den Fixpunkt erreicht. Ein Eigenvektor mit positivem 
    Eigenwert wird auch als IR-attraktiv, mit einem negativen Eigenwert als IR-repulsiv 
    bezeichnet \cite{Weinberg:1976}. 
  \subsection{Sonderfall: nicht hyperbolischer Fixpunkte}
    Bei allgemeinen Betrachtungen (vgl. \cite{Weinberg:1976}) 
    wird der nicht hyperbolische Fall $\lambda_i = 0$ oft als unwichtiger Sonderfall nicht 
    weiter betrachtet. Bei der Untersuchung einer konkreten $\beta$-Funktion kommt dieser 
    Sonderfall aber auf natürliche Weise schnell zu stande.
    
    Eine dimensionslose Kopplungskonstante $\alpha_1:=\mathcal{N}g_1^2$ muss, damit die 
    physikalische Kopplungskonstante $g_1$ reell ist, positiv sein. Es muss demnach 
    $\beta_1(\alpha)\geq 0$ auf der gesamten $\alpha_1$-Achse gelten. Falls es nun einen 
    Fixpunkt $\alpha^*=(0,\alpha_2,\ldots,\alpha_N)^\text{T}$ und es einen Eigenvektor
    $e_1\in\Ms$ gibt\footnote{Dabei 
    soll $\Ms$ nicht nur aus der $\alpha_1$-Achse bestehen. Da $\Ms$ ein linearer Unterraum 
    ist, gibt es zumindest eine Linearkombination $\left(\sum_{i=1}^N c_ie_i\right)\in \Ms$, die 
    folgende Argumentation kann dann analog geführt werden.},
    dann folgt aus \eqref{eq:beta_allgemein:K}
    \begin{equation}
     \beta(K_1(t)e_1) \ddt K_1(t) = K_1(t) \lambda_1
    \end{equation}

  
    



      


  
  
  
  
