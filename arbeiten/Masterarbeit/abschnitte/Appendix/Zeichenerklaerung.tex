\clearpage
\section*{Zeichenerklärung}

\begin{itemize}
\item "`o.E."' für ohne Einschränkung.

\item "`$\simeq$"' für asymptotisch gleich.
	\begin{beispiel}
	 \begin{equation}
	  f(x)\simeq f(x_0)+f'(x_0)(x-x_0)
	 \end{equation}
	  Für eine Linearisierung von $f$ in der Umgebung von $x_0$. Für 
	  $
	  \left|(x-x_0)\right|\rightarrow 0
	  $
	  gilt \\ 
	  $
	  \left|f(x)-\left(f(x_0)+f'(x_0)(x-x_0)\right)\right|\rightarrow 0 
	  $
	  hinreichend schnell.
	\end{beispiel}

  \item "`$[]$"' für die Massendimension in natürlichen Einheiten.
  
  \item "`$\Big\langle \phi \Big\rangle$"' für den Vakuumerwartungswert eines Feldes $\phi$ 
	  unter einer angegebenen Wirkung.
  
  \item "`$\dim$"' für die Dimension eines Vektorraums oder einer Mannigfaltigkeit.

	
  \item "`$\text{diag}(a_1,a_2,\ldots)$"' für eine Diagonalmatrix mit Einträgen $a_1,a_2,\ldots$
% 	\begin{equation}
% 	 \begin{pmatrix}
% 	  a_1 & 0   &  0& \dots \\
% 	  0   & a_2 & 0 & \dots \\
% 	  0   &  0  &\ddots & \\
% 	  \vdots & \vdots
% 	 \end{pmatrix} \quad .
% 	\end{equation}


  \item "`$\textbf{T}$"' für das zeitgeordnete Produkt.
  
  \item "`$\e$"' für die Exponentialfunktion, insb. auch Matrix- und Operatorexponentiale.
  
  \item "`$\mathcal{D}\phi$"' für das Pfadintegralmaß von Feldern $\phi$.
      
 
 
\end{itemize}
