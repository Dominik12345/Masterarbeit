\clearpage
\section{Einleitung}
  Die Entdeckung des Higgs-Bosons im Jahr 2012 \cite{Higgs} und die erstmalige 
  Beobachtung von Gravitationswellen im Jahr 2016 \cite{gravitational_waves} 
  sind weitere Bestätigung zweier fundamentaler Theorien der heutigen 
  Physik, des \textit{Standardmodells der Teilchenphysik (SM)} und der 
  \textit{allgemeinen Relativitätstheorie (ART)}, die es vermögen, Phänomene 
  auf Größenordnungen der Quantenphysik beziehungsweise der Kosmologie mit 
  großer Genauigkeit zu beschreiben und zu erklären. Obwohl das SM durch die 
  Entdeckung des Higgs in sich geschlossen ist, gibt es Hinweise sowohl von 
  experimenteller als auch von theoretischer Seite, dass dieses Theoriegebäude 
  nicht endgültig sein kann. 
  
  Kosmologische Experimente zeigen, dass es eine Art der Materie, 
  \textit{dunke Materie (DM)}, mit einer Energiedichte von 
  $\Omega_\text{DM}\approx 0,2$ in Einheiten der kritischen Massendichte 
  im Universum geben muss \cite{PDG:DM}, die eine schwache Ankopplung an die 
  SM-Materie haben muss, jedoch gravitativ wechselwirkt. Im Rahmen von 
  $\Lambda$CDM-Theorien wird eine kalte, d.h. nichtrelativistische, DM 
  vorhergesagt, sodass Teilchentheorien mit schweren, schwach ans SM 
  gekoppelten Elementarteilchen\footnote{Sog. WIMPs, weakly interacting massive 
  particles} naheliegend sind. Y. Bai und P. Schwaller schlagen dagegen vor, 
  die DM Masse analog zur Masse der bekannten baryonischen Materie zu erzeugen.  
  Die Masse der normalen Materie, d.h. der Protonen und Neutronen, wird im 
  wesentlichen durch Quark und Gluon Wechselwirkungen der 
  \textit{Quantenchromodynamik (QCD)} erzeugt. Eine analoge Dynamik, genannt 
  \textit{dark QCD (dQCD)}, zusammen mit einem Mechanismus, der die 
  Baryogenese in den dunklen Sektor erweitert, kann damit eine alternative 
  Erklärung für DM sein. \cite{Scale_of_dark_QCD}
  
  Von theoretischer Seite aus sind SM und ART völlig disjunkt, in dem Sinne, 
  dass das SM keine gravitativen Wechselwirkungen enthält und die ART eine 
  klassische, d.h. nicht quantisierte, Theorie ist. Um die in der ART 
  auftretenden Singulatritäten konsistent beschreiben zu können, ist eine 
  umfassendere Theorie der Gravitation auf quantenphysikalischen Skalen nötig 
  \cite{GR_Introductory}. Da es sich in beiden 
  Fällen um eine Feldtheorie handelt, ist es naheliegend, die Wirkung der ART 
  analog zum SM über den Pfadintegralformalismus nach Feynman zu quantisieren 
  \cite{GR_Hawking}. Praktisch führt dieses Vorgehen jedoch zu 
  nichtrenormierbaren Theorien \cite{GR_Weinberg}, die in der Regel unendlich 
  viele Renormierungskonstanten erfordern um auf allen Energieskalen gültig 
  zu sein und die deshalb häufig als unphysikalisch oder ungeeignet als 
  fundamentale Theorie gelten. Durch ein \textit{asymptotic safety (AS)} 
  Szenario ist es jedoch möglich, die Theorie bis auf eine endliche Zahl von 
  Parametern zu bestimmen, sodass AS als Erweiterung der üblichen Forderung 
  nach Renormierbarkeit verstanden werden kann \cite{GR_Weinberg}
  \cite{Weinberg:1976}. Hierbei 
  wird die Energieskalenabhängigkeit der Kopplungskonstanten einer Theorie 
  mit Hilfe von \textit{Renormierungsgruppen (RG)} untersucht; wenn 
  die Kopplungskonstanen auf bestimmten Hyperfläche im Phasenraum der 
  Kopplungskonstanten liegen ist das Energieskalenverhalten bis auf den  
  Freiheitsgrad der Dimension dieser Hyperfläche bestimmt.
  
 
    