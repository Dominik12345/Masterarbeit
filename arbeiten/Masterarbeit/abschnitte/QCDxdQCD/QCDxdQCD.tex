\clearpage
\section{QCD $\times$ dark QCD}
  
  Das Verhalten der QCD-Kopplungskonstanten ist im Standardmodell allein nicht 
  im Stande ein asymptotic safety Szenario zu entwickeln, wie in Abschnitt 
  \ref{beta_im_SM} gezeigt wird. Durch die 
  Erweiterung der QCD um eine weitere Eichgruppe, die dark QCD, ergeben sich 
  qualitativ völlig neue Möglichkeiten im Hochenergieverhalten der 
  Kopplungskonstanten. 
  
  \subsection{Die Standardmodell QCD}
    Im Standardmodell wird die QCD durch die Symmetriegruppe $SU(\Nc)$ 
    dargestellt, unter der sich die Quarks in der 
    \textit{fundamentalen Darstellung} und die Gluonen in der 
    \textit{adjungierten Darstellung} transformieren. Im SM gibt es 
    $N_\text{Flavour}=6$
    verschiedene Quarkflavour und $N_\text{Colour}=3$ Colours, da die QCD 
    Flavour-Blind ist, d.h. da die Wechselwirkung unabhängig von der 
    Flavour-Quantenzahl ist, ist die 
    Erweiterung auf $\Nc$ Colour und $\nfc$ Flavour jedoch trivial. Die 
    Lagrangedichte kann als 
    \begin{equation}
     \mathcal{L}_\text{QCD} =\bar{\psi}_i^f \left( 
     \i \gamma^\mu \partial_\mu \delta_{ij} 
     -g_1 \gamma^\mu t^A_{ij} \mathcal{A}_\mu^A
     -m_f \delta_{ij}
     \right)\psi_j^f -\frac{1}{4} F_{\mu \nu}^A F^{A \,\,\, \mu\nu}
     \label{eq:QCDxdQCD:L_QCD}
    \end{equation}
    geschrieben werden \cite{PDG:QCD}. Dabei stellt $\psi_b^f$ ein Quarkfeld mit 
    Colour $i$ und Flavour $f\in\{1,2,\ldots,\nfc\}$ 
    und mit der Masse 
    $m_f$ dar, $g_1$ ist die Kopplungskonstante der QCD. Die Indizes $i,j\in
    \{1,2,\ldots \Nc\}$ gehören zur fundamentalen Darstellung, während $A\in 
    \{1,2,\dots,\Nc^2-1\}$ zur adjungierten Darstellung gehört. Man 
    nennt $\psi=(\psi_1,\psi_2,\ldots,\psi_{\Nc})^\text{T}$ ein 
    Colour-Multiplett\footnote{Der Flavour-Index $f$ wird ab jetzt 
    weggelassen.} unter der fundamentalen Darstellung, wenn es unter 
    Anwendung der $SU(\Nc)$ gemäß
    \begin{equation}
      \psi(x) \longrightarrow \underbrace{U(x)}_{\in \mathbb{C}^{\Nc\times\Nc}}
      \psi(x) \quad ,\quad \bar{\psi}(x) \longrightarrow\bar{\psi}(x) 
      U(x)^\dagger
    \end{equation}
    transformiert. Die Gluonfelder $\mathcal{A}_\mu^A$ mit Erzeugern 
    $t^A\in \mathbb{C}^{\Nc\times\Nc}$ 
    transformieren dagegen in der 
    adjungierten Darstellung \cite{Weinberg:QFT_2}
    \begin{equation}
      t^A_{ij}\mathcal{A}_\mu^A(x) \longrightarrow
      U(x) t^A_{ij}\mathcal{A}_\mu^C(x) U(x)^\dagger - \i (\partial_\mu U(x) )
      U(x)^\dagger \quad .
    \end{equation}
    Die Dynamik und Propagation der Gluonen wird dabei durch den 
    Feldstärketensor vermittelt,
    \begin{equation}
      F^A_{\mu\nu} = \partial_\mu \mathcal{A}_\nu^A-\partial_\nu 
      \mathcal{A}_\mu^A - g f^{ABC} \mathcal{A}_\mu^B \mathcal{A}_\nu^C 
      \quad , \quad [t^A,t^B]=\i f^{ABC} t^C  \quad .
      \label{eq:QCDxdQCD:Feldstaerketensor}
    \end{equation}
    Versucht man die QCD im Feynmanformalismus zu quantisieren, werden außerdem 
    Eichfixierung und Faddeev-Popov-de-Witt Ghosts benötigt  
    \cite{Weinberg:QFT_2}, diese können jedoch nach der Modellbildung 
    hinzugefügt werden, sodass sie hier nicht auftauchen. 
    
    %DIAGRAMME
    
  \subsection{Dark QCD}
    In \cite{Scale_of_dark_QCD} wird die dQCD eingeführt, um die DM Massendichte 
    $\Omega_\text{DM}$ im Universum zu erklären. Dazu wird das 
    Niederenergieverhalten der neu eingeführten Kopplungskonstanten 
    $g_2$ auf die Confinement Scale $\Lambda_\text{dQCD}$ untersucht, 
    die analog zur QCD Confinement Scale $\Lambda_\text{QCD}$ die Größenordnung 
    der Baryonmasse bestimmt. Die Theorie ist aber, ebenfalls analog zur QCD, 
    auch bis zu beliebig hohen Energieskalen anwendbar und soll hier auf ihr 
    Hochenergieverhalten untersucht werden. 
    
    Die Lagrangedichte der dQCD kann analog zu \eqref{eq:QCDxdQCD:L_QCD} als
    \begin{equation}
     \mathcal{L}_\text{dQCD} = \bar{\xi}_r^f \left( 
     \i \gamma^\mu \partial_\mu \delta_{rs} 
     -g_2 \gamma^\mu \widetilde{t}^M_{ts} \widetilde{\mathcal{A}}_\mu^M
     -\widetilde{m}_f \delta_{rs}
     \right)\xi_s^f -\frac{1}{4} \widetilde{F}_{\mu \nu}^M 
     \widetilde{F}^{M \,\,\, \mu\nu}
     \label{eq:QCDxdQCD:L_dQCD}
    \end{equation}
    definiert werden. Die Fermionen $\xi^f$ der dQCD sollen in der fundamentalen 
    Darstellung der $SU(\Nd)$ sein, entsprechend laufen die Indizes 
    $r,s\in\{1,2,\ldots,\Nd\}$ und $f\in\{ 1,2,\ldots,\nfd \}$, während die 
    Felder der Eichbosonen $\widetilde{\mathcal{A}}^M_\mu$ wieder in der 
    adjungierten Darstellung transformieren, $M\in\{1,2,\ldots,\Nd^2-1\}$, 
    der Feldstärketensor 
    $\widetilde{F}^M_{\mu \nu}$ wird wie in 
    \eqref{eq:QCDxdQCD:Feldstaerketensor} definiert. Wie sich in Abschnitt 
    \ref{beta_QCDxdQCD} zeigen wird, begünstigt eine hohe Teilchenzahl 
    die auftretenden Fixpunkte dahingehend, dass sie betragsmäßig kleiner 
    werden und somit unter perturbative Kontrolle kommen, ohne ihr 
    UV-attraktives Verhalten zu verlieren, wie es bei einer Theorie mit 
    ausschließlich Fermionen der Fall wäre. Skalare $\phi^f$ und $\chi^f$ 
    unter der QCD bzw. dQCD können über 
    \begin{align}
     \mathcal{L}_\text{QCD}^\text{S} &=
     \left( \left[ \i \delta_{ji} \partial_\mu - 
     g_1 t^A_{ji} \mathcal{A}^A_\mu -\frac{1}{2}\left(m_f^\text{S}\right)^2 
     \delta_{ji}
     \right] \phi^f_i \right)^\dagger
     \left( \left[ \i \delta_{ji} \partial^\mu- 
     g_1 t^A_{ji} \mathcal{A}^{A\,\,\, \mu} -\frac{1}{2}\left(m_f^\text{S}
     \right)^2
     \delta_{ji}
     \right] \phi^f_i \right) \\
     \mathcal{L}_\text{dQCD}^\text{S} &=
     \left( \left[ \i \delta_{sr}\partial_\mu - 
     g_2 \widetilde{t}^M_{sr} \widetilde{\mathcal{A}}^M_\mu -\frac{1}{2}
     \left(\widetilde{m}_f^\text{S}\right)^2 \delta_{sr}
     \right] \chi^f_r \right)^\dagger
     \left( \left[ \i \delta_{sr}\partial^\mu - 
     g_2 \widetilde{t}^M_{sr} \widetilde{\mathcal{A}}^{M\,\,\, \mu} -
     \frac{1}{2}\left(\widetilde{m}_f^\text{S}\right)^2 \delta_{sr}
     \right] \chi^f_r \right)
    \end{align}
    eingeführt werden (vgl. \cite{Scalar_QCD}). Der Flavour-Index $f$ läuft 
    wieder 
    über alle $\nsc$ QCD Skalare bzw. alle $\nsd$ dQCD Skalare. Die 
    Colour-Indizes $i,j$ und $r,s$ über alle $\Nc$ Colours bzw. $\Nd$ dark 
    Colours.

    Bis zu diesem Punkt sind die QCD und dQCD voneinander unabhängig, in dem 
    Sinne, dass es keine zusammenhängenden Feynmangraphen gibt, in denen 
    Teilchen des QCD- und dQCD-Sektors gleichzeitig auftreten. Um dies zu 
    ermöglichen wird ein joint-Sektor eingeführt, Teilchen die unter beiden 
    Symmetriegruppen geladen sind und somit an $\mathcal{A}_\mu^C$ und 
    $\widetilde{\mathcal{A}}_\mu^C$ mit der entsprechenden Kopplungskonstanten 
    koppeln \cite{Scale_of_dark_QCD}. Unter der Annahme, dass sie ebenfalls 
    in den fundamentalen Darstellungen der Symmetriegruppen transformieren, 
    können die entsprechenden Teile der Lagrangedichte als 
    \begin{align}
      \mathcal{L}_\text{joint} &= \bar{\zeta}_{ir}^f \left( 
     \i \gamma^\mu \partial_\mu \delta_{ij} \delta_{rs} 
     -g_1 \gamma^\mu t^A_{ij} \mathcal{A}_\mu^A \delta_{rs}
     -g_2 \gamma^\mu \widetilde{t}^M_{rs} \widetilde{\mathcal{A}}_\mu^M 
     \delta_{ij}
     -m^\text{joint}_f \delta_{ij}\delta_{rs}
     \right)\zeta_{js}^f 
     \\
     \mathcal{L}_\text{joint}^\text{S}&=
     \left( \left[ \i \delta_{ji}\delta_{sr}\partial_\mu- 
     g_1 t^A_{ji} \mathcal{A}^A_\mu \delta_{sr} 
     - 
     g_2 \widetilde{t}^M_{sr} \widetilde{\mathcal{A}}^M_\mu \delta_{ji} 
     -\frac{1}{2}\left(m_f^\text{S,joint}\right)^2 \delta_{ji}\delta_{sr}
     \right] \eta^f_{ir} \right)^\dagger
     \\ &
     \left( \left[ \i \delta_{ji}\delta_{sr}\partial_\mu- 
     g_1 t^A_{ji} \mathcal{A}^A_\mu \delta_{sr} 
     - 
     g_2 \widetilde{t}^M_{sr} \widetilde{\mathcal{A}}^M_\mu \delta_{ji} 
     -\frac{1}{2}\left(m_f^\text{S,joint}\right)^2 \delta_{ji}\delta_{sr}
     \right] \eta^f_{ir} \right)    
     \label{eq:QCDxdQCD:joint}
    \end{align}
    geschrieben werden. Dabei laufen die Indizes $a,b,j\in \{1,2,\ldots 
    \Nc\}$, $c,d,i\in\{1,2,\ldots,\Nd\}$ und $f\in\{1,2, \ldots \nfc\}$ für 
    die Fermionen $\zeta$ bzw. $f\in\{1,2,\ldots,\nsc\}$ für Skalare $\eta$.

    In \cite{Weinberg:QFT_2} zeigt Weinberg, dass durch die Reskalierung zu 
    einer Energieskala $\mu$ alle Massen $m$ gemäß $m/\mu$ skalieren und 
    im Grenzwert hoher Energien gegen Null gehen. Entsprechend ist es bei der 
    Untersuchung von UV-Fixpunkten zulässig, alle Massen gleich Null zu setzen.
  
    Die bisher beschriebenen Lagrangedichten beinhalten ausschließlich 
    Eichwechslewirkungen, es gibt jedoch weitere Arten von Wechselwirkungen 
    bei denen keine Eichfelder beteiligt sind, 3-Skalar-Kopplungen, 
    4-Skalar-Kopplungen und Yukawa-Kopplungen. Im Folgenden wird kurz begündet, 
    warum diese drei Arten der Kopplung im weiteren Verlauf der Arbeit nicht 
    beitragen.
    \begin{description}
     \item[4-Skalar-Kopplung:]
    Korrekturen durch 
    4-Skalar-Kopplungen tragen erst ab 3-Schleifen Ordnung zur $\beta$-Funktion 
    der Eichkopplungen bei \cite{quartic_scalar}, da in dieser Arbeit jedoch 
    nur 2-Schleifen $\beta$-Funktionen betrachtet werden, gibt es keine 
    Beiträge. 
      \item[Yukawa-Kopplung:]
    Bei reinen QCD-, dQCD- oder joint-Sektor Yukawa-Kopplungen, als 
    Lagrangedichte
    \begin{equation}
     \mathcal{L}_\text{Yukawa}= y_1 \bar{\psi} \phi \psi +
     +y_2 \bar{\xi} \chi \xi + y_3 \bar{\zeta} \eta \zeta 
     \quad ,
    \end{equation}
    müssen die Darstellungen $\bar{N}\otimes N \otimes N$ ein Singlett der 
    $SU(N)$ ergeben\footnote{Für $N$ ist entsprechend $\Nc$ oder $\Nd$ 
    einzusetzen.}, eine Berechnung über Yang-Tableaus zeigt jedoch, dass das 
    Ergebnis mindestens ein $N$-Tupel der $SU(N)$ ist. Solche 
    Yukawa-Kopplungen sind demnach nicht möglich.
    
    Da der joint-Sektor die einzige Verbindung zwischen der $SU(\Nc)$ und 
    $SU(\Nd)$ ist, müssen alle weiteren möglichen Yukawa-Kopplungen je ein 
    Teilchen aus jedem Sektor enthalten. Die Lagrangedichte ist 
    \begin{equation}
     \mathcal{L}_\text{Yukawa}=y_1 \bar{\psi} \eta \xi + 
     y_2 \bar{\psi} \zeta \chi
    \end{equation}

    
    
    
    \end{description}
    
    
    
    
  Aufgrund\cite{Ade:2015xua}