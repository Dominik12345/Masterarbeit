\clearpage
\section{QCD $\times$ dark QCD}
  
  Das Verhalten der QCD-Kopplungskonstanten ist im Standardmodell allein nicht 
  im Stande ein asymptotic safety Szenario zu entwickeln, wie in Abschnitt 
  \ref{Laufende_Kopplungen_im_Standardmodell} gezeigt wird. Durch die 
  Erweiterung der QCD um eine weitere Eichgruppe, die dark QCD, ergeben sich 
  qualitativ völlig neue Möglichkeiten im Hochenergieverhalten der 
  Kopplungskonstanten. 
  
  \subsection{Die Standardmodell QCD}
    Im Standardmodell wird die QCD durch die Symmetriegruppe $SU(\Nc)$ 
    dargestellt, unter der sich die Quarks in der 
    \textit{fundamentalen Darstellung} und die Gluonen in der 
    \textit{adjungierten Darstellung} transformieren. Im SM gibt es 
    $N_\text{Flavour}=6$
    verschiedene Quarkflavour und $N_\text{Colour}=\Nc=3$ Colours, da die QCD 
    Flavour-Blind ist, d.h. da die Wechselwirkung unabhängig von der 
    Flavour-Quantenzahl ist, ist die 
    Erweiterung auf $\Nc$ Colour und $\nfc$ Flavour jedoch trivial. Die 
    Lagrangedichte kann als 
    \begin{equation}
     \mathcal{L}_\text{QCD} = \sum\limits_f \bar{q}_a^f \left( 
     \i \gamma^\mu \partial_\mu \delta_{ab} 
     -g \gamma^\mu t^C_{ab} \mathcal{A}_\mu^C
     -m_f \delta_{ab}
     \right)q_b^f -\frac{1}{4} F_{\mu \nu}^A F^{A \,\,\, \mu\nu}
    \end{equation}
    geschrieben werden \cite{PDG:QCD}. Dabei stellt $q_b^f$ ein Quarkfeld mit 
    Colour $b\in \{1,2,\ldots,\Nc\}$ und Flavour $f\in\{1,2,\ldots , 
    N_\text{Flavour}\}$ 
    und mit der Masse 
    $m_f$ dar. Man nennt $q=(q_1,q_2,\ldots,q_{\Nc})^\text{T}$ ein 
    Colour-Multiplett\footnote{Der Flavour-Index $f$ wird ab jetzt 
    weggelassen.} unter der fundamentalen Darstellung, wenn es unter 
    Anwendung der $SU(\Nc)$ gemäß
    \begin{equation}
      q(x) \longrightarrow \underbrace{U(x)}_{\in \mathbb{C}^{\Nc\times\Nc}}
      q(x) \quad ,\quad \bar{q}(x) \longrightarrow\bar{q}(x) U(x)^\dagger
    \end{equation}
    transformiert. Die Gluonfelder $\mathcal{A}_\mu^C$ mit $C\in\{1,2,\ldots,
    \Nc^2-1\}$ und Erzeugern $t^C\in \mathbb{C}^{\Nc\times\Nc}$ 
    transformieren dagegen in der 
    adjungierten Darstellung \cite{Weinberg:QFT_2}
    \begin{equation}
      t^C_{ab}\mathcal{A}_\mu^C(x) \longrightarrow
      U(x) t^C_{ab}\mathcal{A}_\mu^C(x) U(x)^\dagger - \i (\partial_\mu U(x) )
      U(x)^\dagger \quad .
    \end{equation}
    Die Dynamik und Propagation der Gluonen wird dabei durch den 
    Feldstärketensor vermittelt,
    \begin{equation}
      F^A_{\mu\nu} = \partial_\mu \mathcal{A}_\nu^A-\partial_\nu 
      \mathcal{A}_\mu^A - g f_{ABC} \mathcal{A}_\mu^B \mathcal{A}_\nu^C 
      \quad , \quad [t^A,t^B]=\i f_{ABC} t^C  \quad .
    \end{equation}
    Versucht man die QCD im Feynmanformalismus zu quantisieren, werden außerdem 
    Eichfixierung und Faddeev-Popov-de-Witt Ghosts benötigt  
    \cite{Weinberg:QFT_2}, diese können jedoch nach der Modellbildung 
    hinzugefügt werden, sodass sie hier nicht auftauchen. 
    
    
  \subsection{Dark QCD}
    In \cite{Scale_of_dark_QCD} wird die dQCD eingeführt, um die DM Massendichte 
    $\Omega_\text{DM}$ im Universum zu erklären. Dazu wird das 
    Niederenergieverhalten der neu eingeführten Kopplungskonstanten 
    $g_\text{dQCD}$ auf die Confinement Scale $\Lambda_\text{dQCD}$ untersucht, 
    die analog zur QCD Confinement Scale $\Lambda_\text{QCD}$ die Größenordnung 
    der Baryonmasse bestimmt. Die Theorie ist aber, ebenfalls analog zur QCD, 
    auch bis zu beliebig hohen Energieskalen anwendbar.
    
    

    
  
  
  
  Aufgrund\cite{Ade:2015xua}