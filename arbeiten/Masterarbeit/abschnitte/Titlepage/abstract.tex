\cleardoublepage
\section*{Zusammenfassung}

In dieser Arbeit wird eine $SU(\Nc)\times SU(\Nd)$-Eichtheorie als Erweiterung 
der asymptotisch freien Quantenchromodynamik des Standardmodells der Teilchenphysik auf mögliche 
asymptotic safety Szenarien bei hohen Energien untersucht. 
Nach einer Einführung in das Standardmodell werden Beiträge zur 
$\beta$-Funktion der Eichtheorie durch Eichwechselwirkungen und 
Yukawa-Kopplungen diskutiert,
danach wird die Theorie mathematisch formuliert. Es werden allgemeine 
Eigenschaften einer $\beta$-Funktion untersucht und Stabilitätskriterien für 
Fixpunkte der reinen Eichtheorie abgeleitet. Durch das Anwenden dieser 
Kriterien werden 
Obergrenzen für die Anzahl neuer Teilchen und für die Dimension $\Nd$ der 
dunklen Eichgruppe 
gefunden, unter denen asymptotic safety möglich ist. Außerdem wird 
festgestellt, dass eine hohe Teilchenzahl im joint-Sektor förderlich 
für die Perturbativität der Fixpunkte ist.
Abschließend werden die vier gefundenen asymptotic safety Szenarien mit 
Standardmodellvorhersagen verglichen und festgestellt, dass sich zwei Szenarien 
ähnlich dem Standardmodell, die anderen zwei Szenarien jedoch qualitativ 
anders verhalten.

\vspace{2cm}
\section*{Abstract}

In this thesis I will investigate a $SU(\Nc)\times SU(\Nd)$ gauge theory for 
the potential of evolving UV fixed points and asymptotically safe running 
couplings as a modification to the asymptotically free quantum chromodynamics of the 
standardmodel. After a brief introduction to the standard model of particle physics 
I will discuss the contributions of gauge couplings and Yukawa couplings to 
the gauge coupling $\beta$-function. I will derive 
criteria for stable UV fixed points and apply these to the pure gauge theory 
to deduce upper limits for the number of new particles and for the group 
parameter $\Nd$. It will turn out that a maximum number of join-particles 
leads to perturbatively controlled fixed points. 
Four possible scenarios of asymptotic safety will emerge, two of which lead 
to running couplings that 
behave like the $SU(\Nc)$ gauge theory of the standardmodel, while the 
remaining two differ qualitatively from the standard model predictions.