\clearpage
\section{UV-Fixpunkte der $SU_\text{QCD}\times SU_\text{dQCD}$}

  Die allgemeinste Form der $\beta$-Funktion auf 2-loop Ordnung wurde von 
  D.R.T. Jones berechnet \cite{Jones}.
  Die $\beta$-Funktion hat die Form
  \begin{equation}
   \beta (g) = \begin{pmatrix}
                     X_1^g g_1^3 + Y_1^g g_1^5 + Z_1^g g_1^3 g_2^2 \\ 
                     X_2^g g_2^3 + Y_2^g g_2^5 + Z_2^g g_2^3 g_1^2 
                    \end{pmatrix}\quad . \label{eq:beta_QCDxdQCD:beta_g}
  \end{equation}
  Für die Darstellungen $R_1$, $R_2$, $S_1$ und $S_2$ der Fermionen bzw. Skalare 
  sind die Koeffizienten von $\beta_1$ gegeben durch 
  \begin{align}
   X_1^g &= (16 \uppi^2)^{-1}\left[ \frac{2}{3} T(R_1) d(R_2) + \frac{1}{3} 
    T(S_1)d(S_2)-\frac{11}{3} C_2(G_1) \right] \\
   Y_1^g &= (16 \uppi^2)^{-2} \left[ 
    \left( 
    \frac{10}{3} C_2(G_1)+2C_2(R_1)
    \right) T(R_1) d(R_2) \right. \\
     & \quad \quad \quad \quad \quad + \left. \left(
    \frac{2}{3} C_2(G_1) +4C_2(S_1) 
    \right)T(S_1) d(S_2)
    -\frac{34}{3} C_2(G_1)^2
    \right] \\
   Z_1^g &= (16 \uppi^2)^{-2} \left[
      2 C_2(R_2) d(R_2) T(R_1) +4C_2(S_2)d(S_2) T(S_1)
    \right] \quad .
  \end{align}
  Die Berechnung ist dabei für chirale Fermionen geschehen, was einen 
  zusätzlichen Faktor von $2$ in $d(R_1)$ und $d(R_2)$ bewirkt, wenn man sonst 
  Dirac-Fermionen betrachtet. Außerdem wurde bei der Berechnung dieselben 
  Darstellungen $R_1$ und $R_2$ bzw. $S_1$ und $S_2$ bezüglich $G_1$ und $G_2$ 
  für alle Fermionen bzw. Skalare angenommen. Wenn Teilchen des QCD-, dQCD- und 
  joint-Sektors verschiedene Darstellungen besitzen, müssen $d(R_1)$, $d(R_2)$, 
  $d(S_1)$ und $d(S_2)$ angepasst werden. Am einfachsten und anschaulichsten 
  geschiet dies über das Zeichnen von Feynmandiagrammen und abzählen der 
  möglichen Teilchen, die im entsprechenden Diagramm erlaubt sind.

%   
%  DIAGRAMME
%   
  
  Die ermittelten Koeffizienten stimmen mit denen in \cite{Scale_of_dark_QCD} 
  überein. Mit $\Nfc := \nfc +\Nd \nfj$, $\Nsc := \nsc +\Nd \nsj$, 
  $\Nfd := \nfd +\Nc \nfj$ und $\Nsd := \nsd +\Nc \nfj$ lassen sie sich 
  schreiben als
  \begin{align}
   X_1^g &= (16 \uppi^2)^{-1} \left[
    \frac{2}{3} \Nfc + \frac{1}{6} \Nsc - \frac{11}{3} \Nc \right] 
    \label{eq:beta_QCDxdQCD:X1g} \\ 
   Y_1^g &= (16 \uppi^2)^{-2} \left[ \left(\frac{13}{3}\Nc-\frac{1}{\Nc}\right)
    \Nfc + \left( \frac{4}{3} \Nc -\frac{1}{\Nc}\right)\Nsc -\frac{34}{3}
    \Nc^2\right] \label{eq:beta_QCDxdQCD:Y1g}\\
   Z_1^g &= (16 \uppi^2)^{-2}\left[(\Nd^2-1)(\nfj + \nsj) \right] \quad .
   \label{eq:beta_QCDxdQCD:Z1g}
  \end{align}

  Da beide Kopplungskonstanten in einer $4$-dimensionalen Raumzeit die 
  Massendimension $[g_1]=[g_2]=0$ besitzen, werden die neuen 
  Kopplungskonstanten $\alpha_i := \frac{g_i^2}{4 \uppi}$ 
  eingeführt. Mit $X_i := 8\uppi X_i^g $, $Y_i := 32\uppi^2 Y_i^g $ und 
  $Z_i := 32\uppi^2 Z_i^g $ folgt
  \begin{equation}
   \beta (g) = \begin{pmatrix}
                     X_1 \alpha_1^2 + Y_1 \alpha_1^3 + Z_1 \alpha_1^2 \alpha_2\\ 
                     X_2 \alpha_2^2 + Y_2 \alpha_2^3 + Z_2 \alpha_1 \alpha_2^2 
                    \end{pmatrix} \label{eq:beta_QCDxdQCD:beta_alpha}
  \end{equation}
  und als Nullstellen findet man 
  \begin{itemize}
    \item den Gaußschen Fixpunkt $\alpha^{*1}=(0,0)$,
    \item den teilweise wechselwirkenden Fixpunkt 
	$\alpha^{*2}=\left(0,-\frac{X_2}{Y_2}\right)$ , falls $Y_2\neq 0$,
    \item den teilweise wechselwirkenden Fixpunkt 
	$\alpha^{*3}=\left(-\frac{X_1}{Y_1},0\right)$ , falls $Y_1\neq 0$,
    \item den vollständig wechselwirkenden Fixpunkt 
	$\alpha^{*4}=\left(\frac{Z_1X_2-X_1Y_2}{Y_1Y_2-Z_1Z_2},
	\frac{X_1Z_2-Y_1X_2}{Y_1Y_2-Z_1Z_2}\right)$ .
  \end{itemize}
  An den Fixpunkten gilt außerdem 
  \begin{equation}
    \left. \Sp \right|_*=(\alpha_1^*)^2 Y_1+(\alpha_2^*)^2 Y_2
    \quad
    \text{sowie}
    \quad
    \left. \Det \right|_*=(\alpha_1^*\alpha_2^*)^2(Y_1Y_2-Z_1Z_2) \quad.
    \label{eq:beta_QCDxdQCD:spur_determinante}
  \end{equation}

  
  \subsection{UV-Verhalten bei $\alpha^{*4}$}
    \subsubsection{attraktiver Fixpunkt}
      Für komplett UV-attraktives Verhalten muss die Bedingung 
      \begin{equation}
      \alpha_1^* > 0 \quad \land \quad
      \alpha_2^* > 0 \quad \land \quad
      \left. \Det \right |_* > 0 \quad \land \quad 
      \left. \Sp  \right |_*  < 0 \label{eq:beta_QCDxdQCD:alpha4}
      \end{equation}
      erfüllt sein. Es folgt 
      \begin{itemize}
      \item $Y_1>0 \land Y_2>0 \Rightarrow \blitz$ zu 
	\eqref{eq:beta_QCDxdQCD:alpha4}
      \item $Y_1<0 \land Y_2<0 \Rightarrow X_1>0 \lor X_2>0$
      \item $Y_1$ und $Y_2$ haben verschiedene Vorzeichen $\Rightarrow$ 
	$Z_1$ und $Z_2$ haben verschiedene Vorzeichen, d.h. genau ein $Z_i$ 
	ist negativ.
      \end{itemize}
      Ohne Skalare, d.h. für $\Nsc=\Nsd=\nsj=0$ ist dies nicht möglich
      \footnote{Hier nur für die Koeffizienten $X_1$, $Y_1$ und $Z_1$ gezeigt, 
      für $1\leftrightarrow 2$ müssen nur $\text{c} \leftrightarrow \text{d}$ 
      getauscht werden.}:
      \begin{enumerate}
       \item $Z_1<0$ ist nicht möglich (vgl. \eqref{eq:beta_QCDxdQCD:Z1g}),
       \item Für $X_1>0 \land Y_1<0$ müsste
        \begin{align}
	 &X_1>0 \overset{\eqref{eq:beta_QCDxdQCD:X1g}}{\Rightarrow}
	\Nfc>\frac{11}{2}\Nc \quad \land \quad 
	Y_1<0 \overset{\eqref{eq:beta_QCDxdQCD:Y1g}}{\Rightarrow} 
	\Nfc < \frac{34}{13-\frac{3}{\Nc^2}} \Nc \\
	 &\Rightarrow \frac{11}{2}  < \frac{34}{13-\frac{3}{\Nc^2}} \quad 
	 \blitz
	\end{align}
      \end{enumerate}
      Das Einführen von Skalaren begünstigt $X_1 > 0$, erfordert aber 
      $\Nc = 1$ oder $\Nd = 1$. Da die $SU(1)$ die Multiplikation mit Eins ist, 
      ist dieser Fall uninteressant.
      \begin{enumerate}
       \item $Z_1<0$ ist auch mit Skalaren nicht möglich ,
       \item Aus $X_1>0 \land Y_1<0$ mit Skalaren folgt
        \begin{align}
	 \Nfc &> \frac{11}{2} \Nc -\frac{1}{4} \Nsc \quad \land \quad
	 \Nfc < \left[ \frac{34}{3} \Nc^2 -\left(\frac43 \Nc -\frac{1}{\Nc}
	  \right) \Nsc \right] \left( \frac{13}{3}\Nc -\frac{1}{\Nc} 
	  \right)^{-1} \\ \notag\\
	  \Rightarrow & \left[ \left(\frac43 \Nc -\frac{1}{\Nc}\right)
	   -\frac14 \left( \frac{13}{3}\Nc -\frac{1}{\Nc}\right)\right] \Nsc <
	   \frac{34}{3} \Nc^2 -\frac{11}{2}\Nc \left( \frac{13}{3} \Nc-
	   \frac{1}{\Nc} \right)
	\end{align}
	Für $\Nc = 1$ folgt die untere Grenze $\Nsc \geq 14$, für $\Nc\geq 2$ 
	die obere Grenze $\Nsc \lessapprox -200$. 
      \end{enumerate}
      
     \subsubsection{Sattelpunkt}
      Am Sattelpunkt muss gelten
      \begin{equation}
      \alpha_1^* > 0 \quad \land \quad
      \alpha_2^* > 0 \quad \land \quad
      \left. \Det \right |_* < 0  \quad .
      \label{eq:beta_QCDxdQCD:alpha4_Sattelpunkt}
      \end{equation}
      In Koeffizienten ausgedrückt bedeutet das 
      \begin{equation}
       Z_1 X_2 < X_1 Y_2 \quad \land \quad Z_2 X_1 < X_2 Y_1 \quad \land \quad 
       Z_1 Z_2 > Y_1Y_2\quad , 
       \label{eq:beta_QCDxdQCD:sattelpunkt}
      \end{equation}
      auch hier ist eine Fallunerscheidung nötig, zunächst wieder ohne 
      Skalare.
      \begin{enumerate}
       \item $Y_1 > 0 \land Y_2 > 0$:
	 \begin{enumerate}
	  \item $X_1>0 \land X_2 >0$: Aus 
	  \eqref{eq:beta_QCDxdQCD:sattelpunkt} folgt 
	  \begin{equation}
	  Z_1<\frac{X_1}{X_2} Y_2 \quad \land \quad Z_2 < \frac{X_2}{X_1} Y_1
	  \quad\land\quad Z_1 Z_2 > Y_1 Y_2
	  \end{equation}
	  was jedoch nicht gleichzeitig möglich ist.
	 \item $X_1<0 \land X_2<0$:
	  Man erhält obere Begrenzungen für die Anzahl der joint-Fermionen 
	  \begin{align}
	   \nfc+\Nd \nfj < \Nc \quad\land\quad \text{c} \leftrightarrow\text{d}
	   \\\Rightarrow
	   \nfj^2 + \frac{\nfc}{\Nc}+\frac{11}{2}\frac{\nfc}{\Nc} < 
	   \left( \frac{11}{2}\right)^2
	    \quad\land\quad \text{c} \leftrightarrow\text{d} \quad .
	  \end{align}
	  Es gibt also eine allgemeine Obergrenze von $\nfj<\frac{11}{2}$.
	  Das Einführen von weiteren Fermionen verschiebt die Grenze weiter 
	  nach unten, für $\Nc=3$ und $\nfc = 6$ folgt $\nfj\leq 4$. Die 
	  einzigen Lösungen, die Gleichzeitig zu sinnvollen Fixpunkten 
	  führen sind
	  \begin{equation}
	   \Nc=3 \quad \Nd = 2 \quad \nfc =6 \quad 0\leq\nfd \leq 2 \quad 
	   \nfj =1 \quad ,
	  \end{equation}
	  weitere Lösungen gibt es nur für $\nfc<6$ oder $\Nc>4$.
	 \item $X_1<0 \land X_2 >0$:
	  Dann müsst
	  \begin{equation}
	   \underbrace{Z_1 X_2}_{>0} < \underbrace{X_1 Y_2}_{<0} \quad 
	   \blitz \quad ,
	  \end{equation}
	  dieser Fall kommt für physikalische Fixpunkte also nicht in Frage.
	 \end{enumerate}
	\item $Y_1 > 0 \land Y_2 < 0$:
	  \begin{enumerate}
	   \item $X_2>0$:
	      Wie schon gezeigt ist $Y_2<0 \land X_2>0$ nicht möglich.
	   \item $X_1>0 \land X_2<0$: Es kommt direkt zum Widerspruch, 
	      \begin{equation}
	       \underbrace{Z_2 X_1}_{>0} <\underbrace{ X_2 Y_1}_{<0} \quad 
	       \blitz \quad .
	      \end{equation}
	   \item $X_1<0 \land X_2<0$:
	      Auch hier erhält man eine Begrenzung für $\nfj$
	      \begin{align}
	        \nfc + \Nd \nfj<\frac{34}{13-\frac{3}{\Nc^2}} \Nc \quad
	        \land \quad \nfd + \Nc \nfj < \frac{11}{2} \Nd \\
	        \Rightarrow \nfj < \sqrt{\frac{11}{2} \frac{34}{13-
	        \frac{3}{\Nc^2}} } \lessapprox 3,9 \quad .
	      \end{align}
	      Gleichzeitige Lösungen zu \eqref{eq:beta_QCDxdQCD:sattelpunkt} 
	      mit $\Nc=3$ und $\nfc\geq 6$ gibt es nicht.
	  \end{enumerate}
	 \item $Y_1<0 \land Y_2<0$:
	  \begin{enumerate}
	   \item Ein $X_i>0$: Wieder ist $Y_i<0 \land X_i>0$ nicht möglich.
	   \item $X_1<0 \land X_2 < 0$: Hier folgt
	    \begin{align}
	     \nfc + \Nd \nfj<\frac{34}{13-\frac{3}{\Nc^2}}  
	        \quad \land \quad \text{c} \leftrightarrow \text{d} \\
	     \Rightarrow \nfj< 
	     \frac{34}{\sqrt{\left(13-\frac{1}{\Nc^2}\right)
	     \left(13-\frac{1}{\Nc^2}\right)}} \lessapprox 2,7  \quad .  
	    \end{align}
	    Auch hier gibt es keine Lösungen, die nah am SM sind.
	  \end{enumerate}
      \end{enumerate}
      
      Die Flussdiagramme für die drei Standardmodell-artigen Sattelpunkte 
      sind in den Abbildungen \ref{fig:beta_QCDxdQCD:Sattelpunkt1}, 
      \ref{fig:beta_QCDxdQCD:Sattelpunkt2} und 
      \ref{fig:beta_QCDxdQCD:Sattelpunkt3} zu sehen.


      
  \subsection{UV-Verhalten bei $\alpha^{*3}$}
    Damit der 
    An \eqref{eq:beta_QCDxdQCD:spur_determinante} erkennt man direkt 
    $\left.\Det \right|_*=0$