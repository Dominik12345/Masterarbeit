\section{Zusammenstellung der Ergebnisse}
  In dieser Arbeit wurde die $SU(\Nc)$-Quantenchromodynamik des Standarmodells 
  nach der Idee von Bai und Schwaller um eine neue $SU(\Nd)$-Eichgruppe zur 
  \QCDxdQCD
  erweitert und auf das Verhalten der laufenden Kopplungskonstanten im 
  UV-Grenzwert untersucht. 
  
  Aus den Stabilitätskriterien für einen UV-Fixpunkt war es möglich, 
  Obergrenzen für die Anzahl der Teilchen abzuleiten.
  Für teilweise Wechselwirkende Fixpunkte können dabei die Ungleichungen
  \begin{equation}
	  \left(\nfj+\frac{1}{4} \nsj \right)^2 + \left( \nfj+\frac{1}{4}\nsj 
	  \right) \left( \nfd +\frac14 \nsd \right)\frac1\Nc 
	  + \frac{11}{2} \left(\nfc+\frac14 \nsc \right)\frac1\Nc < 
	  \left( \frac{11}{2} \right)^2 \quad \land \quad \text{c}
	  \leftrightarrow
	  \text{d} \notag
  \end{equation}
  als notwendiges Kriterium für UV-Stabilität verstanden werden, und teilweise 
  konnten sogar noch strengere Grenzen gesetzt werden. Hieran wird klar, dass 
  insbesondere die Anzahl von joint-Fermionen und -Skalaren stark Begrenzt ist. 
  Bei der Extrapolation des Fixpunktes $(\alpha_\text{QCD}^*,0)$, der einer 
  endlichen QCD-Kopplung im UV entspricht, ergab sich ab der Entkopplungsskala 
  $Q$ ein Verhalten, dass qualitativ von den Standardmodellvorhersagen 
  abweicht. Die von Bai und Schwaller berechneten Entkopplungsskalen liegen 
  (teilweise) im Bereich einiger $\text{TeV}$ und somit in einem Bereich, der 
  experimentell zugänglich ist bzw. eine reelle Chance hat, an Collidern 
  zugänglich zu werden. Die Extrapolation zum Fixpunkt 
  $(0,\alpha_\text{dQCD})$ verhält sich dagegen auch über der Entkopplungsskala 
  ähnlich zur Standardmodell QCD, vermag es jedoch nicht, dieser einen 
  endlichen UV-Grenzwert zu geben. Der QCD Bereich ist in diesem Szenario also 
  wieder asymptotisch frei. 
  
  Der vollständig wechselwirkende Fixpunkt ist einerseits besonders interessant, 
  da er zur asymptotic safety in beiden Kopplungen gleichzeitig führen kann. 
  Gleichzeitig tritt hier jedoch das Problem auf, dass er betragsmäßig sehr 
  groß ist, sodass es hier zwingend notwendig ist, Skalare im joint-Sektor 
  einzuführen. Außerdem kann dieser Fixpunkt höchstens ein Sattelpunkt sein. 
  Die Extrapolation in Richtung großer QCD Kopplungen kann 
  prinzipiell zu einem Standarmodell ähnlichem Verhalten führen, jedoch 
  mit einem endlichen UV-Grenzwert. In der reinen \QCDxdQCD waren die 
  Fixpunkte zu betragsmäig sehr groß und $\alpha_\text{QCD}(Q)$ bereits sehr 
  nah am Fixpunkt, sodass die Abweichungen quantitativ sehr groß sind, wenn es 
  durch einen weiteren Mechanismus die Möglichkeit gäbe, den Fixpunkt zu 
  kleineren Werten zu verschieben, könnten hier durchaus sinnvolle 
  asymptotic safety Szenarien gefunden werden.
  
  Sowohl für teilweise als auch für den vollständig wechselwirkenden 
  UV-Fixpunkt muss jedoch erwähnt werden, dass der Gaußsche Fixpunkt eine 
  zweidimensionale kritische Hyperfläche besitzt. Bond und Litim haben 
  gezeigt, dass dies bei reinen Eichtheorien immer der Falls ist. Außerdem 
  verstehen sie ein bzw. zwei Trajektorien, die in einen solchen Fixpunkte 
  hineinlaufen, als Nullmenge unter allen möglichen UV-endlichen Trajektorien. 
  Hier ist es daher diskussionswürdig, ob die Wahl einer
  