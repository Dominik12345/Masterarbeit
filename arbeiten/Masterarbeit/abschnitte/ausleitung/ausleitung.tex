\clearpage
\section{Zusammenstellung der Ergebnisse}
  In dieser Arbeit wurde die $SU(\Nc)$-Quantenchromodynamik des Standarmodells 
  nach der Idee von Bai und Schwaller um eine neue $SU(\Nd)$-Eichgruppe zur 
  \QCDxdQCD
  erweitert und auf das Verhalten der laufenden Kopplungskonstanten im 
  UV-Grenzwert untersucht. 
  Aus den Stabilitätskriterien für einen UV-Fixpunkt war es möglich, 
  Obergrenzen für die Anzahl der Teilchen abzuleiten.
  Für teilweise Wechselwirkende Fixpunkte können dabei die Ungleichungen
%  \begin{equation}
%	  \left(\nfj+\frac{1}{4} \nsj \right)^2 + \left( \nfj+\frac{1}{4}\nsj 
%	  \right) \left( \nfd +\frac14 \nsd \right)\frac1\Nc 
%	  + \frac{11}{2} \left(\nfc+\frac14 \nsc \right)\frac1\Nc < 
%	  \left( \frac{11}{2} \right)^2 \quad \land \quad \text{c}
%	  \leftrightarrow
%	  \text{d} \notag
%  \end{equation}
	\eqref{eq:beta_QCDxdQCD:Fix3_ohne_Skalare}  
  als notwendiges Kriterium für UV-Stabilität verstanden werden, teilweise 
  konnten sogar noch strengere Grenzen gesetzt werden. Hieran wird klar, dass 
  insbesondere die Anzahl von joint-Fermionen und -Skalaren stark begrenzt ist. 
  Bei der Extrapolation des Fixpunktes $(\alpha_\text{QCD}^*,0)$, der 
  AS der QCD-Kopplung im UV entspricht, ergab sich ab der Entkopplungsskala 
  $Q$ ein Verhalten, dass qualitativ von den Standardmodellvorhersagen 
  abweicht. Die Extrapolation zum Fixpunkt 
  $(0,\alpha_\text{dQCD})$ verhält sich dagegen auch über der Entkopplungsskala 
  ähnlich zur Standardmodell QCD, führt aber stets zu einer asymptotisch 
  freien QCD.
  
  Der vollständig wechselwirkende Fixpunkt ist interessant, 
  da er zur asymptotic safety in beiden Kopplungen führen kann. 
  Gleichzeitig tritt hier jedoch das Problem auf, dass er betragsmäßig sehr 
  groß ist, sodass es hier notwendig ist, Skalare im joint-Sektor 
  einzuführen. Außerdem kann dieser Fixpunkt höchstens ein Sattelpunkt sein. 
  Die Extrapolation in Richtung großer QCD Kopplungen kann 
  prinzipiell zu einem Standarmodell ähnlichem Verhalten, jedoch 
  mit AS in QCD, führen. In der reinen \QCDxdQCD waren die 
  Fixpunkte betragsmäig sehr groß und $\alpha_\text{QCD}(Q)$ bereits sehr 
  nah am Fixpunkt, sodass die Abweichungen vom Standardmodell sehr groß sind. Wenn es 
  durch einen weiteren Mechanismus die Möglichkeit gäbe, den Fixpunkt zu 
  kleineren Werten zu verschieben, könnten hier aber sinnvolle 
  asymptotic safety Szenarien gefunden werden.
  
  Sowohl für teilweise als auch für den vollständig wechselwirkenden 
  UV-Fixpunkt muss jedoch erwähnt werden, dass der Gaußsche Fixpunkt eine 
  zweidimensionale kritische Hyperfläche besitzt. In \cite{Bond_Litim} werden die  
  ein bzw. zwei Trajektorien, die bei einer reinen Eichtheorie 
  in den 
  Fixpunkt 
  hineinlaufen, als Nullmenge unter allen möglichen UV-endlichen Trajektorien verstanden. 
  Hier steht daher zur Diskussion, ob die Wahl einer solchen Trajektorie als 
  \textit{fine tuning} zu verstehen ist. Desweiteren muss bemerkt werden, 
  dass die Theorie als dunkle Materie Modell, zumindest auf 2-Schleifen 
  Ordnung, nicht jeden UV-Fixpunkt erreichen kann. Da im dunkle Materie 
  Modell der vollständig wechselwirkende Fixpunkt als IR-Fixpunkt auftaucht, 
  ist $(0,\alpha_\text{dQCD})$ der sinnvollste UV Kandidat. Die gleichzeitige 
  Beschreibung von baryonischer DM und einer asymptotisch sicheren QCD 
  ist daher nur bei Entkopplungsskalen mindestens im hohen 
  $\text{TeV}$ Bereich möglich.
  
  