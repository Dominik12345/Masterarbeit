\section{Resummation der Ergebnisse}
  In dieser Arbeit wurde die $SU(\Nc)$-Quantenchromodynamik des Standarmodells 
  nach der Idee von Bai und Schwaller um eine neue $SU(\Nd)$-Eichgruppe zur 
  \QCDxdQCD
  erweitert und auf das Verhalten der laufenden Kopplungskonstanten im 
  UV-Grenzwert untersucht. 
  
  Am Beispiel einer skalaren $SU(N)$ wurden die wichtigsten Schritte der 
  Berechnung einer 1-Schleifen $\beta$-Funktion im Pfadintegralformalismus 
  nach Feynman nachvollzogen und argumentiert, warum Beiträge durch 
  Materiefelder und Selbstwechselwirkungen der Eichbosonen entgegengesetzte 
  Auswirkungen auf die laufende Kopplung haben. Die Lagrangedichte der 
  \QCDxdQCD für Fermionen und Skalare des QCD-, dQCD- und joint-Sektors 
  wurde formuliert und es wurde begründet, warum 3- und 4-Skalar 
  Wechselwirkungen nicht berücksichtigt werden, außerdem wurden unter der 
  Forderung nach Lorentz- und Eichinvarianz die Struktur möglicher Yukawa-Terme 
  diskutiert. Aus der allgemeinen Form der $\beta$-Funktion wurde die Rolle der 
  Eigenwerte und Eigenvektoren der Stabilitätsmatrix hinsichtlich des 
  UV-Verhaltens eines Fixpunktes abgeleitet. Für ein System aus zwei 
  Kopplungskonstanten wurde das UV-Verhalten eines hyperbolischen Fixpunktes 
  aus Spur und Determinante der Stabilitätsmatrix bestimmt und es wurde ein 
  alternatives Stabilitätskriterium für nicht-hyperbolische Fixpunkte 
  hergeleitet. Es wurde außerdem begründet, warum vollständig wechselwirkende 
  Fixpunkte in der Regel hyperbolisch und teilweise wechselwirkende Fixpunkte 
  nicht-hyperbolisch sind. Aus der 2-Schleifen 
  $\beta$-Funktion von D.R.T. Jones wurde 
  die konkrete $\beta$-Funktion der \QCDxdQCD mit Fermionen und Skalaren in 
  fundamentaler Darstellung  bestimmt und weiter untersucht. Neben dem 
  Gaußschen Fixpunkt $\alpha^{*1}$ wurden die teilweise wechselwirkenden 
  Fixpunkte $\alpha^{*2}$ und $\alpha^{*3}$ sowie der vollständig 
  wechselwirkende Fixpunkt $\alpha^{*4}$ gefunden. Unter Anwendung 
  Stabilitätskriterien wurde gezeigt, dass die kritische Hyperfläche 
  der wechselwirkenden Fixpunkte höchstens eindimensional ist, unabhängig 
  davon, ob Fermionen, Skalare oder beide Teilchensorten eingeführt werden. 
  Aus den Stabilitätsbedingungen wurden desweiteren Obergrenzen für die 
  Anzahl der neuen Teilchen des joint-Sektors von $\nfj \lesssim 3 $ bzw. 
  $\nsj \lesssim 9$ sowie für die Eichgruppengröße $\Nd\lesssim 10$ gefunden. 
  Die Teilchen des dQCD-Sektors haben einen deutlich kleineren Einfluss auf 
  die $\beta$-Funktion, sodass die Obergrenzen $\nfj \sim \nsj \lesssim 70$ 
  höher liegen. Eine hohe Anzahl an joint-Fermionen und -Skalaren führt dabei 
  auf betragsmäßig kleine Fixpunkte. Es wurde diskutiert, dass Yukawa-Terme zu 
  betragsmäßig größeren Fixpunkten führen, ohne die Yukawa-$\beta$-Funktion zu 
  berechnen kann hier jedoch keine quantitative Aussage getroffen werden. 
  Abschließend wurden die laufenden Kopplungen der Standardmodell QCD mit der 
  $\alpha_1$ Kopplung der \QCDxdQCD verglichen. 
  Dazu wurde eine numerische Methode zur Berechnung der kritischen Hyperläche 
  vorgestellt und auf Beispielmodelle angewandt. Die Extrapolation von 
  $\alpha^{*3}$ und $\alpha^{*4}$ in Richtung kleiner QCD-Kopplungen zeigt 
  dabei ab der Entkopplungsskala $Q$ 
  
  