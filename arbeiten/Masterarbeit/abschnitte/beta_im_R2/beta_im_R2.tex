\clearpage
\section{$\beta$-Funktion im $\mathbb{R}^2$}\label{beta_im_R2}
   Um 2-dimensionale autonome DGL-Systeme zu veranschaulichen eignet sich das 
   Bild des Kopplungskonstanten-Flusses. Dabei "`fließt"' ein Anfangswert 
   $\alpha(t_0)$ mit Geschwindigkeit und Richtung $\dot\alpha(t) = 
   \beta(\alpha(t))$ entlang einer Trajektorie durch 
   den Phasenraum. Der Satz von Picard-Lindelöf stellt dabei sicher, dass 
   sich zwei Trajektorien nicht schneiden. Ein Flussdiagramm wie in Abbildung 
   \ref{beta_allgemein:fig:fluss_beispiel} zeigt das Verhalten von $\alpha(t)$ 
   indem das Geschwindigkeitsfeld $\beta$ durch Pfeile im Phasenraum 
   dargestellt wird.
    
%    \begin{figure}[H]
%     \centering
%     \includegraphics[scale=0.5]{abschnitte/beta_allgemein/abb/fluss_beispiel}
%     \caption{Das Flussdiagramm für $\beta(\alpha_1,\alpha_2 =$.}
%     \label{beta_allgemein:fig:fluss_beispiel}
%    \end{figure}

  \subsection{Stabilitätsbedingungen}
    Für ein System mit zwei Kopplungskonstanten vereinfacht sich die 
    Untersuchung erheblich, da der Phasenraum der $\mathbb{R}^2$ ist und, wie in 
    Abschnitt \ref{Schleifen} gezeigt wurde, die Stabilitätsmatrix allgemein 
    die Form
	\begin{equation}
	 \dbda = \begin{pmatrix}
	          \sum\limits_{i=2,j=0}i \alpha_1^{i-1} \alpha_2^j X^1_{ij} &
	          \sum\limits_{i=2,j=1}j \alpha_1^i \alpha_2^{j-1} X^1_{ij}\\
	          \sum\limits_{i=1,j=2}i \alpha_1^{i-1} \alpha_2^j X^2_{ij} &
	          \sum\limits_{i=0,j=2}j \alpha_1^i \alpha_2^{j-1} X^2_{ij}
	         \end{pmatrix}\label{eq:beta_im_R2:stab_matrix}
	\end{equation}
    annimmt. Eigenwerte von $\dbda$ können explizit 
    als\begin{equation}
    \lambda_{+/-}=\frac12 \Sp \pm \sqrt{ \left( \frac{\Sp}{2} \right)^2-\Det }
    \end{equation}
    angegeben werden. 
    \subsubsection{hyperbolischer Fixpunkt}
      Sind $\Sp$ und $\Det$ nicht Null, lässt sich das Vorzeichen der 
      Eigenwerte leicht Nachrechnen, das Ergebnis ist in Tabelle 
      \ref{beta_im_R2:tab:stabilitaet_hyperbolisch} zu sehen.
  
      \begin{table}[h]
	\centering
	\begin{tabular}{ccccc}
	\toprule \midrule
	 $\Sp $& $\Det $ & $\Re\lambda_+$ &$\Re\lambda_-$ & UV-Verhalten \\
	 \midrule 
	 $>0$	& $>0$	&$>0$ & $>0$	& repulsiv   \\
	 $<0$	& $>0$	&$<0$ & $<0$	& attraktiv  \\
	  -     & $<0$	&$>0$ & $<0$	& Sattelpunkt\\
	  \midrule
	  \bottomrule
	\end{tabular}
	\caption{Das UV-Verhalten hyperbolischer Fixpunkte für $\Sp\neq0$ 
	und $\Det \neq 0$.}
	\label{beta_im_R2:tab:stabilitaet_hyperbolisch}
\end{table}
      
      Da man in der Reihenentwicklung der $\beta$-Funktion davon ausgeht, 
      dass höhere Ordnungen vernachlässigbar klein werden, kann man davon 
      ausgehen, dass hyperbolische Fixpunkte der 
      Ordnung $n$ auch für höhere Ordnungen hyperbolisch bleiben. Falls ein 
      vollständig wechselwirkender Fixpunkt $(\alpha_1^*\neq 0,
	 \alpha_2^*\neq 0)$ in Ordnung $n$ entweder $\left.\Sp \right|_*=0$ 
	 oder $\left. \Det \right|_*=0$ ergibt, kann man davon ausgehen, dass 
	 sich die Koeffizienten $X^k_{ij}$ zufällig aufheben und der Fixpunkt in 
	 höherer Ordnung wieder hyperbolisch wird. Vollständig 
	 wechselwirkende nicht-hyperbolische Fixpunkte können also als Folge 
	 einer zu groben Berechnung der $\beta$-Funktion verstanden werden. 
	 Es ist nicht klar, ob der Eigenvektor $e_i$ zum Eigenwert $\lambda_i=0$ 
	 eine UV-attraktiven oder repulsiven Richtung entspricht, da sich dies 
	 jedoch direkt auf die Dimension der kritischen Hyperfläche auswirkt ist 
	 die Untersuchung solcher Fixpunkte nur bedingt sinnvoll.

    \subsubsection{nicht-hyperbolischer Fixpunkt}
%       In dem Fall, dass bei $\alpha^*$ ein Eigenwert 
%       verschwindet, ist $\left.\Det\right|_*=\lambda_+|_* \lambda_-|_* = 0$ und
%       \begin{alignat}{3}
%       &\left.\lambda_+\right|_* = \left.\Sp\right|_* \quad && \lambda_-|_*=0  
%       \quad && \text{für }\left.\Sp\right|_*\geq 0 \\
%       &\lambda_+|_* = 0   \quad && \lambda_-|_*=\left.\Sp\right|_*  \quad && 
%       \text{für }\left.\Sp\right|_*\leq 0 
%       \quad .
%       \end{alignat}
%       Außerdem 
%       \begin{equation}
% 	\left.\frac{\partial \lambda_{+/-}}{\partial \alpha_i}\right|_*=
% 	\left[-\frac{1}{2} \frac{\partial \Sp}{\partial \alpha_i} 
% 	+ \frac{1}{\lambda_{-/+} }
% 	\frac{\partial \Det}{\partial \alpha_i} \right]_*
% 	\quad ,
%       \end{equation}
%       wobei $\lambda_{+/-}|_*$ der verschwindende Eigenwert ist.
	 Ein teilweise wechselwirkender Fixpunkt, o.E. $\alpha^*=(\alpha_1^*,0)$,  
	 führt zu der Stabilitätsmatrix 
	 \begin{equation}
	 \dbdafix = \begin{pmatrix}
	          \sum\limits_{i=2}i (\alpha_1^*)^{i-1}  X^1_{i0} &
	          \sum\limits_{i=2} (\alpha_1^*)^i  X^1_{i1}\\
	          0&
	          0
	         \end{pmatrix} \quad . \label{eq:beta_im_R2:stab_matrix_0}
	 \end{equation}
	 Es folgt $e_1=(1,0)^\text{T}$ zu $\lambda_1 = \sum_{i=2}i 
	 (\alpha_1^*)^{i-1}  X^1_{i0}$ und $e_2\propto(-(\sum_{i=2} (\alpha_1^*)^i  
	 X^1_{i1} )/(\sum_{i=2}i (\alpha_1^*)^{i-1}  X^1_{i0}) ,1)^\text{T}$ 
	 zu $\lambda_2=0$. Da in $\beta_2$ die Variable $\alpha_2$ in jedem Monom  
	 mindestens zur zweiten Potenz auftaucht, ist dies für 
	 beliebig hohe Ordnung zu erwarten, sodass hier ein alternatives Kriterium 
	 für das UV-Verhalten des Fixpunktes gefunden werden muss.  


	Zunächst wird \eqref{eq:beta_allgemein:beta_linear} um die zweite 
	Ordnung der Taylorentwicklung ergänzt,
     \begin{equation}
     \beta_i(\alpha) \simeq \sum\limits_{m=1}^2 \left. \frac{\partial \beta_i
     }{\partial
      \alpha_m}\right|_* \left(\alpha_m-\alpha^*_m\right) + \frac12 
      \sum\limits_{m,n=1}^2 
       \left(\alpha_m-\alpha^*_m\right)
      \left.\frac{\partial^2 \beta_i}{\partial\alpha_m \partial\alpha_n}
      \right|_* \left(\alpha_n-\alpha^*_n\right) \quad ,
     \end{equation}
     oder in in vektorieller Schreibweise in der Basis der Eigenvektoren
     \begin{equation}
     \ddt (K_1e_1+K_2e_2)\simeq\dbdafix (K_1e_1+K_2e_2) +
     (K_1e_1+K_2e_2) \cdot \left.\left(\nabla \dbda \right)\right|_*
      (K_1e_1+K_2e_2) \quad .
     \end{equation}
     Sei nun wieder\footnote{An Gleichung \eqref{eq:beta_im_R2:stab_matrix_0} 
     sieht man, dass beide Eigenwerte reell sind.} $\lambda_2=0$ und 
     $\lambda_1\neq 0$. Auf $\Mc$ gilt 
     \begin{itemize}
     	\item $\lambda_1<0 \Rightarrow K(t)
     		\overset{t\to \infty}{\longrightarrow} 0$
     	\item $\lambda_1<0 \Rightarrow K(t) \equiv 0$
	\end{itemize}
	
      
  \subsection{Fixpunktextrapolation}
    Ein besonderer Vorteil einer Erweiterung $G \to G_1\times G_2$ ist die 
    Möglichkeit einen UV-Fixpunkt eindeutig extrapolieren zu können, da die 
    kritische Hyperfläche die Dimension $\dim(\alpha^*)=0$, 
    $\dim(\text{Trajektorie})=1$ oder $\dim(\text{Phasenraum})=2$ hat. Im 
    ersten Fall, $\dim \Mc=0$ besteht sie nur aus dem Fixpunkt selbst, 
    dieser Fall ist also eher als eine mathematische, triviale Lösung zu 
    betrachten, die keine physikalische Bedeutung im Sinne laufender 
    Kopplungskonstanten hat. Im Fall $\dim \Mc =2$ besteht sie aus dem gesamten 
    Phasenraum. Weil in diesem Fall jede Trajektorie in den Fixpunkt 
    hineinläuft kann keine Vorhersage für die Größen der Kopplungskonstanten 
    gemacht werden, dafür ist aber das UV-Verhalten von dem Startwert 
    $\left(\alpha_1(t_0),\alpha_2(t_0)\right)$ unabhängig. Der für die 
    Extrapolation interessanteste Fall ist also $\dim \Mc = 1$, da die 
    UV-Hyperfläche dann aus zwei Trajektorien 
    $s^{+/-}:(0,\infty)\to \mathbb{R}^2$ besteht\footnote{Jeweils eine in, eine 
    entgegen der Richtung des attraktiven Eigenvektors.} und deshalb 
    eindeutige Wertepaare $(\alpha_1(t),\alpha_2(t))$ vorhersagt. Wenn eine 
    Kopplungskonstante (o.E.) $\alpha_1(t_0)$ bei einer Renormierungsskala 
    $t_0$ bekannt ist, und unter der Annahme, dass der Fixpunkt für 
    $t\to\infty$ erreicht wird, ist somit auch $\alpha_2(t_0)$ sowie das 
    gesamte Verhalten beider Kopplungskonstanten bekannt.
    
    Für die Extrapolation werden außerdem die folgende Beobachtungen ausgenutzt.
    \begin{enumerate}
     \item Eine Trajektorie, welche in einen Sattelpunkt hineinläuft, ist 
     gleichzeitig eine Separatrix, d.h. sie Teilt den Phasenrau in Gebiete mit 
     qualitativ unterschiedlichem  Verhalten für $t \to \infty$.
     \item Ein Sattelpunkt und ein attraktiver Fixpunkt sind mit einer 
     Trajektorie verbunden, ebenso ist ein Sattelpunkt mit einem repulsiven 
     Fixpunkt mit einer Separatrix verbunden, sofern die Fixpunkte existieren 
     und in der Nähe des Sattelpunktes liegen. 
    \end{enumerate}

    Da das gesuchte $\Mc$ folglich immer eine Separatrix ist, kann wie folgt 
    verfahren werden. Zunächst werden zwei Gebiete $L$ und $R$ definiert, die 
    zu qualitativ verschiedenen Trajektorien führen. Beispielsweise lassen sich 
    oft Abschätzungen der Art finden: Wenn es 
    ein $t_1$ gibt mit
    \begin{equation}
     \alpha_j(t_1) > \max \left\{ \left. \alpha^{*i}_j \right|\text{alle 
     Fixpunkte } 
     \alpha^{*i}\right\} \quad ,
    \end{equation}
    dann kann der gewünschte Fixpunkt nicht mehr für $t>t_1$ erreicht werden.
    Am Fixpunkt wird eine Orthonormalbasis $\{f_1,f_2\}$ gewählt\footnote{Der 
    Einfachheit halber kann die Basis aus Eigenvektoren oder die 
    $\alpha_{1,2}$-Achsen gewählt werden, sofern dies 
    zu keinen numerischen Schwierigkeiten führt.}
    und der Phasenraum in Ebenen mit Abstand $\epsilon$ 
    eingeteilt, die nullte Ebene geht dabei durch den Fixpunkt. Rekursiv werden 
    dann
    \begin{equation}
     s^{L/R}_n = s^{L/R}_{n-1} + \epsilon f_1 + d^{L/R} \delta f_2 
    \end{equation}
    definiert. Für festes $\delta$ wird $d^{L/R}$ so eingestelle, dass die 
    Trajektorie 
    mit Anfangswert $s^\text{L}_n$ in den Bereich $L$ hineinläuft, analog für 
    $R$. 
    Mit $s^L_0:=\alpha-\nicefrac{\delta}{2}  f_2$ und 
    $s^R_0:=\alpha+\nicefrac{\delta}{2} f_2$ 
    ergibt sich so ein Schlauch $\left(s^{L}_n,s^R_n \right)_{n=0,1,\ldots}$ 
    der Breite 
    $\delta$, der die Separatrix beinhaltet.

    