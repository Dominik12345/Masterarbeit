\section{2-Schleifen $\beta$-Funktion im $\mathbb{R}^2$}

\section{Die $\beta$-Funktion}

  \subsection{$\beta(g)$}
      
	Aus ``Two-loop $\beta$-function for a $G_1\times G_2$ gauge symmetry''\cite{PhysRevD.25.581} übernehmen 
	wir die allgemeine Form einer 2-loop-$\beta$-Funktion
	\begin{align}
	      \begin{aligned}
	      \beta_1(g_1,g_2)&=X_1 g_1^3+ Y_1 g_1^5+ Z_1g_1^3g_2^2\\
	      \beta_2(g_1,_g2)&=X_2 g_2^3+Y_2 g_2^5+ Z_2 g_1^2 g_2^3 \quad. 
	      \end{aligned} \label{beta(g)}
	\end{align}
	Zur besseren Übersichtlichkeit, definiere $\beta(g):=(\beta_1(g_1,g_2),\beta_2(g_1,g_2))^T$ sowie $g=(g_1,g_2)^T$.
	Außerdem ist $g=g(\mu)$ und
	\begin{equation}
	\beta=\mu \frac{\d g}{\d \mu}
	\end{equation}