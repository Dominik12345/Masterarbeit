\clearpage
\section{$\beta$-Funktion im $\mathbb{R}^2$}\label{beta_im_R2}
  Durch die Erweiterung einer Eichgrupppe $G$ mit Kopplungskonstante $g$ (bzw. $\alpha$) auf die 
  kombinierte Eichgruppe $G_1\times G_2$ mit den Kopplungskonstanten $g_1$ und $g_2$ (bzw. 
  $\alpha_1$ und $\alpha_2$)

  \subsection{Stabilitätsbedingungen}
    Für ein System mit zwei Kopplungskonstanten vereinfacht sich die Untersuchung erheblich, 
    da der Phasenraum der $\mathbb{R}^2$ ist und die Eigenwerte von 
    $\dbda \in\mathbb{R}^{2\times 2}$ 
    explizit als 
    \begin{equation}
    \lambda_{+/-}=\frac12 \Sp \pm \sqrt{ \left( \frac{\Sp}{2} \right)^2-\Det }
    \end{equation}
    angegeben werden können. In dem Fall, dass bei $\alpha^*$ ein Eigenwert verschwindet, ist 
    $\left.\Det\right|_*=\lambda_+|_* \lambda_-|_* = 0$ und
    \begin{alignat}{3}
    &\left.\lambda_+\right|_* = \left.\Sp\right|_* \quad && \lambda_-|_*=0  
    \quad && \text{für }\left.\Sp\right|_*\geq 0 \\
    &\lambda_+|_* = 0   \quad && \lambda_-|_*=\left.\Sp\right|_*  \quad && 
    \text{für }\left.\Sp\right|_*\leq 0 
    \quad .
    \end{alignat}
    Außerdem 
    \begin{equation}
      \left.\frac{\partial \lambda_{+/-}}{\partial \alpha_i}\right|_*=
      \left[-\frac{1}{2} \frac{\partial \Sp}{\partial \alpha_i} 
      + \frac{1}{\lambda_{-/+} }
      \frac{\partial \Det}{\partial \alpha_i} \right]_*
      \quad ,
    \end{equation}
    wobei $\lambda_{+/-}|_*$ der verschwindende Eigenwert ist.
    
  \subsection{Fixpunktextrapolation}
    Ein besonderer Vorteil einer Erweiterung $G \to G_1\times G_2$ ist die Möglichkeit einen 
    UV-Fixpunkt eindeutig extrapolieren zu können, da die kritische Hyperfläche die Dimension 
    $\dim(\alpha^*)=0$, $\dim(\text{Trajektorie})=1$ oder $\dim(\text{Phasenraum})=2$ hat. Im 
    ersten Fall, $\dim \Mc=0$ besteht sie nur aus dem Fixpunkt selbst, 
    dieser Fall ist also eher als eine mathematische, triviale Lösung zu betrachten, die keine 
    physikalische Bedeutung im Sinne laufender Kopplungskonstanten hat. Im Fall $\dim \Mc =2$ 
    besteht sie aus dem gesamten Phasenraum. Weil in diesem Fall jede Trajektorie in den Fixpunkt 
    hineinläuft 
    kann keine Vorhersage für die Größen der Kopplungskonstanten gemacht werden, dafür ist aber 
    das UV-Verhalten von dem Startwert $\left(\alpha_1(t_0),\alpha_2(t_0)\right)$ unabhängig. 
    Der für die Extrapolation interessanteste Fall ist also $\dim \Mc = 1$, da die UV-Hyperfläche 
    dann aus zwei Trajektorien $s^{+/-}:(0,\infty)\to \mathbb{R}^2$ besteht\footnote{Jeweils 
    eine in, eine entgegen der Richtung des attraktiven Eigenvektors.} und deshalb 
    eindeutige Wertepaare $(\alpha_1(t),\alpha_2(t))$ vorhersagt. Wenn eine Kopplungskonstante 
    (o.E.) $\alpha_1(t_0)$ 
    bei einer Renormierungsskala $t_0$ bekannt ist, und unter der Annahme, dass der Fixpunkt für 
    $t\to\infty$ erreicht wird, ist somit auch $\alpha_2(t_0)$ sowie das gesamte Verhalten beider 
    Kopplungskonstanten bekannt.
    
    Für die Extrapolation werden außerdem die folgende Beobachtungen ausgenutzt.
    \begin{enumerate}
     \item Eine Trajektorie, welche in einen Sattelpunkt hineinläuft, ist gleichzeitig eine 
      Separatrix, d.h. sie Teilt den Phasenrau in Gebiete mit qualitativ unterschiedlichem 
      Verhalten für $t \to \infty$.
     \item Ein Sattelpunkt und ein attraktiver Fixpunkt sind mit einer Trajektorie verbunden, 
      ebenso ist ein Sattelpunkt mit einem repulsiven Fixpunkt mit einer Separatrix verbunden, 
      sofern die Fixpunkte existieren und in der Nähe des Sattelpunktes liegen. 
    \end{enumerate}

    Da das gesuchte $\Mc$ folglich immer eine Separatrix ist, kann wie folgt verfahren werden. 
    Zunächst werden zwei Gebiete $L$ und $R$ definiert, die zu qualitativ verschiedenen 
    Trajektorien führen. Beispielsweise lassen sich oft Abschätzungen der Art finden: Wenn es 
    ein $t_1$ gibt mit
    \begin{equation}
     \alpha_j(t_1) > \max \left\{ \left. \alpha^{*i}_j \right|\text{alle Fixpunkte } 
     \alpha^{*i}\right\} \quad ,
    \end{equation}
    dann kann der gewünschte Fixpunkt nicht mehr für $t>t_1$ erreicht werden.
    Am Fixpunkt wird eine Orthonormalbasis $\{f_1,f_2\}$ gewählt\footnote{Der Einfachheit halber 
    kann die Basis aus Eigenvektoren oder die $\alpha_{1,2}$-Achsen gewählt werden, sofern dies 
    zu keinen numerischen schwierigkeiten führt.}
    und der Phasenraum in Ebenen mit Abstand $\epsilon$ 
    eingeteilt, die nullte Ebene geht dabei durch den Fixpunkt. Rekursiv werden dann
    \begin{equation}
     s^{L/R}_n = s^{L/R}_{n-1} + \epsilon f_1 + d^{L/R} \delta f_2 
    \end{equation}
    definiert. Für festes $\delta$ wird $d^{L/R}$ so eingestelle, dass die Trajektorie 
    mit Anfangswert $s^\text{L}_n$ in den Bereich $L$ hineinläuft, analog für $R$. 
    Mit $s^L_0:=\alpha-\nicefrac{\delta}{2}  f_2$ und $s^R_0:=\alpha+\nicefrac{\delta}{2} f_2$ 
    ergibt sich so ein Schlauch $\left(s^{L}_n,s^R_n \right)_{n=0,1,\ldots}$ der Breite 
    $\delta$, der die Separatrix beinhaltet.

    