\clearpage
\section{Quantenfeldtheorie und Standardmodell}
  In einer \textit{Quantenfeldtheorie (QFT)} werden physikalische Entitäten 
  als Anregungszustände von Quantenfeldern 
  verstanden. Ein Schlüssel zu den experimentell zugänglichen 
  Wirkungsquerschnitten ist die Berechnung von \textit{Korrelatorfunktionen} 
  oder \textit{$n$-Punkt Funktionen}
  \begin{equation}
   \Big\langle \textbf{T} \,\,\, \phi^{r_1}(x_1) \ldots \phi^{r_N}(x_N) 
   \Big\rangle
   =\frac{\int \mathcal{D}\phi \,\,\, \phi^{r_1}(x_1) \ldots \phi^{r_N}(x_N) 
   \e^{\i S[\phi]}}{\int \mathcal{D}\phi \,\,\, \e^{\i S[\phi]} }
   \label{renormierung:Korrelator_definition}
  \end{equation}
  im Pfadintegralformalismus von Feynman \cite{Schwartz}. Die Information über 
  die möglichen 
  physikalischen Prozesse der Quantenfelder $\phi$ 
  ist dabei in dem Wirkungsfunktional $S$ bzw. der Lagrangedichte $\mathcal{L}$ 
  enhalten, welche in einer $d$-Dimensionalen Raumzeit über 
  \begin{equation}
    S[\phi] = \int \d^d x \,\,\, \mathcal{L}(\phi,\partial \phi, t) 
    \label{renormierung:Wirkung_definition}
  \end{equation}
  verknüpft sind \cite{Schwartz}. Die Felder $\phi$ können dabei qualitativ 
  unterschiedlich sein, d.h. in verschiedene Räume abbilden und somit 
  verschiedene Teilcheneigenschaften darstellen. Dies soll durch den Index 
  $r$ verdeutlicht werden.

  \subsection{Das Standardmodell der Teilchenphysik}
    Das SM ist eine QFT nach dem Prinzip, invariant unter bestimmten 
    Symmetrietransformationen zu sein. Die Dynamik wird dabei durch die 
    Eichgruppe $SU(3)\times SU(2)\times U(1)$ in \\der Lagrangedichte 
    repräsentiert sowie der Angabe, 
    unter welcher Darstellung der Eichgruppe die Felder transformieren. 
    Neben den Feldern der Quarks, Leptonen und des Higgs wird so die 
    Existenz von Eichfeldern und Wechselwirkungen mit den übrigen Feldern 
    gefordert, welche keine Singletts der entsprechenden Eichgruppe sind.

    Quarkfelder unterscheiden sich unter anderem in ihren Quantenzahlen 
    Flavour und Colour. Im SM ist die Anzahl verschiedener 
    Flavour $N_\text{Flavour}=6$ und die Colour Anzahl $N_\text{Colour} = 3$. 
    Sie transformieren als Colour-Tripletts unter der fundamentalen Darstellung 
    der $SU(3)$. Als Yang-Mills Theorie werden außerdem $\Nc^2-1 = 8$ 
    Eichfelder, die Gluonen, sowie Wechselwirkungen gefordert \cite{Zinn}. 
    Diese Wechselwirkung wird  als \textit{Quantenchromodynamik (QCD)} oder 
    \textit{starke Wechselwirkung} bezeichnet.
    
    
  \subsection{Effektive Quantenwirkung und Gell-Mann-Low Gleichung}
    Um das Hochenergieverhalten von Wechselwirkungen zu untersuchen, ist 
    es sinnvoll die \textit{Quantenwirkung} $\Gamma$ einzuführen. Dazu werden 
    nun die wichtigsten Punkte der Berechnung von Korrelationsfunktionen mit 
    Hilfe von erzeugenden Funktionalen gezeigt\footnote{Für das Vorgehen beim 
    reellen, skalaren Feld vgl. \cite{Zinn}, für die Erweiterung mit Fermionen 
    vgl. \cite{Schwartz}.}.
    
    Der Einfachheit halber wird nun nur ein komplexes Fermionfeld $\psi$ und ein 
    komplexes Skalarfeld $\phi$ betrachtet. Wegen der linearen Eigenschaften von 
    \eqref{renormierung:Korrelator_definition} ist die Erweiterung auf mehrere 
    Felder trivial. Lorentz- und Eichinvarianz der Lagrangedichte erfordern 
    direkt die Einführung der Felder $\psibar$ und $\phistar$\footnote{Cite}. 
    Das \textit{erzeugende Funktional} wird definiert als 
    \begin{equation}
    Z[J] := \int \mathcal{D}\Psi \,\,\, 
    \e^{\i S[\Psi]+ 
    \i 
    \int \d^d x \, \left(J \cdot \Psi \right)}
    \quad ,
    \label{renormierung:erzeugendes_Funktional_definition} 
    \end{equation}
    dabei wurden die Ströme $\eta$, $\etabar$, $\zeta$ und $\zetastar$ und die 
    Schreibweise $J=(\eta,\etabar,\zeta,\zetastar)^\text{T}$ und 
    $\Psi=(\psibar,\psi,\phi^*,\phi)^\text{T}$
    eingeführt. Eine Reihenentwicklung kann durch
    \begin{equation}
     Z[J] = \sum\limits_\alpha \frac{ \i^{|\alpha|}}{\alpha!}  
     \int \d x_1 \ldots \d x_{|\alpha|} Z^{\alpha}(x_1,\ldots,x_{|\alpha|} )
     J^\alpha (x) \label{renormierung:Reihenentwicklung}
    \end{equation}
    definiert werden, mit einem vierer 
    Multiindex $\alpha=(\alpha_1,\alpha_2
    ,\alpha_3,\alpha_4)$ und der Schreibweise 
    \begin{equation}
      J^\alpha=
      \eta	(x_1)		\ldots \eta(x_{\alpha_1}) 	\,\,
      \etabar(x_{\alpha_1 +1})	\ldots \etabar(x_{\alpha_1+\alpha_2})	\,\,
      \zeta(x_{\alpha_1+\alpha_2 +1})	\ldots 
      \zeta(x_{\alpha_1+\alpha_2+\alpha_3})	\,\,
      \zetastar(x_{\alpha_1+\alpha_2+\alpha_3 +1})	\ldots 
      \zetastar(x_{|\alpha|})   \quad .
      \label{renormierung:Strom}
    \end{equation}
    Die Funktionen $Z^\alpha$ können über die Funktionalableitung 
    \begin{equation}
     Z^\alpha (y_1,\ldots , y_{|\alpha|}) \overset{
     \eqref{renormierung:Reihenentwicklung}}{=} \left[
      \frac{\partial}{\i^{|\alpha|}\partial J^\alpha(y)}\right]_{J=0} Z[J] 
     \overset{\eqref{renormierung:erzeugendes_Funktional_definition}}{=} 
     \int \mathcal{D}\Psi \,\,\, \Psi^\alpha(y) e^{\i S[\Psi]} 
     \overset{\eqref{renormierung:Korrelator_definition}}{=} 
     \Big\langle \textbf{T} \Psi^\alpha(y) \Big\rangle     
    \end{equation}
    als $n$-Punkt Funktionen verstanden werden. Zur Vereinfachung wurde dabei 
    $\Psi^\alpha$ analog zu \eqref{renormierung:Strom} definiert und 
    $Z[0]=0$ gesetzt.
    
    Mit der Definition des Funktionals 
    \begin{equation}
     W[J] := \ln Z[J]
    \end{equation}
    und einer Reihenentwicklung wie in \eqref{renormierung:Reihenentwicklung} 
    erhät man Funktionen $W^\alpha$, die die Cluster-Eigenschaft erfüllen
    \cite{Zinn}. Sind $A$ und $B$ Ströme mit disjunkten Trägern\footnote{Dabei 
    können $A$ und $B$ eine Zerlegung eines Stroms $J_i=A+B$ oder zwei 
    verschiedene, z.B $A=\eta$, $B=\phi$ sein.} und $y_1\in\text{supp}A$, 
    $y_2\in\text{supp}B$, dann müssen $W^\alpha(\ldots,y_1,\ldots,y_2,\ldots) 
    \overset{||y_1-y_2||\to \infty}{\longrightarrow} 0 $. Aufgund dieser 
    Eigenschaft ist $W[J]$ das erzeugende Funktional ver




