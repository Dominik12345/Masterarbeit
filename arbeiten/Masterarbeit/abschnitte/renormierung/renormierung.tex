\clearpage
\section{Quantenfeldtheorie und Standardmodell}
  In einer \textit{Quantenfeldtheorie (QFT)} werden physikalische Entitäten 
  als Anregungszustände von Quantenfeldern 
  verstanden. Ein Schlüssel zu den experimentell zugänglichen 
  Wirkungsquerschnitten ist die Berechnung von \textit{Korrelatorfunktionen} 
  oder \textit{$n$-Punkt Funktionen}
  \begin{equation}
   \Big\langle \textbf{T} \,\,\, \phi^{r_1}(x_1) \ldots \phi^{r_N}(x_N) 
   \Big\rangle
   =\frac{\int \mathcal{D}\phi \,\,\, \phi^{r_1}(x_1) \ldots \phi^{r_N}(x_N) 
   \e^{\i S[\phi]}}{\int \mathcal{D}\phi \,\,\, \e^{\i S[\phi]} }
   \label{renormierung:Korrelator_definition}
  \end{equation}
  im Pfadintegralformalismus von Feynman \cite{Schwartz}. Die Information über 
  die möglichen 
  physikalischen Prozesse der Quantenfelder $\phi$ 
  ist dabei in dem Wirkungsfunktional $S$ bzw. der Lagrangedichte $\mathcal{L}$ 
  enhalten, welche in einer $d$-Dimensionalen Raumzeit über 
  \begin{equation}
    S[\phi] = \int \d^d x \,\,\, \mathcal{L}(\phi,\partial \phi, t) 
    \label{renormierung:Wirkung_definition}
  \end{equation}
  verknüpft sind \cite{Schwartz}. Die möglicherweise verschiedenartigen Felder  
  $\phi^{r_i}$ können dabei in verschiedene Räume abbilden und somit 
  Teilcheneigenschaften, insbesondere Transformationseigenschaften unter 
  Eichsymmetrien, darstellen. 

  \subsection{Das Standardmodell der Teilchenphysik}
    Das SM ist eine QFT nach dem Prinzip, invariant unter bestimmten 
    Symmetrietransformationen zu sein. Die Dynamik wird dabei durch die 
    Eichgruppe $SU(3)\times SU(2)\times U(1)$ in \\der Lagrangedichte 
    repräsentiert, sowie der Angabe, 
    unter welcher Darstellung der Eichgruppe die Felder transformieren. 
    Neben den postulierten Feldern der Quarks, Leptonen und des Higgs wird so 
    die Existenz von Eichfeldern und Wechselwirkungen mit den übrigen Feldern 
    gefordert, welche keine Singletts der entsprechenden Eichgruppe sind.

    Im SM kennt man bisher sechs Quark-Flavour, \textit{up (u), down (d), 
    charm (c), strange (s), top (t), }und \textit{bottom (b)}, die als 
    Dirac-Fermionen in der Lagrangedichte auftauchen. Ebenfalls als 
    Dirac-Fermionen werden die geladenen Leptonen, das \textit{Elektron 
    ($\text{e}^-$)}, das \textit{Myon} ($\mu^-$) und das \textit{Tauon} 
    ($\tau^-$) 
    eingeführt. Die drei \textit{Neutrinos} $\nu_\text{e}$, $\nu_\mu$ und 
    $\nu_\tau$ 
    kommen dagegen ausschließlich als linkshändige Weyl-Spinoren vor, sodass 
    sie im SM masselos sein müssen. Das einzige skalare Feld des SM ist 
    das \textit{Higgs-Feld ($\phi$)}, welches im Higgsmechanismus für die 
    Brechung der Symmetriegruppe verantwortlich ist.
    
    Als eine der fundamentalen Kräfte des SM beschreibt die 
    \textit{Quantenchromodynamik (QCD)} die Wechselwirkungen 
    zwischen Quarks, den Bausteinen der Hadronen, und den Gluonen, den 
    Eichfeldern der QCD. Die mathematische Beschreibung erfolgt durch 
    Darstellungen der $SU(3)$, bei der einem Quark $\psi^f$ mit Flavour $f$ ein 
    Colour-Triplett $\psi = \left(\psi^f_1, \psi_2^f, \psi_3^f\right)^\text{T}$ 
    zugeordnet wird. 
    Außerdem folgt die Existenz von acht masselosen Gluonen, die mit Quarks und 
    untereinander wechselwirken \cite{Ellis_Webber}. Eine genauere 
    mathematische Beschreibung folgt in Abschnitt \ref{QCDxdQCD}. 
    Charakteristisch für die QCD des SM ist die in Abbildung 
    \ref{renormierung:fig:running_couplings} dargestellte Energieabhängigkeit 
    der QCD Kopplungskonstanten. Der Landau-Pol bei $\Lambda_\text{QCD}$ ist 
    dabei für die hadronische Bindung bei niedrigen Energien verantwortlich 
    und somit insbesondere für die baryonische Massendichte $\Omega_\text{B}$ 
    im Universum, während es bei hohen Energien gerade zum AF kommt.
    
    Die Symmetriegruppe $SU(2)\times U(1)$ beschreibt die Elektroschwache 
    Wechselwirkung. In ihr werden die Wechselwirkungen zwischen  
    linkshändigen Isospin-Dupletts 
    \begin{equation}
      \begin{pmatrix}
 u_L \\ d_L
\end{pmatrix}
\,,\quad
\begin{pmatrix}
 c_L \\ s_L
\end{pmatrix}
\,,\quad
\begin{pmatrix}
 t_L \\ b_L
\end{pmatrix}
\,,\quad
\begin{pmatrix}
 \nu_e\\ e^-_L 
\end{pmatrix}
\,,\quad
\begin{pmatrix}
  \nu_\mu \\\mu^-_L 
\end{pmatrix}
\,,\quad
\begin{pmatrix}
 \nu_\tau\\ \tau^-_L 
\end{pmatrix}
\,,

    \end{equation}
    über insgesamt vier Eichbosonen beschrieben. Durch den Higgsmechanismus 
    wird die Symmetriegruppe zu einer $U(1)$ gebrochen, welche als 
    Elektromagnetismus identifiziert werden kann. Als Eichbosonen der 
    gebrochenen Symmetrie entstehen die massiven Bosonen $W^+$, $W^-$ und $Z$ 
    sowie das masselose Photon.
    
    Im Higg-Sektor gibt es neben einem Potenzial $V[\phi]$ sogenannte 
    Yukawa-Kopplungen der 
    Form $Y_{ij} Q_L^i \phi q_R^j $
    zwischen einem Isospin-Duplett $Q_L^i$, dem Higgsfeld $\phi$ und einem 
    rechtshändigen Fermion $q_R^j$, mit der Kopplungsstärke $Y_{ij}$. Im 
    Higgsmechanismus wird das 
    Potenzial $V[\phi]$ minimiert und die $SU(2)\times U(1)$ Symmetrie 
    unterhalb einer charakteristischen Energieskala gebrochen, was 
    die Massenerzeugung der Teilchen und das Flavour-Mixing der Quarks zur 
    Folge hat. \cite{PDG:Higgs}
    
    Da in dieser Arbeit wird eine Standardmodellerweiterung im QCD-Sektor 
    untersucht werden soll, werden der Elektroschwache Sektor sowie der 
    Higgs-Sektor nicht weiter Betrachtet. Die einzige Außnahme ist Abschnitt 
    \ref{beta_im_SM}, in dem die $\beta$-Funktionen des Standardmodells 
    vorgestellt werden.
    
    
    

    
    
  \subsection{Effektive Quantenwirkung und Gell-Mann-Low Gleichung}
    Um das Hochenergieverhalten von Wechselwirkungen zu untersuchen, ist 
    es sinnvoll die \textit{Quantenwirkung} $\Gamma$ einzuführen. Dazu werden 
    nun die wichtigsten Punkte der Berechnung von Korrelationsfunktionen mit 
    Hilfe von erzeugenden Funktionalen gezeigt\footnote{Für das Vorgehen beim 
    reellen, skalaren Feld vgl. \cite{Zinn}, für die Erweiterung mit Fermionen 
    vgl. \cite{Schwartz}.}.
    
    Der Einfachheit halber wird nun nur ein komplexes Fermionfeld $\psi$ und ein 
    komplexes Skalarfeld $\phi$ betrachtet. Wegen der linearen Eigenschaften von 
    \eqref{renormierung:Korrelator_definition} ist die Erweiterung auf mehrere 
    Felder trivial. Lorentz- und Eichinvarianz der Lagrangedichte erfordern 
    direkt die Einführung der Felder $\psibar$ und $\phistar$\footnote{Cite}. 
    Das \textit{erzeugende Funktional} wird definiert als 
    \begin{equation}
    Z[J] := \int \mathcal{D}\Psi \,\,\, 
    \e^{\i S[\Psi]+ 
    \i 
    \int \d^d x \, \left(J \cdot \Psi \right)}
    \quad ,
    \label{renormierung:erzeugendes_Funktional_definition} 
    \end{equation}
    dabei wurden die Ströme $\eta$, $\etabar$, $\zeta$ und $\zetastar$ und die 
    Schreibweise $J=(\eta,\etabar,\zeta,\zetastar)^\text{T}$ und 
    $\Psi=(\psibar,\psi,\phi^*,\phi)^\text{T}$
    eingeführt. Eine Reihenentwicklung kann durch
    \begin{equation}
     Z[J] = \sum\limits_\alpha \frac{ \i^{|\alpha|}}{\alpha!}  
     \int \d x_1 \ldots \d x_{|\alpha|} Z^{\alpha}(x_1,\ldots,x_{|\alpha|} )
     J^\alpha (x) \label{renormierung:Reihenentwicklung}
    \end{equation}
    definiert werden, mit einem vierer 
    Multiindex $\alpha=(\alpha_1,\alpha_2
    ,\alpha_3,\alpha_4)$ und der Schreibweise 
    \begin{equation}
      J^\alpha=
      \eta	(x_1)		\ldots \eta(x_{\alpha_1}) 	\,\,
      \etabar(x_{\alpha_1 +1})	\ldots \etabar(x_{\alpha_1+\alpha_2})	\,\,
      \zeta(x_{\alpha_1+\alpha_2 +1})	\ldots 
      \zeta(x_{\alpha_1+\alpha_2+\alpha_3})	\,\,
      \zetastar(x_{\alpha_1+\alpha_2+\alpha_3 +1})	\ldots 
      \zetastar(x_{|\alpha|})   \quad .
      \label{renormierung:Strom}
    \end{equation}
    Die Funktionen $Z^\alpha$ können über die Funktionalableitung 
    \begin{equation}
     Z^\alpha (y_1,\ldots , y_{|\alpha|}) \overset{
     \eqref{renormierung:Reihenentwicklung}}{=} \left[
      \frac{\partial}{\i^{|\alpha|}\partial J^\alpha(y)}\right]_{J=0} Z[J] 
     \overset{\eqref{renormierung:erzeugendes_Funktional_definition}}{=} 
     \int \mathcal{D}\Psi \,\,\, \Psi^\alpha(y) e^{\i S[\Psi]} 
     \overset{\eqref{renormierung:Korrelator_definition}}{=} 
     \Big\langle \textbf{T} \Psi^\alpha(y) \Big\rangle     
    \end{equation}
    als $n$-Punkt Funktionen verstanden werden. Zur Vereinfachung wurde dabei 
    $\Psi^\alpha$ analog zu \eqref{renormierung:Strom} definiert und 
    $Z[0]=0$ gesetzt.
    
    Mit der Definition des Funktionals $W[J] := \ln Z[J]$
    und einer Reihenentwicklung wie in \eqref{renormierung:Reihenentwicklung} 
    erhät man Funktionen $W^\alpha$, die die Cluster-Eigenschaft erfüllen
    \cite{Zinn}. Sind $A$ und $B$ Ströme mit disjunkten Trägern\footnote{Dabei 
    können $A$ und $B$ eine Zerlegung eines Stroms $J_i=A+B$ oder zwei 
    verschiedene, z.B $A=\eta$, $B=\phi$ sein.} und $y_1\in\text{supp}A$, 
    $y_2\in\text{supp}B$, dann müssen $W^\alpha(\ldots,y_1,\ldots,y_2,\ldots) 
    \overset{||y_1-y_2||\to \infty}{\longrightarrow} 0 $. Aufgund dieser 
    Eigenschaft ist $W[J]$ das erzeugende Funktional ver




