\clearpage
\section{Quantenfeldtheorie und Standardmodell}
  In einer \textit{Quantenfeldtheorie (QFT)} werden physikalische Entitäten 
  als Anregungszustände von Quantenfeldern 
  verstanden. Ein Schlüssel zu den experimentell zugänglichen 
  Wirkungsquerschnitten ist die Berechnung von \textit{Korrelatorfunktionen} 
  oder \textit{$n$-Punkt Funktionen}
  \begin{equation}
   \Big\langle \textbf{T} \,\,\, \phi^{r_1}(x_1) \ldots \phi^{r_N}(x_N) 
   \Big\rangle
   =\frac{\int \mathcal{D}\phi \,\,\, \phi^{r_1}(x_1) \ldots \phi^{r_N}(x_N) 
   \exp\left\{{\i S[\phi]}\right\}}{\int \mathcal{D}\phi \,\,\, \exp\left\{\i S[\phi]\right\} }
   \label{renormierung:Korrelator_definition}
  \end{equation}
  im Pfadintegralformalismus von Feynman \cite{Schwartz}. Die Information über 
  die möglichen 
  physikalischen Prozesse der Quantenfelder $\phi$ 
  ist dabei in dem Wirkungsfunktional $S$ bzw. der Lagrangedichte $\mathcal{L}$ 
  enhalten, welche in einer $D$-Dimensionalen Raumzeit über 
  \begin{equation}
    S[\phi] = \int \d^D x \,\,\, \mathcal{L}(\phi,\partial \phi, t) 
    \label{renormierung:Wirkung_definition}
  \end{equation}
  verknüpft sind \cite{Schwartz}. Die möglicherweise verschiedenartigen Felder  
  $\phi^{r_i}$ können dabei in verschiedene mathematische Räume abbilden und somit 
  Teilcheneigenschaften, insbesondere Transformationseigenschaften unter 
  Eichsymmetrien, darstellen. 

  \subsection{Das Standardmodell der Teilchenphysik}
    Das SM ist eine QFT nach dem Prinzip, invariant unter bestimmten 
    Symmetrietransformationen zu sein. Die Dynamik wird dabei durch die kombinierte 
    Eichgruppe $SU(3)\times SU(2)\times U(1)$ in der Lagrangedichte 
    repräsentiert und durch die Angabe 
    unter welcher Darstellung der Eichgruppe die Felder transformieren. 
    Neben den postulierten Feldern der Quarks, Leptonen und des Higgs wird so 
    die Existenz von Eichfeldern und Wechselwirkungen mit den übrigen Feldern 
    gefordert, welche keine Singletts der entsprechenden Eichgruppe sind.

    Im SM kennt man bisher sechs Quark-Flavour, \textit{up, down, 
    charm, strange, top (t), }und \textit{bottom}, die als 
    Dirac-Fermionen in der Lagrangedichte auftauchen. Ebenfalls als 
    Dirac-Fermionen werden die geladenen Leptonen, das \textit{Elektron}, 
    das \textit{Myon} und das \textit{Tauon} 
    eingeführt. Die drei \textit{Neutrinos} 
    kommen dagegen ausschließlich als linkshändige Weyl-Spinoren vor, sodass 
    sie im SM masselos sein müssen. Das einzige skalare Feld des SM ist 
    das \textit{Higgs-Feld}, welches im Higgsmechanismus für die 
    Brechung der Symmetriegruppe verantwortlich ist.
    
    Als eine der fundamentalen Kräfte des SM beschreibt die 
    QCD die Wechselwirkungen 
    zwischen Quarks, den Bausteinen der Hadronen, und den Gluonen, den 
    Eichfeldern der QCD. Die mathematische Beschreibung erfolgt durch 
    Darstellungen der $SU(3)$, bei der einem Quark $\psi^f$ mit Flavour $f$ ein 
    Colour-Triplett $\psi^f = \left(\psi^f_1, \psi_2^f, \psi_3^f\right)^\text{T}$ 
    zugeordnet wird. 
    Außerdem folgt die Existenz von acht masselosen Gluonen, die mit Quarks und 
    untereinander wechselwirken \cite{Ellis_Webber}. Eine genauere 
    mathematische Beschreibung folgt in Abschnitt \ref{QCDxdQCD}. 
    Charakteristisch für die QCD des SM ist die in Abschnitt \ref{beta_im_SM} 
    vorgestellte Energieabhängigkeit 
    der QCD Kopplungskonstanten. Der Pol bei $\Lambda_\text{QCD}$ ist 
    dabei für die hadronische Bindung bei niedrigen Energien verantwortlich 
    und somit insbesondere für die baryonische Massendichte $\Omega_\text{B}$ 
    im Universum, während es bei hohen Energien gerade zur asymptotisch 
    freien QCD kommt.
    
    Die Symmetriegruppe $SU(2)\times U(1)$ repräsentiert die Elektroschwache 
    Kraft. In ihr werden die Wechselwirkungen zwischen  
    linkshändigen Isospin-Dupletts 
    \begin{equation}
      \begin{pmatrix}
 u_L \\ d_L
\end{pmatrix}
\,,\quad
\begin{pmatrix}
 c_L \\ s_L
\end{pmatrix}
\,,\quad
\begin{pmatrix}
 t_L \\ b_L
\end{pmatrix}
\,,\quad
\begin{pmatrix}
 \nu_e\\ e^-_L 
\end{pmatrix}
\,,\quad
\begin{pmatrix}
  \nu_\mu \\\mu^-_L 
\end{pmatrix}
\,,\quad
\begin{pmatrix}
 \nu_\tau\\ \tau^-_L 
\end{pmatrix}
\,,

    \end{equation}
    über insgesamt vier Eichbosonen beschrieben. Durch den Higgsmechanismus 
    wird die Symmetriegruppe zu einer $U(1)$ gebrochen, welche als 
    Elektromagnetismus identifiziert werden kann. Als Eichbosonen der 
    gebrochenen Symmetrie entstehen die massiven Bosonen $W^+$, $W^-$ und $Z$ 
    sowie das masselose Photon.
    
    Im Higg-Sektor gibt es neben einem Potenzial $V[\phi]$ sogenannte 
    Yukawa-Kopplungen der 
    Form $Y_{ij} Q_L^i \phi q_R^j $
    zwischen einem Isospin-Duplett $Q_L^i$, dem Higgsfeld $\phi$ und einem 
    rechtshändigen Fermion $q_R^j$, mit der Kopplungsstärke $Y_{ij}$. Im 
    Higgsmechanismus wird das 
    Potenzial $V[\phi]$ minimiert und die $SU(2)\times U(1)$ Symmetrie 
    unterhalb einer charakteristischen Energieskala spontan gebrochen. Dies hat 
    die Massenerzeugung der Teilchen und das Flavour-Mixing der Quarks zur 
    Folge \cite{PDG:Higgs}.
    
    Für eine effektive QCD$\times$QED 
    hat Bednyakov gezeigt, dass ab der Energieskala der $t$-Masse die QED 
    korrekturen vernachlässigt werden können \cite{Bednyakov2015262}. Daraus 
    lässt sich schließen, dass es zulässig ist, den Elektroschwachen- und 
    den Higgs-Sektor nicht weiter zu betrachten. 
    
    

    
    
  \subsection{Gluon-Selbstenergie und laufende Kopplung}\label{selbstenergie}
    Nun werden 
    die wichtigsten Punkte bei der Berechnung von Korrelatorfunktionen mit 
    Hilfe von erzeugenden Funktionalen gezeigt. Ein umfassendes Bild 
    dazu ist in \cite{Schwartz} und \cite{Zinn} zu finden.
    Zunächst werden die erzeugenden Funktionale $Z$ und $W$ eingeführt um 
    die Gluon-Selbstenergie in einer vereinfachten Theorie zu berechnen. Aus der 
    Rechnung wird die 
    Notwendigkeit der Regularisierung und Renormierung abgeleitet. Schließlich 
    wird die $\beta$-Funktion der Beispieltheorie gefunden.
    \newpage
    Die Rechnung erfolgt am Beispiel einer $SU(N)$ mit  
    skalaren Materiefeldern $\phi_i$ in der Darstellung $S$ und 
    Eichfeldern $\A^A$ in \textit{adjungierter Darstellung}. Der Eichfixierungsterm 
    $- \frac{1}{2\xi} \partial_\mu 
     \A^{A\,\,\, \mu} \partial_\nu \A^{A\,\,\, \nu}$ wird vorerst 
     allgemein gelassen. Aus der 
    Lagrangedichte 
    \begin{equation}
    \begin{aligned}
     \mathcal{L}(\phi,\partial\phi,\mathcal{A},\partial\mathcal{A})=&
     \frac12 \left(\left[\i \delta_{ji}\partial_\mu-g t^A_{ji} \mathcal{A}^A_\mu
     \right]\phi_i\right)^*
     \left(\left[\i \delta_{ji}\partial^\mu-g t^A_{ji} \mathcal{A}^{A\,\,\,\mu}
     \right]\phi_i\right)
      \\& -
     \frac14 F^{A\,\,\, \mu\nu} F_{\mu\nu}^A 
     - \frac{1}{2\xi} \partial_\mu 
     \A^{A\,\,\, \mu} \partial_\nu \A^{A\,\,\, \nu}
     \end{aligned}
    \end{equation}
    mit $F_{\mu \nu}^A=\partial_\mu \mathcal{A}^A_\nu-
    \partial_\nu\mathcal{A}^A_\mu-g f^{ABC}\A^B_\mu \A^C_\nu$ lassen sich die Wirkungsfunktionale 
    \begin{align}
     S_\text{M}[\phi] &= \dx \frac12 \partial_\mu \phi_i \partial^\mu \phi_i \\
     S_\text{G}[\A] &= \dx \frac{1}{2} \partial_\mu \A^A_\nu 
     \partial^\nu \A^{A \, \nu} - \frac12 \partial_\mu \A_\nu^A 
     \partial^\mu \A^{A\, \nu} - \frac{1}{2\xi} \partial_\mu 
     \A^{A\, \mu} \partial_\nu \A^{A\, \nu} \\
     S_1[\phi,\A]&= \dx \i g  \partial^\mu \phi_j t_{ji}^A \phi_i \A_\mu^A 
    \end{align}
    ableiten. Gluon-Selbstwechselwirkungen werden nicht betrachtet.  
    Außerdem wurde eine reelle Darstellung $S$ gewählt, sodass $\phi^*=\phi$ und 
    $(t^A_{ji})^*=(-t^A_{ji})$. 
    Mit den Propagatoren 
    \begin{align}
     \Delta(x,y) &= \i \dppi  \frac{\e^{\i p(x-y)}}{p^2}  \\
     \Delta^{\rho \epsilon}(x,y) &= \frac{\i}{D}\dppi 
     \frac{\e^{\i p(x-y)}}{p^2}\left[ 
     \eta^{\rho \epsilon} -(1-\xi) \frac{p^\rho p^{\epsilon}}{p^2}\right]
     \label{eq:renormierung:gluonpropagator}
    \end{align}
    und dem erzeugenden Funktional 
    \begin{equation}
     Z[J,J_\mu]:= \int \mathcal{D}\phi \,\,\, \mathcal{D}\A \,\,\,
     \exp\left\{\i S_\text{M}[\phi]
     +\i S_\text{G}[\A]+\i S_1[\phi,\A] +\i J_i\phi_i+\i J^A_\mu 
     \A^{A\, \mu}\right\} \label{eq:renormierung:erzeugendes_funktional}
    \end{equation}
    gilt die wichtige Beziehung 
    \begin{equation}
     Z[J,J_\mu]=\exp\left\{\i S_1\left[\frac{\delta}{\i\delta J},\frac{\delta}{\i\delta J_\mu}\right]\right\} 
     \,\,\, \exp\left\{-\frac{1}{2} J_i \Delta J_i\right\} 
     \,\,\, \exp\left\{-\frac{1}{2} J^A_\mu \Delta^{\mu\nu} 
     J^A_\nu \right\} \quad . \label{eq:renormierung:funktional}
    \end{equation}
    Dabei wurden die erzeugenden Ströme $J_i(x)$ und $J_\mu^A(x)$ eingeführt, 
    sowie die Schreibweisen
    \begin{equation}
     J_i \phi_i = \dx J_i(x) \phi_i(x) \quad , \quad  J_i \Delta J_i =
     \dx \d^D y \,\,\,  J_i(x) \Delta(x,y) J_i(y) \quad ,
    \end{equation}
    und 
    \begin{equation}
     J_\mu^A \A^{A\,\mu} = \dx J_\mu^A(x) \A^{A\, \mu}(x) \quad , \quad  J_\rho^A 
     \Delta^{\rho \epsilon} J_\epsilon^A =
     \dx \d^D y \,\,\,  J_\rho^A(x) \Delta^{\rho\epsilon}(x,y) J_\epsilon^A(y) \quad .
    \end{equation}
    Durch das Ausführen von Funktionalableitungen in 
    \eqref{eq:renormierung:erzeugendes_funktional} lassen sich nun die 
    Korrelatorfunktionen aus \eqref{renormierung:Korrelator_definition} 
    rekonstruieren, 
    \begin{equation}
     \left.\frac{\delta}{\i \delta J_i(x)}\right|_{J=J_\mu=0} Z[J,J_\mu]
     = \Big\langle \phi_i(x) \Big\rangle \quad . 
     \label{eq:renormierung:ableitungen}
    \end{equation}
    Entsprechend können durch mehrmaliges Ableiten höhere Korrelatorfunktionen 
    berechnet werden. Die formale Entwicklung
    \begin{equation}
    \begin{aligned}
     &Z[J,J_\mu]= \sum\limits_{m,n=0}^\infty 
      \frac{\i^{m+n}}{m!n!} 
     \int \d^D x_1 \ldots \d^D x_m \d^D y_1 \ldots \d^D y_n   
     \\& Z^{(m,n)\,\,\,A_1\ldots A_n}_{i_1\ldots i_m\, , \,\mu_1\ldots\mu_n} 
     (x_1,\ldots,x_m,y_1,\ldots,y_n)      
      \,\,\,J_{i_1}(x_1)\ldots J_{i_m}(x_m) \,\,\,
     J^{A_1}_{\mu_1}(y_1)\ldots J^{A_n}_{\mu_n}(y_n)
    \end{aligned} \label{eq:zfunktional_entwicklung}
    \end{equation}
    erlaubt dabei eine kompakte Schreibweise.
    
    \subsubsection{Gluonpropagator in $\mathcal{O}(g^0)$ und 
    $\mathcal{O}(g^1)$}
      
      Es wird nun gezeigt, dass der Gluonpropagator als die 2-Punkt Funktion
      \begin{equation}
      \Big\langle \A^A_\mu(x_1)\A^B_\nu(x_2) 
      \Big\rangle
      =Z^{(0,2)\,\,\, AB}_{\mu\nu}(x_1,x_2)
      \end{equation}
      in $\mathcal{O}(g^0)$ verstanden werden kann. 
      In $\mathcal{O}(g^0 J_i^0 J_\rho^2)$ kann 
      \eqref{eq:renormierung:funktional} vereinfacht als 
      \begin{equation}
       Z[J,J_\rho] = -\frac{1}{2} J^A_{\rho} \Delta^{\rho \epsilon} J^A_\epsilon
      \end{equation}
      geschrieben werden, alle anderen Ordnungen werden verschwinden. Mit 
      \eqref{eq:renormierung:ableitungen}  
      gilt dann 
      \begin{equation}
      Z^{(0,2)\,\,\, AB}_{\mu\nu}(x_1,x_2) = 
       \frac{\delta}{ \delta J^A_\mu(x_1)}\frac{\delta}{ \delta J^B_\nu(x_2)}
       \frac{1}{2} J^C_{\rho} \Delta^{\rho \epsilon} J^C_\epsilon = \delta^{AB}
       \Delta_{\mu\nu}(x_1,x_2) \quad .
      \end{equation}
      Der Gluonpropagator aus \eqref{eq:renormierung:gluonpropagator} ist 
      demnach die 2-Punkt Funktion $Z^{(0,2)\,\,\, AB}_{\mu\nu}(x_1,x_2)$ in 
      $\mathcal{O}(g^0)$.
      
      Es lässt sich leicht nachrechnen, dass der 
      $\mathcal{O}(g^1)$ Beitrag zu $Z^{(0,2)\,\,\, AB}_{\mu\nu}(x_1,x_2)$ 
      verschwinden muss.

      
    \subsubsection{Gluonpropagator in $\mathcal{O}(g^2)$}

    Die einzige beitragende Ordnung ist hier $\mathcal{O}(g^2 J_i^4 J_\rho^4)$, 
    \begin{equation}
    \begin{aligned}
      Z[J,J_\mu]&= \frac12 g^2 t^C_{ji}t^D_{lk} \int \d^D y_1 \d^D y_2 
     \left(\partial_\rho \frac{\delta}{\i\delta J_j(y_1)} \right)  \frac{\delta}{\i \delta 
     J_i(y_1)} \frac{\delta}{\i \delta J^C_\rho(y_1)} \\&
     \left(\partial_\sigma \frac{\delta}{\i\delta J_l(y_2)} \right)   \frac{\delta}{\i \delta 
     J_k(y_2)} \frac{\delta}{\i \delta J^D_\sigma (y_2)}
%     \frac{\delta}{\i \delta J^A_\mu(x_1)} \frac{\delta}{\i \delta J^B_\nu(x_2)}
%      
    \frac18(J_m \Delta J_m)(J_n \Delta J_n) 
%      
    \frac18(J_\alpha^E \Delta^{\alpha \beta} J_\beta^E)
     (J_\gamma^F \Delta^{\gamma \delta} J_\delta^F) \quad .
    \end{aligned}
    \end{equation}
    Es folgt
    \begin{equation}
     \begin{aligned}
      &Z^{(0,2)\,\,\, AB}_{\mu\nu}(x_1,x_2) = \frac12 g^2 t^C_{ji}t^D_{lk}
      \int \d^D y_1 \,\,\, d^D y_2 \,\,\,
      \\ &
      \left[
      \Delta_{\mu\nu}(x_1,x_2) \Delta_{\rho \sigma}(y_1,y_2) \delta^{AB} 
      \delta^{CD}
      +
      \Delta_{\mu\rho}(x_1,y_1) \Delta_{\nu \sigma}(x_2,y_2) \delta^{AC} 
      \delta^{BC}
      \right. \\ & \quad \left.
      +
      \Delta_{\mu\sigma}(x_1,y_2) \Delta_{\nu \rho}(x_2,y_1) \delta^{AD} 
      \delta^{BC}
      \right]
      \\
      &\left[
      \Delta(y_1,y_2) \partial^\rho\partial^\sigma \Delta(y_1,y_2) \delta_{ki}
      \delta_{lj}
      +
      \partial^\rho \Delta(y_1,y_1) \partial^\sigma \Delta(y_2,y_2) \delta_{ij}
      \delta_{lk}
      \right. \\ & \quad  \left.
      +
      \partial^\sigma \Delta(y_1,y_2) \partial^\rho \Delta(y_2,y_1) \delta_{lj}
      \delta_{ik}
      \right] \quad .
     \end{aligned}\label{eq:renormierung:Og2}
    \end{equation}
    
    Um die Berechnung zu vereinfachen ist es nun hilfreich, ein neues 
    Funktional $W$ als\begin{equation}Z[J,J_\mu] = \exp\left\{W[J,J_\mu]\right\}
    \end{equation}
    einzuführen. Es gibt wieder eine formale Reihenentwicklung 
    \begin{equation}
    \begin{aligned}
     &W[J,J_\mu]= \sum\limits_{m,n=1}^\infty 
      \frac{\i^{m+n}}{m!n!} 
     \int \d^D x_1 \ldots \d^D x_m \d^D y_1 \ldots \d^D y_n   
     \\& W^{(m,n)\,\,\,A_1\ldots A_n}_{i_1\ldots i_m\, , \,\mu_1\ldots\mu_n} 
     (x_1,\ldots,x_m,y_1,\ldots,y_n)      
      \,\,\,J_{i_1}(x_1)\ldots J_{i_m}(x_m) \,\,\,
     J^{A_1}_{\mu_1}(y_1)\ldots J^{A_n}_{\mu_n}(y_n)
    \end{aligned}
    \end{equation}
    analog zu \eqref{eq:zfunktional_entwicklung}. Für 
    $Z^{(0,2)\,\,\, AB}_{\mu\nu}$ lässt sich die Beziehung 
    \begin{equation}
     Z^{(0,2)\,\,\, AB}_{\mu\nu}(x_1,x_2)=W^{(0,2)\,\,\, AB}_{\mu\nu} (x_1,x_2)
     +W^{(0,1)\,\,\, A}_\mu (x_1) W^{(0,1)\,\,\, B}_\nu (x_2)
    \end{equation}
    ableiten. Dabei erfüllt jede Funktion $W^{(m,n)}$ die Cluster-Eigenschaft 
    $W^{(m,n)} \longrightarrow 0$,
    wenn die erzeugenden Ströme $J$ und $J_\mu$ separierbare Träger haben. 
    Für \eqref{eq:renormierung:Og2} bedeutet das, dass die äußeren Punkte 
    $x_1, x_2$ und die Vertizes $y_1,y_2$ durch Propagatoren verbunden sein 
    müssen, um zu $W^{(0,2)\,\,\, AB}_{\mu\nu} (x_1,x_2)$ beizutragen. Man 
    spricht deshalb vom erzeugenden Funktional verbundener 
    Korrelatorfunktionen. Damit bleibt von \eqref{eq:renormierung:Og2}
    mit der Symmetrie unter 
    $y_1 \leftrightarrow y_2$ und der Antisymmetrie von $t_{ji}^A$
    \begin{equation}
    \begin{aligned}
    W^{(0,2)\,\,\, AB}_{\mu\nu} (x_1,x_2) &= -
    g^2 \text{Tr}(t^A t^B) \int \d^D y_1 \,\,\, d^D y_2 \,\,\, 
    \Delta_{\mu\rho}(x_1,y_1)\Delta_{\nu \sigma}(x_2,y_2)
    \\&\left[
       \Delta(y_1,y_2)
       \partial^\rho 
      \partial^\sigma \Delta(y_1,y_2) 
       -\partial^\rho 
      \Delta(y_1,y_2) \partial^\sigma \Delta(y_2,y_1)
    \right] \quad .
    \end{aligned}
    \end{equation}
    Die Fouriertransformation dieses Ausdrucks ergibt
    \begin{equation}
    \begin{aligned}
     \widetilde{W}^{(0,2)\,\,\, AB}_{\mu\nu} (k_1,k_2) = & -
      g^2 \text{Tr}(t^A t^B) 
     \frac{\delta(k_1+k_2)}{D^2 k_1^4} \int \frac{\d^D q}{(2\uppi)^D} 
     \left( \frac{q^\sigma q^\rho}{q^2(k_1+q)^2 } +
     \frac{q^\sigma (k_1+q)^\rho}{q^2 (k_1+q)^2} \right)\\ &
     \left( \eta_{\mu \rho}-(1-\xi)\frac{k_{1\,\mu} k_{1\,\rho}}{k_1^2}\right)
     \left( \eta_{\nu \sigma}-(1-\xi)\frac{k_{2\,\nu}k_{1\,\sigma}}{k_1^2}\right)
     \delta(k_1+k_2)
     \quad ,
    \end{aligned}
    \end{equation}
    wobei das Integral abhängig von $D$ divergent sein kann. Eine Berechnung mit 
    Hilfe der Schwinger-Parametrisierung ergibt 
   \begin{equation}
    \begin{aligned}
     &\widetilde{W}^{(0,2)\,\,\, AB}_{\mu\nu}  (k_1,k_2) = 
      g^2 \text{Tr}(t^A t^B)  
     \frac{\delta(k_1+k_2)}{D^2 k_1^4}
     %
     \Bigg\{
     \i
     (k_{1\,\text{E}}^2)^{\nicefrac{D}{2}-2}(4\uppi)^{-\nicefrac{D}{2}}  \\
     &     
     \left[
     \left(B(\nicefrac{D}{2},\nicefrac{D}{2}-1)-B(\nicefrac{D}{2}+1,\nicefrac{D}{2}-1) \right) k_1^\rho k_1^\sigma 
     - B(\nicefrac{D}{2},\nicefrac{D}{2}-1)k_{1\,\text{E}}^2
     \, \delta^{\rho\sigma} \frac{\Gamma(1-\nicefrac{D}{2})}{\Gamma(2-\nicefrac{D}{2})}
     \right] \\&
      \Gamma(2-\nicefrac{D}{2}) \Bigg\}
     %
     \left( \eta_{\mu \rho}-(1-\xi)\frac{k_{1\,\mu}k_{1\,\rho}}{k_1^2}\right)
     \left( \eta_{\nu \sigma}-(1-\xi)\frac{k_{2\,\nu}k_{2\,\sigma}}{k_2^2}\right)
      \delta(k_1+k_2)
     \quad .
    \end{aligned}
    \end{equation}
    In dimensionaler Regularisierung kann nun $D=4-2\epsilon$ mit einem 
    reellen $\epsilon>0$ gesetzt werden. Die Massendimension $[g]=(4-D)/2$ der 
    Kopplungskonstanten wird durch $g \to \mu^{(4-D)/2}g$ berücksichtig, wobei eine 
    unphysikalische Massenskala $\mu$ eingeführt wird.
    Den divergenten Term 
    $\widetilde{W}_\text{div}$ kann man nun als
    \begin{align}
     \widetilde{W}_\text{div}(k) &= \i \frac{g^2}{16\uppi^2} \text{Tr}(t^At^B)
       \frac{1}{6}\left[
     k^2_{ \text{E}}\, \delta_{\rho\sigma}- k_{\rho}k_{ \sigma} \right]
      \Gamma(\epsilon) \left( \frac{4\uppi \mu^2}{k_\text{E}^2}\right)^{\epsilon}
      \\
      &
      \simeq \i \frac{g^2}{16\uppi^2} \text{Tr}(t^At^B)
       \frac{1}{6}\left[
     k^2_{ \text{E}}\, \delta_{\rho\sigma} -k_{\rho}k_{\sigma} \right]
      \left( \frac{1}{\epsilon} + \ln\left( \frac{4\uppi \e^{-\gamma_\text{E}}
      \mu^2}{k_\text{E}^2} \right) \right) \label{eq:renormierung:W_div}
    \end{align}
    schreiben. Dabei wurden die Propagatoren der äußeren Gluonen absepariert und 
    die Feynmaneichung $\xi=1$ gewählt. Durch die Wahl $4\uppi 
    \e^{-\gamma_\text{E}}
      \mu^2\sim k_\text{E}^2$ verschwindet der logarithmische Term, und die 
      Massenskala $\mu$ kann mit der Energieskala des physikalischen Prozesses 
      identifiziert werden. Um die $1/\epsilon$ Divergenz interpretieren zu 
      können, werden renormierte Felder und Kopplungskonstanten gemäß
      \begin{equation}
       Z_\phi^{\nicefrac12}\phi_\text{R}=\phi \quad , \quad
       Z_\A^{\nicefrac12}\A_\text{R}=\A\quad , \quad 
       Z_g g_\text{R}=\mu^{(D-4)/2}g 
      \end{equation}
      eingeführt. Dabei sind $Z_\phi$, $Z_\A$ und $Z_g$ Renormierungskonstanten. 
      Damit wird \eqref{eq:renormierung:W_div} zu
      \begin{equation}
       \widetilde{W}_\text{div}(k)= \i \delta^{AB}\left[
      k^2_{ \text{E}}\, \delta_{\rho\sigma}- k_{\rho}k_{\sigma}\right]
      \left\{ 
      \frac{g_\text{R}^2}{16\uppi^2}  \frac{1}{6} T(S)     
      \frac{1}{\epsilon} +(Z_\A-1) \right\} \quad .
      \end{equation}
      Dabei wurde die Gruppenkonstante $T(S)d(S) \delta^{AB}=\text{Tr}(t^At^B)$ 
      und die Multiplizität $d(S)$ der Darstellung eingeführt.
      Die Renormierungskonstante $Z_\A$ ist nun
      \begin{equation}
       Z_\A = 1-\frac{g_\text{R}^2}{16 \uppi^2} \frac{1}{6} T(S)d(S) \quad .
      \end{equation}
      Auf gleiche Weise können $\widetilde{W}^{(2,0)}$ und 
      $\widetilde{W}^{(2,1)}$ berechnet werden, um die weiteren 
      Renormierungskonstanten zu bestimmen. Unter Vernachlässigung dieser 
      Beiträge kann jedoch 
      \begin{equation}
       Z_g = Z_\A^{\nicefrac12} \simeq 1+\frac{g_\text{R}^2}{16 \uppi^2}
       \frac{1}{12} T(S) d(S)
      \end{equation}
      benutzt werden. Es folgt 
      \begin{equation}
       \beta(g_\text{R}):= \mu \frac{\d g_\text{R}}{\d\mu} = \frac{g_\text{R}^3}
       {16 \uppi^2}
       \frac{1}{6} T(S)d(S) + \mathcal{O}(g_\text{R}^5) \quad .
       \label{eq:renormierung:beta-funktion}
      \end{equation}
      
	



    
    



