\clearpage
\section{Quantenfeldtheorie und Standardmodell}
  In einer \textit{Quantenfeldtheorie (QFT)} werden physikalische Entitäten 
  als Anregungszustände von Quantenfeldern 
  verstanden. Ein Schlüssel zu den experimentell zugänglichen 
  Wirkungsquerschnitten ist die Berechnung von \textit{Korrelatorfunktionen} 
  oder \textit{$n$-Punkt Funktionen}
  \begin{equation}
   \Big\langle \textbf{T} \,\,\, \phi^{r_1}(x_1) \ldots \phi^{r_N}(x_N) 
   \Big\rangle
   =\frac{\int \mathcal{D}\phi \,\,\, \phi_1(x_1)\ldots \phi_N(x_N) 
   \e^{\i S[\phi]}}{\int \mathcal{D}\phi \,\,\, \phi^{r_1}(x_1) \ldots 
   \phi^{r_N}(x_N) }
   \label{renormierung:Korrelator_definition}
  \end{equation}
  im Pfadintegralformalismus von Feynman \cite{Schwartz}. Die Information über 
  die möglichen 
  physikalischen Prozesse der Quantenfelder $\phi$ 
  ist dabei in dem Wirkungsfunktional $S$ bzw. der Lagrangedichte $\mathcal{L}$ 
  enhalten, welche in einer $d$-Dimensionalen Raumzeit über 
  \begin{equation}
    S[\phi] = \int \d^d x \,\,\, \mathcal{L}(\phi,\partial \phi, t) 
    \label{renormierung:Wirkung_definition}
  \end{equation}
  verknüpft sind \cite{Schwartz}. Die Felder $\phi$ können dabei qualitativ 
  unterschiedlich sein, d.h. in verschiedene Räume abbilden und somit 
  verschiedene Teilcheneigenschaften darstellen. Dies soll durch den Index 
  $r$ verdeutlicht werden.

  \subsection{Das Standardmodell der Teilchenphysik}
    Das SM ist eine QFT nach dem Prinzip, invariant unter bestimmten 
    Symmetrietransformationen zu sein. Die Dynamik wird dabei durch die 
    Eichgruppe $SU(3)\times SU(2)\times U(1)$ in \\der Lagrangedichte 
    repräsentiert sowie der Angabe, 
    unter welcher Darstellung der Eichgruppe die Felder transformieren. 
    Neben den Feldern der Quarks, Leptonen und des Higgs wird so die 
    Existenz von Eichfeldern und Wechselwirkungen mit den übrigen Feldern 
    gefordert, welche keine Singletts der entsprechenden Eichgruppe sind.

    Quarkfelder unterscheiden sich unter anderem in ihren Quantenzahlen 
    Flavour und Colour. Im SM ist die Anzahl verschiedener 
    Flavour $N_\text{Flavour}=6$ und die Colour Anzahl $N_\text{Colour} = 3$. 
    Sie transformieren als Colour-Tripletts unter der fundamentalen Darstellung 
    der $SU(3)$. Als Yang-Mills Theorie werden außerdem $\Nc^2-1 = 8$ 
    Eichfelder, die Gluonen, sowie Wechselwirkungen gefordert \cite{Zinn}. 
    Diese Wechselwirkung wird  als \textit{Quantenchromodynamik (QCD)} oder 
    \textit{starke Wechselwirkung} bezeichnet.
    
    
  \subsection{Effektive Quantenwirkung und Gell-Mann-Low Gleichung}
    Wechelwirkungen von Quantenfeldern ( o.E. $\phi_1$, $\phi_2$ und $\phi_3$) 
    werden in der 
    Lagrangedichte über Terme der Form $\mathcal{L}_\text{int} 
    \propto g \phi_1 \phi_2 \phi_3$ erzeugt. Dabei stellt $g$ die 
    \textit{Kopplungskonstante} dieser Wechselwirkung dar. Eine elegante 
    Möglichkeit $n$-Punkt Funktionen zu berechnen bietet das \textit{erzeugende 
    Funktional}
    \begin{equation}
    Z[J] := \int \mathcal{D}\phi \,\,\, \e^{\i S[\phi]+ \i S_\text{int}[\phi] +
    \i \sum\limits_{r=1}^n
    \int \d^d x \, \phi^r(x)J^r(x)} \quad ,
    \label{renormierung:erzeugendes_Funktional_definition} 
    \end{equation}
    mit dem Gleichung \eqref{renormierung:Korrelator_definition} als 
    Funktionalableitung
    \begin{equation}
    \left[ \frac{\partial}{\partial J^{r_1}(x_1)} \ldots 
    \frac{\partial}{\partial J^{r_N}(x_N)}  Z[J] \right]_{J=0} =
    (-\i)^k \Big\langle \textbf{T} \,\,\, \phi^{r_1} (x_1) \ldots 
    \phi^{r_N} (x_N)\Big\rangle 
    \end{equation}
    geschrieben werden kann \cite{Schwartz}. Dazu wird für jede Art von 
    Feld $\phi^r$ ein erzeugender Strom $J^r$ eingeführt. In der praktischen 
    Berechnung wird nun die Beziehung
    \begin{equation}
     Z[J] = \e^{S_\text{int}\left[\frac{\partial}{\partial J}\right]} 
      \int \mathcal{D}\phi \,\,\, \e^{\i S[\phi]+
    \i \sum\limits_{r=1}^n
    \int \d^d x \, \phi^r(x)J^r(x)} 
    \end{equation}


