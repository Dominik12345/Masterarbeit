\subsection{Laufende Kopplungen in der Standardmodell QCD}\label{beta_im_SM}
	Im Folgenden werden stets renormierte Größen betrachtet, sodass der 
	entsprechende Index ab jetzt unterdrückt wird.
	
	Mit $T(S)=1/2$ für Darstellungen einer $SU(N)$ folgt aus 
   \eqref{eq:renormierung:beta-funktion} 
   \begin{equation}
	\beta(g)>0 \quad \text{und} \quad    
   g(\mu) \longrightarrow \infty
   \label{eq:beta_im_SM:landau_pol}
   \end{equation}
   für steigendes $\mu$. Hier wird es sogar einen Landaupol geben, d.h. 
   $g(\mu)\to \infty$ wenn $\mu\to \mu_\text{Landau}<\infty$.  
   
   Eine Berechnung der vollen $\beta$-Funktion in 
   $\mathcal{O}(g^3)$ ergibt 
   \cite{Luo_Wang_Xiao} 
   \begin{equation}
   \beta(g)=\frac{g^3}{16\uppi^2} \left( \frac{4}{3} T(F)d(F) +
   \frac{1}{6} T(S) d(S) - \frac{11}{3} C_2(SU(N)) \right)\quad .
   \label{eq:beta_im_SM:volle_beta}
   \end{equation}
   Dabei ist $F$ die Darstellung der Fermionen und $C_2(SU(N))=N$ der 
   quadratische Casimiroperator der adjungierten Darstellung. Im SM 
   gibt es $d(S)=0$ Skalare mit Colour-Ladung, $d(F)=n_\text{f}$ Quarks und 
   es ist $N=3$. Unter der Bedingung $33/2  >  n_\text{f}$ gilt nun 
	\begin{equation}
	\beta(g)<0 \quad \text{und somit} \quad g(\mu) \longrightarrow 0
	\end{equation}	   
	für steigendes $\mu$. Für die $n_\text{f}=6$ bekannten Quarks ist diese 
	Bedingung erfüllt, das heißt Quarks sind für hohe Energieskalen 
	asymptotisch frei. 
	Nun tritt jedoch ein Pol für niedrige Energieskalen bei $\Lambda_\text{QCD}$ 
	auf. Diese Skala ist als Confinement Scale bekannt.	
	An Gleichung \eqref{eq:beta_im_SM:volle_beta} 
	sieht man außerdem, dass das Einführen weiterer Materiefelder 
	die $\beta$-Funktion in den positiven Bereich verschieben kann. Für 
	UV-finite Kopplungskonstanten gibt es also Obergrenzen an die Anzahl der 
	Teilchen, abhängig von der Gruppe und Darstellung der Teilchen.
	
   
   