\clearpage
\section{Experimentelle Nachweisbarkeit}

  Abschließend soll nun diskutiert werden, inwieweit die Einführung der dQCD 
  Abweichungen von Standardmodell bewirkt. Sei $M$ eine charakteristische 
  Massenskala des joint-Sektors. Da diese Teilchen QCD-Ladung tragen, ist 
  $M > \mathcal{O}(100\,\text{GeV})$, sonst wären diese in aktuellen 
  Experimenten bereits gefunden worden 
  \cite{Scale_of_dark_QCD}\cite{Becciolini:2014lya}. In 
  \cite{Becciolini:2014lya} und \cite{Sannino} wird das Verhältnis 
  der differenziellen Wirkungsquerschnitte von 3- und 2-Jet Zerfällen als 
  sinvolle Observable für hochenergie QCD eingeführt
  und daran argumentiert, dass 
  die \textit{Partonverteilungsfunktionen (PDF)} der neuen Teilchen sowie die 
  Auswirkungen auf die PDFs der SM Teilchen vernachlässibar sind. Sie zeigen 
  weiter, dass solche Verhältnisse nicht sensitiv auf die Gluon PDF sind und 
  schließen daraus, dass das Laufen von $\alpha_\text{QCD}$ den wichtigsten 
  Einfluss auf solche Observablen hat. Aus diesem Grund wird nun das Laufen 
  von $\alpha_\text{QCD}$ im SM mit $\alpha_1$ aus der \QCDxdQCD verglichen. 
  
  Die bisher untersuchten $\beta$-Funktionen gelten nur für Energien oberhalb 
  der Massenskala $M$ der schwersten neuen Teilchen \cite{Becciolini:2014lya}, 
  unterhalb. 
  
  
   