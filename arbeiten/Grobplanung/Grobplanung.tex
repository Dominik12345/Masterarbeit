\documentclass[
  11pt, 
  a4paper,
  fleqn,
  ngerman,
  parskip,
  toc=bibliography
  ]{scrartcl}


\usepackage{amsmath, amssymb, graphics, setspace}

% font
\usepackage{lmodern}
\usepackage{fourier}
%\usepackage{tgheros}
%\usepackage{tgcursor}
\usepackage{tgpagella}
\usepackage{microtype}

% math
\usepackage{amsmath}
\usepackage{amsfonts}
\usepackage{amssymb}
\usepackage{amstext}
%\usepackage{isomath}
\usepackage{mathtools}
\usepackage{units}

% encoding
\usepackage[T1]{fontenc}
\usepackage[utf8]{inputenc}

% language
\usepackage{ngerman}
% useful stuff
\usepackage{enumitem}
\usepackage{booktabs}











\theoremstyle{plain}

\newtheorem{theorem}{Theorem}[section]
\newtheorem{lemma}[theorem]{Lemma}
\newtheorem{proposition}[theorem]{Proposition}
\newtheorem{korollar}[theorem]{Korollar}


\theoremstyle{definition}

\newtheorem{definition}[theorem]{Definition}
\newtheorem{beispiel}[theorem]{Beispiel}




\titleformat*{\section}{\Large\bfseries}
\titleformat*{\subsection}{\large\bfseries}


\renewcommand{\d}{\text{d}}
\newcommand{\ddt}{\frac{\d}{\d t}}
\newcommand{\e}{\text{e}}


\newcommand{\dbda}{\frac{\partial \beta}{\parital \alpha}}
\newcommand{\dbdafix}{\left.\frac{\partial \beta}{\partial \alpha}\right|_*}
\newcommand{\Sp}{ \text{Sp} \frac{\partial \beta}{\partial \alpha}}
\newcommand{\Det}{ \text{Det} \frac{\partial \beta}{\partial \alpha}}



\newcommand{\Nc}{N_\text{c}}
\newcommand{\Nd}{N_\text{d}}
\newcommand{\nfc}{n_{\text{f}_\text{c}}}
\newcommand{\nsc}{n_{\text{s}_\text{c}}}
\newcommand{\nfd}{n_{\text{f}_\text{d}}}
\newcommand{\nsd}{n_{\text{s}_\text{d}}}
\newcommand{\nfj}{n_{\text{f}_\text{j}}}
\newcommand{\nsj}{n_{\text{s}_\text{j}}}

\newcommand{\QCDxdQCD}{QCD$\times$dQCD}
\newcommand{\Ms}{M_\text{s}}
\newcommand{\Mc}{M_\text{c}}







\begin{document}






\section{Inhaltsschwerpunkte}
	\begin{enumerate}
		\item \label{part:Einleitung}	
			Einleitung
		\item \label{part:running_coupling}
			Renormierung und running coupling (genauere Betrachtung)
		\item \label{part:Betafunktion}
			Allgemeine $\beta$-Funktionen
		\item \label{part:Betafunktion im R2}
			$\beta$-Funktion im $\mathbb{R}^2$
		\item \label{part:Betafunktion fuer QCDxdQCD}
			$\beta$-Funktion für QCD$\times$dQCD
		\item \label{part:Diagramme}
			Entwicklung von n-Schleifen-Diagrammen
	\end{enumerate}






\section{Inhalte}
	\begin{description}
	
		\item[zu \ref{part:Einleitung}]
			\begin{itemize}
				\item	Ideen der QFT (inkl. Funktionalintegral, 
						LSZ-Formel, connected functional, proper 
						vertices, Feynmandiagramme)
				\item	das SM der Teilchenphysik
				\item	mögliche Erweiterungen: dunke Materie und 
						dQCD
				\item	Renormierung und Renormierbarkeit
				\item	running couplings im SM, insb. QCD
				\item 	Problem der Quantengravitation und Weinbergs 
						Lösungsvorschlag
				\item	asymptotic safety
			\end{itemize}
			
		\item[zu \ref{part:running_coupling}]
			\begin{itemize}
				\item	Quantenwirkung
				\item	Cut-off Regularisierung
				\item	Callan-Symanzik Gleichung
				\item 	running coupling als Konsequenz der 
						Renormierung
				\item	asymptotic safety Szenario
			\end{itemize}
			
		\item[zu \ref{part:Betafunktion}]
			\begin{itemize}
				\item	Allgemeine Parametriesierung von 
						$\beta$-Funktionen für bestimmte 
						Eichgruppen, Mahacheck-Vaughn
				\item 	Autonomes System, RG-Zeit, $\alpha$
				\item	Fixpunkte, Landaupole
				\item 	Linearisierung, Stabilität von Fixpunkten, 
						Lösungsideen
				\item 	Kritische Hyperflächen
			\end{itemize}
			
		\item[zu \ref{part:Betafunktion im R2}]
			\begin{itemize}
				\item	Stabilitätsmatrix, Eigenwerte, 
						Eigenvektoren
				\item	Entwicklung 2.Ordnung um einen 
						Fixpunkt
				\item	Explizite (asymptotische) Lösung in der 
						Nähe eines Fixpunktes
				\item 	Stabilitätskriterien
				\item	Landaupole
			\end{itemize}
			
		\item[zu \ref{part:Betafunktion fuer QCDxdQCD}]
			\begin{itemize}
				\item	Explizite Form der $\beta$-Funktion
				\item	Fixpunkte und Landaupole
				\item	Bedingung an die Koeffizienten für 
						bestimmte Stabilitätseigenschaften
				\item 	Einschränkung auf sinnvolle 
						Materieinhalte
				\item	Anwendung auf fundamental$\times$fundamental 
						Darstellung ohne/mit Skalare.
				\item 	Extrapolation der Fixpunkte
			\end{itemize}				
			
			
		\item[zu \ref{part:Diagramme}]
			\begin{itemize}
				\item	Einschränkung auf QCD$\times$dQCD für 
						eine explizite Darstellung
				\item	$1$-Schleife (Propagator und Vertex)
				\item	$n+1$-Schleife (Rekursion)
				\item	Form (insb. $g_i(\mu)$ Abhängigkeit) der 
						$\beta$-Funktion durch Diagramme
				\item 	Genaue Form der $\beta$-Funktion
			\end{itemize}
		
		
			
	\end{description}




\clearpage
\section{gewonnene Erkenntnisse (Ideen)}

	\subsection{1-dim. $\beta$-Funktion und Landau-Pol}
		
		Für eine $\beta$-Funktion der Form 
		\begin{equation}
		\beta(\alpha)=\sum\limits_{i=0}^N X_i \alpha^i
		\end{equation}
		definiere die Stammfunktion
		\begin{equation}
		\mathfrak{A}(\alpha):=\int \left( \sum\limits_{i=0}^N X_i 
		\alpha^i\right)^{-1} \text{ d}\alpha \quad .
		\end{equation}
		Mit der Wahl $t_0=0$ und $\alpha_0=\alpha(0)$ beliebig lässt 
		sich die implizite Gleichung
		\begin{equation}
		t+\mathfrak{A}(\alpha_0)=\mathfrak{A}(\alpha)
		\end{equation}
		ableiten. Der Landau-Pol wird mit dem Startwert $(t_0,
		\alpha_0)$ dann bei 
		\begin{equation}
		t_\infty := \lim\limits_{\alpha\to \infty}\mathfrak{A}(\alpha)
		 - \mathfrak{A}
		(\alpha_0) \label{eq:t_inf}
		\end{equation}
		erreicht.
		
		\begin{example}
			\begin{equation}
			\beta = X \alpha \label{eq:beta_beispiel1}
			\end{equation}
			dann ist 
			\begin{equation}
			\mathfrak{A}=\frac{1}{X}\ln(\alpha)
			\end{equation}
			und 
			\begin{equation}
			t_\infty = \lim\limits_{\alpha\to\infty} \frac1X 
			\ln(\alpha) -\mathfrak{A}(\alpha_0) = \text{sgn}(X) 
			\cdot \infty \quad .
			\end{equation}
			Die explizite Lösung \eqref{eq:beta_beispiel1} kann 
			als $\alpha(t)=\alpha_0 \text{e}^{Xt}$ geschrieben werden. 
			Das Verhalten entspricht also dem von \eqref{eq:t_inf} 
			vorhergesagten.
		\end{example}
		
		\begin{example}
			\begin{equation}
			\beta = X \alpha^n \quad , \ n\geq 2 
			\label{eq:beta_beispiel2}
			\end{equation}
			dann ist 
			\begin{equation}
			\mathfrak{A}= \frac{1}{(1-n)X} \alpha^{1-n}
			\end{equation}
			und 
			\begin{equation}
			t_\infty = \lim\limits_{\alpha\to\infty}
			\frac{1}{(1-n)X} \alpha^{1-n} - \mathfrak{A}(\alpha_0)
			=-\mathfrak{A}(\alpha_0)=-\frac{1}{(1-n)X} \alpha_0^{1-n}.
			\label{eq:t_inf_beispiel2} 
			\end{equation}
			Die allgemeine Lösung von \eqref{eq:beta_beispiel2} ist 
			\footnote{Wolfram Alpha}
			\begin{equation}
			\alpha(t) =\left[(n-1) (c_1-t X)\right]^\frac{1}{1-n} 
			\label{eq:beta_loesung_beispiel2}
			\end{equation}
			mit Startwert $\alpha_0:=\alpha(0)=\left((n-1)c_1\right)
			^\frac{1}{1-n}$. Setzten wir diesen Starwert in 
			\eqref{eq:t_inf_beispiel2} ein, ergibt sich
			\begin{equation}
			t_\infty = -\frac{1}{(1-n)X}(n-1)c_1 = \frac{c_1}{X} \quad. 
			\end{equation}
			Auch hier hat die explizite Lösung 
			\eqref{eq:beta_loesung_beispiel2} genau an diesem Wert 
			einen Pol.
		\end{example}		
		
		\begin{example}
			\begin{equation}
			\beta = X \alpha^2 + Y\alpha^3 \quad , \, Y>0,\, X<0
			\end{equation}
			dann ist 
			\begin{equation}
			\mathfrak{A}=-\frac{Y \ln \alpha}{X^2}+
			\frac{Y \ln(Y\alpha +X)}{X^2}-\frac{1}{X\alpha}
			\end{equation}
			und 
			\begin{equation}
			t_\infty = \lim\limits_{\alpha\to\infty}\mathfrak{A} 
			-\mathfrak{A}(\alpha_0)= 
			\frac{Y}{X^2}\ln Y-\mathfrak{A}(\alpha_0) \quad .			
			\end{equation}
			Für $\alpha=-X/Y+\epsilon$ ergibt sich			
			\begin{align}
			-\mathfrak{A}(\alpha)&=\frac{Y\ln \alpha}{X^2}
			-\frac{Y \ln(Y\epsilon)}{X^2}-\frac{Y}{X^2}		\\
			&= \frac{Y}{X^2}\left(\ln\frac{\alpha}{\epsilon}\right) 
			-\frac{Y \ln Y}{X^2}-\frac{Y}{X^2}
			\end{align}
			und damit für $t_\infty$
			\begin{equation}
			t_\infty = \frac{Y}{X^2}\left(  \ln \frac{\alpha}{
			\epsilon}-1\right) \quad ,
			\end{equation}			
			wobei wegen $\ln \alpha/\epsilon = \ln (-XY^{-1}+
			\epsilon)/\epsilon$			
			
			
			Eine numerische Lösung mit $X=$, $Y=$ und $\alpha_0=$ 
			zeigt das in Abbildung \ref{fig:beta_beispiel2} 
			dargestellte Verhalten.
		\end{example}
		


		
		
		
\bibliography{/home/dkahl/Documents/Masterarbeit/arbeiten/Masterarbeit/bib}

\bibliographystyle{plain}

\end{document}