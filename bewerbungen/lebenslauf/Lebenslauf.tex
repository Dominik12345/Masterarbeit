%------------%
%   Header   %
%------------%

\documentclass[margin,12pt]{res} % Use the res.cls style, the font size can be changed to 11pt or 12pt here

\usepackage[T1]{fontenc}
\usepackage[utf8]{inputenc}

\usepackage{amsmath, amssymb, graphics, setspace}
\usepackage{amsfonts}
\usepackage{amstext}
\usepackage{lmodern}

\setlength{\textwidth}{4.5in} % Text width of the document

\begin{document}


%----------------------------------------------------------------------------------------
%	NAME AND ADDRESS SECTION
%----------------------------------------------------------------------------------------

\moveleft.5\hoffset\centerline{\large\bf Dominik Thilo Kahl} % Your name at the top
 
\moveleft\hoffset\vbox{\hrule width\resumewidth height 1pt}\smallskip % Horizontal line after name; adjust line thickness by changing the '1pt'

 
\moveleft.5\hoffset\centerline{Lebenslauf} 
%\moveleft.5\hoffset\centerline{Harkorthof 6} % Your address
%\moveleft.5\hoffset\centerline{45659 Recklinghausen}
%\moveleft.5\hoffset\centerline{dominik.kahl2@gmail.com}

%----------------------------------------------------------------------------------------

\hspace{0.2in}
\begin{resume}

%----------------------------------------------------------------------------------------
%	OBJECTIVE SECTION
%----------------------------------------------------------------------------------------
 
\section{PERSON}  
\begin{tabular}{ll}
%{\sl Name}& Dominik Thilo Kahl \\
{\sl Geboren am}& 19. Mai 1992 \\
{\sl in}& Recklinghausen		  \\
{\sl Familienstand}& ledig		\\
{\sl Adresse}& Harkorthof 6 \\ &45659 Recklinghausen \\
{\sl E-Mail} &dominik.kahl2@gmail.com
\end{tabular}

%----------------------------------------------------------------------------------------
%	EDUCATION SECTION
%----------------------------------------------------------------------------------------
\section{ABSCHLÜSSE}
{\sl Abitur} \\
Marie-Curie-Gymnasium Recklinghausen, 2011\\\\
{\sl Bachelor of Science} Physik \\
Technische Universität Dortmund, 2014 
%Thema der Bachelorarbeit: Charmoniumbeiträge zu $\overline{B}\to 
%\overline{K}^* 
%l^+ l^-$ 
\\\\
{\sl Master of Science} Physik \\
Technische Universität Dortmund, voraussichtlich Januar 2017\\
Arbeitsgebiet: Theoretische Teilchenphysik, Flavourphysik, Eichfeldtheorie
%Thema der Masterarbeit: UV-Fixpunkte einer $SU(N_\text{c})\times 
%SU(N_\text{d})$ Eichtheorie 
%----------------------------------------------------------------------------------------
%	SKILLS SECTION
%----------------------------------------------------------------------------------------

\section{FACHLICHE \\ FÄHIG-\\KEITEN} 

{\sl Computerfähigkeiten:} 
Grundlegende Kenntnisse in Java, Mat\-lab, Root, C++. \\
Fortgeschrittene Kenntnisse im Umgang mit Python, Ma\-the\-ma\-ti\-ca. 
\\\\
{\sl Mathematisch:}  
Schwerpunkt Quantenfeldtheorie und Differenzialgeometrie.

\section{WEITERE \\ FÄHIG-\\KEITEN} 
Fundierte Englischkentnisse \\
Fachsprachliches Englisch: Mathematik und Physik \\
Führerschein Klasse B

%----------------------------------------------------------------------------------------
%	PROFESSIONAL EXPERIENCE SECTION
%----------------------------------------------------------------------------------------
 
\section{ERFAHRUNG}

{\sl Lehre:} Gruppenleitung als Tutor zu den Themen klassische Me\-cha\-nik, Theorie des Elektromagnetismus, nicht-relativistische Quantenmechanik und spezielle Relativitätstheorie.
%----------------------------------------------------------------------------------------
%	COMMUNITY SERVICE SECTION
%---------------------------------------------------------------------------------------- 

\section{AUßER\-FACHLICHES }

Langjährige ehrenamtliche Arbeit mit Jugendgruppen in leitender Position. Erfahrung in der Leitung von Kleingruppen und Organisation größerer Events.
%----------------------------------------------------------------------------------------
%	EXTRA-CURRICULAR ACTIVITIES SECTION
%----------------------------------------------------------------------------------------

%----------------------------------------------------------------------------------------

\end{resume}


\end{document}