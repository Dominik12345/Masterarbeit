%------------%
%   Header   %
%------------%

\documentclass[margin,12pt]{res} % Use the res.cls style, the font size can be changed to 11pt or 12pt here

\usepackage[T1]{fontenc}
\usepackage[utf8]{inputenc}

\usepackage{amsmath, amssymb, graphics, setspace}
\usepackage{amsfonts}
\usepackage{amstext}
\usepackage{lmodern}

\setlength{\textwidth}{4.5in} % Text width of the document

\begin{document}


%----------------------------------------------------------------------------------------
%	NAME AND ADDRESS SECTION
%----------------------------------------------------------------------------------------

\moveleft.5\hoffset\centerline{\large\bf Dominik Thilo Kahl} % Your name at the top
 
\moveleft\hoffset\vbox{\hrule width\resumewidth height 1pt}\smallskip % Horizontal line after name; adjust line thickness by changing the '1pt'

 
\moveleft.5\hoffset\centerline{Curriculum Vitae} 
%\moveleft.5\hoffset\centerline{Harkorthof 6} % Your address
%\moveleft.5\hoffset\centerline{45659 Recklinghausen}
%\moveleft.5\hoffset\centerline{dominik.kahl2@gmail.com}

%----------------------------------------------------------------------------------------

\hspace{0.2in}
\begin{resume}

%----------------------------------------------------------------------------------------
%	OBJECTIVE SECTION
%----------------------------------------------------------------------------------------
 
\section{PERSONAL \\ INFORMA\-TION}  
\begin{tabular}{ll}
%{\sl Name}& Dominik Thilo Kahl \\
{\sl Date of birth}& 19. Mai 1992 \\
{\sl City of birth}& Recklinghausen, Germany		  \\
{\sl Family status}& ledig		\\
{\sl Current adress}& Harkorthof 6 \\ &45659 Recklinghausen, Germany \\
{\sl E-mail} &dominik.kahl2@gmail.com
\end{tabular}

%----------------------------------------------------------------------------------------
%	EDUCATION SECTION
%----------------------------------------------------------------------------------------
\section{ACADEMIC \\ DEGREES}
{\sl Abitur} \\
Marie-Curie-Gymnasium Recklinghausen, 2011\\\\
{\sl Bachelor of Science} Physics \\
Technische Universität Dortmund, 2014 
%Thema der Bachelorarbeit: Charmoniumbeiträge zu $\overline{B}\to 
%\overline{K}^* 
%l^+ l^-$ 
\\\\
{\sl Master of Science} Physics \\
Technische Universität Dortmund, expected january 2017\\
Specialisation on theoretical particle physics.
%Thema der Masterarbeit: UV-Fixpunkte einer $SU(N_\text{c})\times 
%SU(N_\text{d})$ Eichtheorie 
%----------------------------------------------------------------------------------------
%	SKILLS SECTION
%----------------------------------------------------------------------------------------

\section{TECHNICAL \\ SKILLS} 

{\sl programming skills:} 
Basic knowledge of Java, Python, Ma\-the\-ma\-ti\-ca, Mat\-lab, Root. \\\\
{\sl mathematical skills:}  
Concentration on Quantum Fieldtheory and differential geometry.

\section{FURTHER\\ SKILLS} 
Profound knowledge of English.			\\
Technical language: mathematics and physics.	\\


%----------------------------------------------------------------------------------------
%	PROFESSIONAL EXPERIENCE SECTION
%----------------------------------------------------------------------------------------
 
\section{EXPERIENCE}

{\sl teaching:} Teaching experience as tutor at University on the 
to\-pics of classical mechanics, theory of elektromagnetism, non relativistic quantum mechanics and theory of special relativity.
%----------------------------------------------------------------------------------------
%	COMMUNITY SERVICE SECTION
%---------------------------------------------------------------------------------------- 

\section{NON-TECHNICAL \\ EXPERTISE}

Many years of voluntary work with youth groups, experienced in leading 
small groups and organizing greater events.
%----------------------------------------------------------------------------------------
%	EXTRA-CURRICULAR ACTIVITIES SECTION
%----------------------------------------------------------------------------------------

%----------------------------------------------------------------------------------------

\end{resume}


\end{document}