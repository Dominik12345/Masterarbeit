%------------%
%   Header   %
%------------%

\documentclass{scrartcl}

\usepackage[T1]{fontenc}
\usepackage[utf8]{inputenc}

\usepackage[T1]{fontenc}
\usepackage{lmodern}

\setlength{\textwidth}{5.1in} % Text width of the document

%---------------%
% Article Style %
%---------------%
\begin{document}
\pagestyle{empty}
%------------%
% Sections   %
%------------%

\section*{Bewerbung
auf die Stelle\\ Entwicklungsingenieur Modellentwicklung 
für Echtzeitfahrdynamiksimulation}

\vspace{0.5in}

%ANREDE
Sehr geehrte Damen und Herren,
\newline\newline
%TEXT 
an Ihrem Stand auf der Unternehmenskontaktmesse \textit{konaktiva} in Dortmund wurde  
ich über die Berufsmöglichkeiten bei EFS informiert und mein Interesse geweckt, 
mich als Berufseinsteiger auf die ausgeschriebene Stelle Entwicklungsingenieur 
Modellentwicklung 
für Echtzeitfahrdynamiksimulation zu bewerben.
\newline\newline
Ich studiere zur Zeit Physik an der Technischen Universität Dortmund und werde 
voraussichtlich im Januar 2017 den Abschluss Master of Science Physik erwerben.
Das Vertiefungsgebiet während meines Masterstudiums war die theoretische 
Physik. Das Erfassen experimenteller Ergebnisse mittels ma\-the\-ma\-tischer 
Methoden gehörte dabei zu den Kernkompetenzen, ebenso wie die Mo\-del\-lie\-rung neuer mathematischer Modelle und die Simulation von phy\-si\-ka\-li\-schen 
Prozessen. 
Physikalische Experimente finden in idealisierten, von Fremdeinflüssen  
abgeschnittenen Umgebungen statt, im Gegensatz dazu ist es für mich eine 
spannende 
Herausforderung, diese Konzepte auf das Anwendungsgebiet von Fahrzeug und 
Straße zu übertragen. 
Ich würde mich freuen, diese Herausforderung bei EFS 
antreten und meine Fähigkeiten aktiv einbringen zu können. 
Gerne möchte ich Sie in einem persönlichen Gespräch hiervon überzeugen.
\newline\newline
Diesem Bewerbungsschreiben sind ein tabellarischer Lebenslauf und eine Kopie 
des Bachelor-Prüfungszeugnisses beigefügt.
\newline\newline\newline
\begin{flushright}
\begin{tabular}{c}
mit freundlichen Grüßen, \\ \\
Dominik Thilo Kahl \\ \\
Recklinghausen, den 12.11.2016 \\
\end{tabular}
\end{flushright}

\end{document}