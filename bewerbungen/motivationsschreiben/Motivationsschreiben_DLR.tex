%------------%
%   Header   %
%------------%

\documentclass[12pt]{scrartcl}

\usepackage[T1]{fontenc}
\usepackage[utf8]{inputenc}

\usepackage{ngerman}
\usepackage{lmodern}


\setlength{\parindent}{0pt}
%---------------%
% Article Style %
%---------------%
\begin{document}
\pagestyle{empty}
%------------%
% Sections   %
%------------%

\section*{Bewerbung \\
auf eine Doktorandenstelle 
%zum Thema \\ \glqq Instationäre Aerodynamik von Schienenfahrzeugen\grqq \\
(Kz. 04370)}

\vspace{0.7in}

%ANREDE
Sehr geehrte Damen, sehr geehrter Herren,
\\

%TEXT
über die Internetpräsens des DLR bin ich auf die Ausschreibung der 
Doktorandenstelle zum Thema {\glqq Instationäre Aerodynamik von 
Schienenfahrzeugen\grqq} (Kz. 04370) aufmerksam geworden, die gleich mein 
Interesse geweckt hat.
\\

Zur Zeit studiere ich an der Technischen Universität Dortmund und werde 
voraussichtlich im Januar 2017 den Abschluss M.Sc. Physik erwerben. Im 
Verlauf meines Bachelorstudiums konnte ich Einblicke in verschiedene Bereiche 
der Festkörper- und Oberflächenphysik, Thermodynamik, statistischen Physik und 
experimentellen Teilchenphysik gewinnen sowie tiefere Kenntnisse der 
theoretischen Mechanik erwerben. Methoden der mathematischen Analyse und 
statistischen Datenverarbeitung vervollständigten das Gesamtbild. Das 
Vertiefungsgebiet 
meines Masterstudiums war die theoretische Teilchenphysik mit den 
Schwerpunkten Flavourphysik und Quanten-Eichfeldtheorie. Die 
Analyse experimenteller Daten war dabei eine grundlegende Fähigkeit, ebenso wie die 
Bildung- und Untersuchung mathematischer Modelle sowie deren numerische 
Simulation, um die experimentellen 
Befunde in eine sinnvolles Theorie einzubetten.

Die Analyse von mechanischen Problemen war für mich dabei jedoch immer 
besonders interessant, da diese einerseits greifbar und von theoretischer Seite 
aus elegant sind, andererseits aber schnell komplex werden und fortgeschrittene 
Methoden der Mathematik und Numerik verlangen.
Die Begeisterung für physikalische Forschung wurde insbesondere durch meine 
Masterarbeitsphase verstärkt, weshalb ich eine Promotion in der 
Naturwissenschaft anstrebe. 
Die ausgeschriebene Doktorandenstelle stellt für mich einen besonderen 
Reiz dar, da ein für mich äußerst interessantes Gebiet der Physik in direktem  
Zusammenspiel von Experiment und Theorie untersucht werden kann, und ich würde 
mit meinen bisher erworbenen Fähigkeiten gerne aktiv an dieser Forschungsarbeit 
mitwirken.
\\

Gerne würde ich Sie in einem persönlichen Gespräch davon überzeugen, dass ich 
motiviert bin, diese Stelle beim Deutschen Zentrum für Luft- und Raumfahrt 
anzunehmen und mich damit dieser neuen und spannenden Herausforderung zu 
stellen.
\\[0.9cm]

%SCHLUSS
\begin{flushright}
\begin{tabular}{c}
mit freundlichen Grüßen, 		\\[0.2cm]
Dominik Kahl 					\\[0.2cm]
Recklinghausen, den 13.12.2016 	\\[0.2cm]
\end{tabular}
\end{flushright}


\end{document}