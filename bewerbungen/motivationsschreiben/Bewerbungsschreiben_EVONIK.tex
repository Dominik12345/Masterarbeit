%------------%
%   Header   %
%------------%

\documentclass[12pt]{scrartcl}

\usepackage[T1]{fontenc}
\usepackage[utf8]{inputenc}

\usepackage{ngerman}
\usepackage{lmodern}


\setlength{\parindent}{0pt}
%---------------%
% Article Style %
%---------------%
\begin{document}
\pagestyle{empty}
%------------%
% Sections   %
%------------%

\section*{Bewerbung \\
auf eine Praktikumsstelle im Innovations- und Projektmanagement (Kz. 61319)}

\vspace{0.7in}

%ANREDE
Sehr geehrte Frau Vollbrecht,
\\

%TEXT
auf der Unternehmenskontaktmesse \textit{konaktiva} in Dortmund konnte 
ich mich über Ein\-stiegs- und 
Karrieremöglichkeiten bei EVONIK informieren. Über Ihre 
online-Stellen\-aus\-schrei\-bungen bin ich danach auf die Praktikumsstelle im 
Innovations- und Projektmanagement (Kz. 61319) gestoßen, die direkt mein Interesse geweckt 
hat.
\\

Ich studiere zur Zeit Physik an der Technischen Universität Dortmund und 
werde voraussichtlich im Januar 2017 den Abschluss Master of Science Physik 
erwerben. Durch mein Studium konnte ich Einblicke in verschiedene 
Bereiche der Festkörper- und Oberflächenphysik, Teilchenphysik und 
elektromagnetischen Phänomene erhalten.
Während die wissenschaftlichen und 
technischen Bereiche für mich ein gewohntes Arbeitsgebiet darstellen, ist 
die Analyse der wirtschaftlichen Problemstellungen eine spannende neue 
Herausforderung.
Unter Zuhilfenahme von mathematischen 
und statistischen Analyse- und Lösungsmethoden war es während meines Studiums  
immer wieder nötig, 
Kenntnisse 
aus verschiedenen Bereichen einzubringen, um Probleme effizient 
Lösen zu können. Ich glaube, dass ich diese Fähigkeit auch im Zusammenspiel 
von wirtschaftlichen und technischen Aspekten sinnvoll einbringen kann, um 
aktiv an einem Projekt mitzuwirken. 
\\

Ich würde mich freuen, wenn ich Sie in einem persönlichen Gespräch davon 
überzeugen kann, dass ich motiviert bin, im Rahmen dieser Praktikumsstelle 
mehr über die Arbeit bei EVONIK zu erfahren und mich gleichzeitig dem 
Unternehmen vorzustellen.
\\

%SCHLUSS
\begin{flushright}
\begin{tabular}{c}
mit freundlichen Grüßen, 		\\[0.2cm]
Dominik Kahl 					\\[0.2cm]
Recklinghausen, den 22.11.2016 	\\[0.3cm]
\end{tabular}
\end{flushright}

\end{document}