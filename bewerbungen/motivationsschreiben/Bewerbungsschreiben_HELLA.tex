%------------%
%   Header   %
%------------%

\documentclass[12pt]{scrartcl}

\usepackage[T1]{fontenc}
\usepackage[utf8]{inputenc}

\usepackage{ngerman}
\usepackage{lmodern}


\setlength{\parindent}{0pt}
%---------------%
% Article Style %
%---------------%
\begin{document}
\pagestyle{empty}
%------------%
% Sections   %
%------------%

\section*{Bewerbung \\
auf eine Doktorandenstelle 
%zum Thema \\ \glqq Instationäre Aerodynamik von Schienenfahrzeugen\grqq \\
(Kz. 04370)}

\vspace{0.4in}

%ANREDE
Sehr geehrte Dame, sehr geehrter Herr,
\\

%TEXT
über die Internetpräsens des DLR bin ich auf die Ausschreibung der 
Doktorandenstelle zum Thema {\glqq Instationäre Aerodynamik von 
Schienenfahrzeugen\grqq} (Kz. 04370) aufmerksam geworden, die gleich mein 
Interesse geweckt hat.
\\

Zur Zeit studiere ich an der Technischen Universität Dortmund und werde 
voraussichtlich im Januar 2017 den Abschluss M.Sc. Physik erwerben. Im 
Verlauf meines Bachelorstudiums konnte ich Einblicke in verschiedene Bereiche 
der Festkörper- und Oberflächenphysik, Thermodynamik, statistischen Physik und 
experimentellen Teilchenphysik gewinnen und tiefere Kenntnisse der 
theoretischen Mechanik erwerben. Methoden der Differenzialgeometrie in 
hamiltonschen Systemen und der statistischen Datenverarbeitung 
vervollständigten dies von mathematischer Seite aus. Das Vertiefungsgebiet 
meines Masterstudiums war die theoretische Teilchenphysik. Die mathematische 
Analyse experimenteller Daten war dabei eine Kernkompetenz, ebenso wie die 
Bildung- und Untersuchung mathematischer Modelle sowie deren numerische 
Simulation, um die experimentellen 
Befunde in ein sinnvolles Theoriemodell einzubetten.
Die Begeisterung für physikalische Forschung wurde insbesondere durch meine 
Masterarbeitsphase verstärkt, weshalb ich eine Promotion in der 
Naturwissenschaft anstrebe. 
Bereits seit dem Beginn meines Studiums hege ich ein besonderes Interesse an 
mechanischen Systemen, da diese als Probleme der klassischen Physik scheinbar 
leicht zugänglich sind, bei der näheren Untersuchung jedoch oft eine 
unerwartete physikalische und mathematische Komplexität offenbaren. Dies hat 
mich unter anderem dazu bewegt, auch im Masterstudium die 
Differentialgeometrie als Vertiefungsgebiet zu wählen, da sie für mich eine der 
elegantesten Beschreibungen solcher Probleme darstellt. 
Die ausgeschriebene Doktorandenstelle stellt für mich dabei einen besonderen 
Reiz dar, da ein für mich äußerst interessantes Gebiet der Physik in direktem  
Zusammenspiel von Experiment und Theorie untersucht werden kann, und ich würde 
mit meinen bisher erworbenen Fähigkeiten gerne aktiv an dieser Forschungsarbeit 
mitwirken.
\\

Gerne würde ich Sie in einem persönlichen Gespräch davon überzeugen, dass ich 
motiviert bin, diese Stelle beim Deutschen Zentrum für Luft- und Raumfahrt 
anzunehmen und mich damit dieser neuen und spannenden Herausforderung zu 
stellen.
\\[0.2cm]

%SCHLUSS
\begin{flushright}
\begin{tabular}{c}
mit freundlichen Grüßen, 		\\[0.3cm]
Dominik Kahl 					\\[0.3cm]
Recklinghausen, den 07.12.2016 	\\[0.6cm]
\end{tabular}
\end{flushright}

Ein tabellarischer Lebenslauf, das Abitur- und das Bachelorprüfungszeugnis 
sind dem Bewerbungsschreiben angehängt.
\end{document}