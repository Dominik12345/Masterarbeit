\documentclass{article}
\usepackage{amsmath, amssymb, graphics, setspace}

% font
\usepackage{lmodern}
\usepackage{fourier}
%\usepackage{tgheros}
%\usepackage{tgcursor}
\usepackage{tgpagella}
\usepackage{microtype}

% math
\usepackage{amsmath}
\usepackage{amsfonts}
\usepackage{amssymb}
\usepackage{amstext}
%\usepackage{isomath}
\usepackage{mathtools}


% encoding
\usepackage[T1]{fontenc}
\usepackage[utf8]{inputenc}

% language
\usepackage{ngerman}
% useful stuff
\usepackage{enumitem}
\usepackage{booktabs}


\newtheorem{theorem}{Theorem}[section]
\newtheorem{lemma}[theorem]{Lemma}
\newtheorem{proposition}[theorem]{Proposition}
\newtheorem{corollary}[theorem]{Corollary}

\newenvironment{proof}[1][Beweis]{\begin{trivlist}
\item[\hskip \labelsep {\bfseries #1}]}{\end{trivlist}}
\newenvironment{definition}[1][Definition]{\begin{trivlist}
\item[\hskip \labelsep {\bfseries #1}]}{\end{trivlist}}
\newenvironment{example}[1][Example]{\begin{trivlist}
\item[\hskip \labelsep {\bfseries #1}]}{\end{trivlist}}
\newenvironment{remark}[1][Bemerkung]{\begin{trivlist}
\item[\hskip \labelsep {\bfseries #1}]}{\end{trivlist}}

\newcommand{\qed}{\nobreak \ifvmode \relax \else
      \ifdim\lastskip<1.5em \hskip-\lastskip
      \hskip1.5em plus0em minus0.5em \fi \nobreak
      \vrule height0.75em width0.5em depth0.25em\fi}



\newcommand{\twopartdef}[2]
{
	\left(
		\begin{array}{ll}
			#1  \\
			#2 
		\end{array}
	\right)
}

\newcommand{\Sp}{ \text{Sp} \frac{\partial \beta}{\partial \alpha}}
\newcommand{\Det}{ \text{Det} \frac{\partial \beta}{\partial \alpha}}
\renewcommand{\d}{\text{d}}



\begin{document}

\tableofcontents

\clearpage
\section{Die $\beta$-Funktion}

  \subsection{$\beta(g)$}
      
	Aus ``Two-loop $\beta$-function for a $G_1\times G_2$ gauge symmetry'', D.R.T. Jones übernehmen 
	wir die allgemeine Form einer 2-loop-$\beta$-Funktion
	\begin{align}
	      \begin{aligned}
	      \beta_1(g_1,g_2)&=X_1 g_1^3+ Y_1 g_1^5+ Z_1g_1^3g_2^2\\
	      \beta_2(g_1,_g2)&=X_2 g_2^3+Y_2 g_2^5+ Z_2 g_1^2 g_2^3 \quad. 
	      \end{aligned} \label{beta(g)}
	\end{align}
	Zur besseren Übersichtlichkeit, definiere $\beta(g):=(\beta_1(g_1,g_2),\beta_2(g_1,g_2))^T$ sowie $g=(g_1,g_2)^T$.
	Außerdem ist $g=g(\mu)$ und
	\begin{equation}
	\beta=\mu \twopartdef {\frac{\d g_1}{\d \mu}} {\frac{\d g_2}{\d \mu}} 
	\end{equation}

    
  \subsection{$\beta(\alpha)$}
	Durch eine Neudefinition der Kopplungskonstanten kann das Problem vereinfacht werden. Die kritische Dimension für 
	QCD-artige Kopplungen ist $4$; die dimensionslosen Kopplungskonstanten können demnach über
	\begin{equation}
	\alpha_i(\mu):= \mathcal{N}_i g_i^2(\mu) \mu^{d-4}  \label{alpha}
	\end{equation}
	definiert werden, wobei $\mathcal{N}_i$ Normierungskonstanten und $d$ die Dimension der Raumzeit sind. Wir setzen 
	$d\equiv4$.\\
	Es folgt
	\begin{equation}
	\frac{\d \alpha_i}{\d \mu}=\mathcal{N}_i 2 g_i \frac{\d g_i}{\d \mu} \Rightarrow \frac{\d g_i}{\d \mu} g_i=\frac{1}{2\mathcal{N}_i} \frac{\d \alpha_i}{\d \mu} \label{dalpha}
	\end{equation}
	und damit aus \eqref{beta(g)}
	\begin{align}
	&&\mu \frac{\d g_i}{\d \mu}&= X_i g_i^3 +Y_ig_i^5 +Z_ig_i^3 g_j^2  						 & \\
	&\overset{\cdot g_i}{\Rightarrow}&\mu \frac{\d g_i}{\d \mu} g_i &= X_i g_i^4 +Y_ig_i^6 +Z_ig_i^4 g_j^2           & \\
	&\overset{\eqref{alpha},\eqref{dalpha}}{\Rightarrow}& \mu \frac{\d \alpha_i}{\d \mu} \frac{1}{2\mathcal{N}_i} 
	    &= \frac{X_i}{\mathcal{N}_i^2} \alpha_i^2 +\frac{Y_i}{\mathcal{N}_i^3}  \alpha_i^3 +\frac{Z_i}{\mathcal{N}_i^2\mathcal{N}_j}\alpha_i^2 \alpha_j      &\\
	&\overset{\cdot 2\mathcal{N}_i}{\Rightarrow}& \mu \frac{\d \alpha_i}{\d \mu}&= \frac{2X_i}{\mathcal{N}_i} \alpha_i^2+ \frac{2Y_i}{\mathcal{N}_i^2} \alpha_i^3 +
	      \frac{2 Z_i}{\mathcal{N}_i \mathcal{N}_j} \alpha_i^2 \alpha_j \quad .
	\end{align}
	Mit der Redefinition $\frac{2X_i}{\mathcal{N}_i} \longrightarrow X_i$,
	$\frac{2Y_i}{\mathcal{N}_i^2} \longrightarrow Y_i$,$\frac{2Z_i}{\mathcal{N}_i\mathcal{N}_j} \longrightarrow Z_i$ erhält man neue Funktionen
	\begin{align}
	 \begin{aligned}
	  \beta_1(\alpha_1,\alpha_2)&=X_1\alpha_1^2 + Y_1 \alpha_1^3 + Z_1 \alpha_1^2\alpha_2 \\
	  \beta_2(\alpha_1,\alpha_2)&=X_2\alpha_2^2 + Y_2 \alpha_2^3 + Z_2 \alpha_1\alpha_2^2 \quad.
	 \end{aligned}
	\end{align}
	Wieder ist
	\begin{equation}
	\beta=\mu \twopartdef {\frac{\d \alpha_1}{\d \mu}} {\frac{\d \alpha_2}{\d \mu}} 
	\end{equation}
	mit $\beta(\alpha)=(\beta_1(\alpha),\beta_2(\alpha))^T$ und $\alpha=(\alpha_1(\mu),\alpha_2(\mu))^T$.
	
	
  \subsection{Nullstellen der $\beta$-Funktion}
	
	Fixpunkte sind stationäre Lösungen von $\beta(\alpha)=0$, d.h.
	\begin{align}
	 0&=X_1 \alpha_1^2 +Y_1 \alpha_1^3 +Z_1 \alpha_1^2 \alpha_2 \\
	 0&=X_2 \alpha_2^2 +Y_2 \alpha_2^3 +Z_2 \alpha_1 \alpha_2^2  \quad.	 
	\end{align}
	Damit der Fixpunkt physikalisch sinnvoll ist muss $\alpha_i\geq 0$ sein, und der Fixpunkt sollte ein Punkt 
	im $\mathbb{R}^2$ sein, d.h. keine freien Parameter mehr besitzen (keine Untermannigfaltigkeit sein).\\
	Man findet 
	\begin{itemize}
	 \item den trivialen Fixpunkt $\alpha^{*1}=(0,0)$
	 \item den halb-trivialen Fixpunkt $\alpha^{*2}=(0,-\frac{X_2}{Y_2})$ falls $Y_2\neq 0$
	 \item den halb-trivialen Fixpunkt $\alpha^{*3}=(-\frac{X_1}{Y_1},0)$ falls $Y_1\neq 0$
	 \item den nichttrivialen Fixpunkt $\alpha^{*4}=\left(\frac{Z_1X_2-X_1Y_2}{Y_1Y_2-Z_1Z_2},\frac{X_1Z_2-Y_1X_2}{Y_1Y_2-Z_1Z_2}\right)$
	\end{itemize}
	als einzige zulässige Fixpunkte.
	\begin{proof}
	 \begin{itemize}
	  \item $\alpha^{*1}$ klar.
	  \item $\alpha^{*2/3}$ durch Einsetzen der Null und Reduktion auf ein lineares System $\Rightarrow$ Lösung eindeutig.
	  \item $\alpha^{*4}$ durch Reduktion auf ein inhomogenes lineares Gleichungssystem. Für $Y_1Y_2-Z_1Z_2\neq 0$ ist die Lösung 
		eindeutig. Für $Y_1Y_2-Z_1Z_2 = 0$ ist die Lösung eine Gerade im $\mathbb{R}^2$ und somit nicht zulässig.$\Box$
	 \end{itemize}
	\end{proof}



	
	
\clearpage	
\section{Stabilität von Fixpunkten}
 
  \subsection{Stabilitätsmatrix $\frac{\d \beta}{\d \alpha}$}
	Um die Stabilität von Fixpunkten zu untersuchen wird die Stabilitätsmatrix $\frac{\partial \beta}{\partial \alpha}$ als Jacobimatrix der 
	$\beta$-Funktion eingeführt
	\begin{equation}
	 \frac{\partial \beta}{\partial \alpha}=\begin{pmatrix}
	                                         \frac{\partial \beta_1}{\partial \alpha_1} & \frac{\partial \beta_1}{\partial \alpha_2} \\
	                                         \frac{\partial \beta_2}{\partial \alpha_1} & \frac{\partial \beta_2}{\partial \alpha_2}
	                                        \end{pmatrix} 
	                                       =\begin{pmatrix}
	                                         2X_1\alpha_1 +2Y_1 \alpha_1^2+2Z_1\alpha_1 \alpha_2 & Z_1 \alpha_1^2\\
	                                         Z_2 \alpha_2^2 & 2X_2 \alpha_2 +3 Y_2 \alpha_2^2 +2 Z_2 \alpha_1 \alpha_2
	                                        \end{pmatrix} \quad .
	\end{equation}

	
	
	\begin{proposition}
	      Die Vorzeichen der Eigenwerte von $\frac{\partial \beta}{\partial \alpha}$ lassen sich durch Spur und Determinante über die 
	      Tabelle 1 bestimmen.
	      \begin{table}
	      \begin{tabular}{cccc}
	      \toprule
		  $\Sp $ & $\Det $ 	& $\lambda_1$ 	&$ \lambda_2$ \\
		  \midrule 
		  $>0$	& $>0$		&$>0+i\mathbb{R}$		& $>0+i\mathbb{R}$		\\
		  $<0$	& $>0$		&$<0+i\mathbb{R}$		& $<0+i\mathbb{R}$		\\
		  $=0$	& $>0$		&$=0+i\mathbb{R}$		& $=0+i\mathbb{R}$		\\
		  -  	& $<0$		&$>0$		& $<0$		\\
		  $>0$	& $=0$		&$>0$		& $=0$		\\
		  $<0$	& $=0$		&$=0$		& $<0$		\\
		  $=0$	& $=0$		&$=0$		& $=0$		\\
		  \bottomrule
	      \end{tabular}
	      \label{Eigenwerte}\caption{Eigenwerte der Stabilitätsmatrix}
	  \end{table}
	\end{proposition}
	\begin{proof}
	      Das charakteristische Polynom der $2\times 2$-Matrix $\frac{\partial \beta}{\partial \alpha}$ ist
	      \begin{equation}
		P(\lambda)=\lambda^2-\Sp \lambda + \Det \quad ,
	      \end{equation}
	      die Eigenwerte sind die Nullstellen
	      \begin{equation}
		\lambda_{1/2}=\frac{\Sp}{2}\pm \sqrt{\left(\frac{\Sp}{2} \right)^2- \Det} \quad.
	      \end{equation}
		\begin{description}
		  \item[1. $\Det>\left(\frac{\Sp}{2} \right)^2$:] Dann ist $\Re(\lambda_{1/2})= \frac{\Sp}{2}$ 
		  \item[2. $\left(\frac{\Sp}{2} \right)^2>\Det>0$:] Dann ist $\sqrt{\left(\frac{\Sp}{2} \right)^2- \Det}<\left|\frac{\Sp}{2}\right|$ und\\ 
			    damit ${\text{sig } \Sp = \text{sig } \lambda_{1/2}}$
		  \item[3. $\Det<0:$] Dann ist $\sqrt{\left(\frac{\Sp}{2} \right)^2- \Det}>\left|\frac{\Sp}{2}\right|$ und somit
			    $\lambda_1>0$ und $\lambda_2<0$
		  \item[4. $\Det=0$:] Falls $\Sp>0$, dann $\lambda_1>0$, $\lambda_2=0$, falls $\Sp<0$, \\dann $\lambda_1=0$, $\lambda_2<0$. 
			    Falls $\Sp=0$, dann $\lambda_1=\lambda_2=0$.
		\end{description}
	  $\Box$
	\end{proof}

	
  \subsection{Autonome $\beta$-Funktion}
  Sei $\alpha^*$ ein Fixpunkt, d.h. $\beta(\alpha^*)=0$. Zunächst soll der Fixpunkt auf Null verschoben werden
  \begin{equation}
   \tilde{x}(\mu):=\alpha(\mu)-\alpha^* \Leftrightarrow \alpha(\mu)=\tilde{x}(\mu)+\alpha^* \quad.
  \end{equation}
  Wegen $\frac{\d \tilde{x}}{\d \mu}=\frac{\d \alpha}{\d \mu}$ ergibt sich
  \begin{equation}
   \mu \frac{\d \tilde{x}}{\d \mu}(\mu)=\left. \beta \circ \alpha \right|_{\tilde{x}} (\mu) \label{x}
  \end{equation}
  mit dem Fixpunkt $\tilde{x}^*=(0,0)$.\\
  Definiere die RG-Zeit $t:=\ln(\mu) \Leftrightarrow \mu=e^t$, sowie $x(t):=\tilde{x}(e^t)$. Durch Nachrechnen lässt sich feststellen
  \begin{equation}
   \mu \frac{\d \tilde{x}}{\d \mu} (\mu)=\mu \frac{\d \tilde{x}}{\d \mu} (e^t)= \mu \frac{\d x}{\d \mu}(t)=
	\mu \underbrace{\frac{\partial t}{\partial \mu}}_{=1/\mu} \frac{\d x}{\d t}(t)= \frac{\d x}{\d t} \quad ,
  \end{equation}
  außerdem muss $\mu\geq 0$ um physikalisch sinnvoll zu sein. Somit ist die Zuordnung $t=\ln \mu$ bijektiv und
  Gleichung \eqref{x} ist äquivalent zu dem autonomen DGL System
  \begin{equation}
   \frac{\d x}{\d t}=\left. \beta\circ \alpha \right|_{x} \label{x_autonom}
  \end{equation}

  
  \subsection{Lösung in einer Umgebung eines Fixpunktes}
  Durch linearisierung von \eqref{x_autonom} in der Nähe des Fixpunktes erhält man
  \begin{equation}
   \frac{\d x}{\d t}(t)=\left. \frac{\partial \beta}{\partial \alpha} \right|_{\alpha|_{(0,0)}}x (t) \quad.
  \end{equation}
  Die Lösung ist
  \begin{equation}
   x(t)=e^{\frac{\partial \beta}{\partial \alpha}t} x_0
  \end{equation}
  für einen Anfangswert $x_0$ nahe dem Fixpunkt.\\
  Seien nun $\lambda_{1/2}$ die Eigenwerte und $e_1,e_2$ die Eigenvektoren der Stabilitätsmatrix, so erhält man
  mit $T:=(e_1e_2)$
  \begin{align}
   x(t)&=T{\begin{pmatrix} e^{\lambda_1 t} &0\\0&e^{\lambda_2 t} \end{pmatrix}} T^{-1} \underbrace{( K_1 e_1 +K_2 e_2)}_{=x_0} \\
       &= ( e^{\lambda_1 t} e_1 \,  e^{\lambda_2 t} e_2) \begin{pmatrix} K_1 \\ K_2 \end{pmatrix} 
   \end{align}
   und schließlich 
   \begin{equation}
       x(t)= e^{\lambda_1 t} K_1 e_1 + e^{\lambda_2 t} K_2 e_2 \quad . 
       \label{loesung}
   \end{equation}
   

  
  \subsection{Asymptotisches Verhalten in der Nähe eines Fixpunktes (Klassifizierung von Fixpunkten)} \label{Klassifizierung}
	An Gleichung \eqref{loesung} ist leicht zu erkennen, dass
	\begin{itemize}
	\item wenn $\Re(\lambda_1)>0$ und $\Re(\lambda_2)>0$, dann 
			verlässt $x(t)$ die Nähe von $0$ für wachsendes $t$. Der 
			Fixpunkt ist dann repulsiv (IR-Fixpunkt/Bank-Zanks).
	\item wenn $\Re(\lambda_1)>0$ und $\Re(\lambda_2)>0$, dann gilt
			$\lim\limits_{t\to \infty} x(t) = 0$. Es handelt sich 
			um einen attraktiven (UV) Fixpunkt.
	\item wenn $\lambda_1 >0$ und $\lambda_2 <0$, dann 
			handelt es sich um einen Sattelpunkt:
			liegt $x_0$ auf der $e_1$-Geraden ($K_2=0$) entfernt sich $x(t)$,\\
			liegt $x_0$ auf der $e_2$-Geraden ($K_1=0$) nähert sich $x(t)$ dem 
			Fixpunkt an,
			sind $K_1\neq 0$ und $K_2 \neq 0$ so führt die Trajektorie am 
			Fixpunkt vorbei.
	\item Der Fall $\lambda_{1/2}\neq 0$, $\lambda_{2/1}=0$ wird in \ref{lambda0}
			untersucht.
	\end{itemize}


\clearpage
\section{Untersuchung der Fixpunkte}

	\subsection{Nichttrivialer Fixpunkt $\alpha^{*4}$}
 		Am Fixpunkt $\alpha^*:=\alpha^{*4}$ gilt
 		\begin{align}
 		\alpha^*&=\left(\frac{Z_1X_2-X_1Y_2}{Y_1Y_2-Z_1Z_2},\frac{X_1Z_2-Y_1X_2} 
 					{Y_1Y_2-Z_1Z_2}\right)\\
 		\left.\Sp \right|_*&=\frac{ X_1^2Y_2(Y_1Y_2+Z_2^2)
				-2X_1X_2Y_1Y_2(Z_1+Z_2)
				 +X_2^2Y_1(Y_1Y_2+Z_1^2) }{
				\left( Y_1Y_2-Z_1Z_2 \right)^{2}} \\
		\left.\Det \right|_*&=\frac{(X_2Y_1-X_1Z_2)^2(X_1Y_2-X_2Z_1)^2}{
				(Y_1Y_2-Z_1Z_2)^{3}} \quad .
		\end{align} 		 
 		Nach Abschnitt \ref{Klassifizierung} muss an einem UV-Fixpunkt 
 		gelten:
 		\begin{align}
 		\alpha_1^* &\geq 0 \\
 		\alpha_2^* &\geq 0 \\
 		\left.\Sp \right|_*& < 0 \\
 		\left.\Det \right|_*& > 0 \quad	
 		\end{align}
 		mit den vorherige Gleichungen also
 		\begin{align}
 		\frac{Z_1X_2-X_1Y_2}{Y_1Y_2-Z_1Z_2} &\geq 0 \label{I}\\
 		\frac{X_1Z_2-Y_1X_2} {Y_1Y_2-Z_1Z_2} &\geq 0 \label{II}\\
 		\frac{ X_1^2Y_2(Y_1Y_2+Z_2^2)
				-2X_1X_2Y_1Y_2(Z_1+Z_2)  +
				X_2^2Y_1(Y_1Y_2+Z_1^2)}{
				 (Y_1Y_2-Z_1Z_2)^2}&< 0 \label{III} \\
 		\frac{(X_2Y_1-X_1Z_2)^2(X_1Y_2-X_2Z_1)^2}
				{(Y_1Y_2-Z_1Z_2)^{3}}&> 0 	\label{IV} \quad .
 		\end{align}
 		Dies ist jedoch nicht möglich.
 		\begin{proof}
			Aus \eqref{IV} folgt 
			\begin{equation}
				Y_1Y_2-Z_1Z_2 > 0 \Rightarrow Y_1Y_2>Z_1Z_2 \quad . \label{Ia}
			\end{equation}			 		
 			Damit folgt in \eqref{I} und \eqref{II}
 			\begin{align}
 			Z_1X_2 &\geq X_1Y_2 \label{IIa}\\
 			X_1Z_2 &\geq Y_1X_2 \label{IIIa} \quad.
 			\end{align}
 			Gleichung \eqref{III} ist äquivalent zu
 			\begin{align}
 			 X_1^2Y_2(Y_1Y_2+Z_2^2)+X_2^2Y_1(Y_1Y_2+Z_1^2) < 
 			 	2X_1X_2Y_1Y_2(Z_1+Z_2) \quad. \label{IVa}
 			\end{align}
 			Wir können die linke Seite weiter nach unten Abschätzen:
 			\begin{align}
 			&X_1^2Y_2(Y_1Y_2+Z_2^2)+X_2^2Y_1(Y_1Y_2+Z_1^2) \\
 			 &=X_1 (X_1Y_2) (\underbrace{Y_1Y_2}_\eqref{Ia}+Z_2^2)
 			 +X_2 (Y_1X_2)(\underbrace{Y_1Y_2}_\eqref{Ia}+Z_1^2) \\
 			 &>X_1 (X_1Y_2)(Z_1Z_2+Z_2^2)+X_2(Y_1X_2) (Z_1Z_2+Z_1^2)\\
 			 &=(Z_1+Z_2)\left[\underbrace{(X_1 Z_2)}_\eqref{IIIa}
 			  (X_1Y_2)+\underbrace{(Z_1X_2)}_\eqref{IIa}(Y_1X_2) \right] \\
 			 &\geq(Z_1+Z_2)[Y_1X_2X_1Y_2+Y_1X_2Y_2X_1]\\
 			 &=  (Z_1+Z_2) 2 X_1X_2Y_1Y_2
 			\end{align}
 			Also haben wir gefunden:
 			\begin{equation}
 			(Z_1+Z_2)2X_1X_2Y_1Y_2<X_1^2Y_2(Y_1Y_2+Z_2^2)+X_2^2Y_1(Y_1Y_2+Z_1^2) 
 			< 2X_1X_2Y_1Y_2(Z_1+Z_2) \quad,
 			\end{equation}
 			dies ist jedoch falsch.
 			Folglich kann $\alpha^{*4}$ mit $\Sp\neq 0$ und $\Det\neq 0$ kein 
 			kein UV-Fixpunkt sein.			
 		\end{proof}
 		\begin{remark}
 		Der Beweis ist auch gültig, falls in \eqref{III} oder \eqref{IV} 
 		''$\geq$'' bzw. ''$\leq$''  zugelassen wird. Nur bei $\Sp |_*=\Det |_*=0$ 
 		gilt der Beweis nicht.\\
 		In diesem Fall wird das Problem jedoch uninteressant:
 		
 		\end{remark}		


	\subsection{Halb-trivialer Fixpunkt $\alpha^{*3}$}\label{lambda0}
		Am Fixpunkt $\alpha^*:=\alpha^{*3}$ gilt \footnote{Es soll $X_1\neq 0$, 
		sonst wird der Fixpunkt trivial.}
		\begin{align}
		\alpha*&=\left(-\frac{X_1}{Y_1},0\right) \\
		\left.\Sp \right|_* &= \frac{X_1^2}{Y_1} \\
		\left.\Det \right|_* &=0 \quad .
		\end{align}
 		Da hier mindestens ein Eigenwert gleich Null ist kann mit 
 		Tabelle \ref{Eigenwerte} keine Aussage getroffen werden. Hier wird 
 		das Verhalten der Eigenwerte in der Nähe von $\alpha^*$ zur Hilfe 
 		genommen.
 		\subsubsection{Stabilitätsbedingungen für $\alpha^{*3}$}
 		Die Eigenwerte an $\alpha$ sind
 		\begin{equation}
 		\lambda_{1/2}=\frac{\Sp}{2}\pm \sqrt{\left(\frac{\Sp}{2} \right)^2-\Det}
 		\quad .
 		\end{equation}
 		An $\alpha^*$ ist die Stabilitätsmatrix 
 		\begin{equation}
 		\left.\frac{\partial \beta}{\partial \alpha} \right|_* 
	                                       =\begin{pmatrix}
	                                         \frac{X_1^2}{Y_1} & 
	                                         Z_1\left( \frac{X_1}{Y_1}\right)^2\\
	                                         0&0
	                                        \end{pmatrix} \quad .
 		\end{equation}
 		Die Eigenwerte sind $\lambda_a=\Sp|_*=\frac{X_1^2}{Y_1}$ und 
 		$\lambda_b=0$ zu den Einvektoren 
 		$e_a=(1,0)^T$ bzw. $e_b=(0,1)^T$. Das heißt in 
 		$e_a \hat{=} \alpha_1$-Richtung ist 
 		der Fixpunkt attraktiv/repulsiv, wenn $\Sp< / > 0$ ist (vgl. 
 		\eqref{loesung}). Es bleibt also noch die $e_b\hat{=}\alpha_2$-Richtung 
 		zu untersuchen:\\
 		
 		
 		$\alpha^*$ liegt am Rand des physikalisch 
 		sinnvollen Definitionsbereiches (in $\alpha_2$-Richtung) mit 
 		$\lim\limits_{\alpha_2\to 0}\lambda_b|_{\alpha_1=\alpha^*_1}=0$.
 		\begin{itemize}
 		\item $\left.\frac{\partial \lambda_b}{\partial \alpha_2}\right|_*<0$, 
 				dann nähert sich $\lambda_2$ \\ von unten (d.h. $\lambda_2<0$ 
 				bei $\alpha=(\alpha^*_1,\epsilon)$, $0<\epsilon<<1$).
 		\item $\left.\frac{\partial \lambda_b}{\partial \alpha_2}\right|_*>0$, 
 				dann nähert sich $\lambda_2$ von oben. 
 		\item $\left.\frac{\partial \lambda_b}{\partial \alpha_2}\right|_*=0$,
 				es müssen höhere Ableitungen betrachtet werden (vgl. 
 				\ref{dlambda0}). 
		\end{itemize} 
		Wir erhalten
		\begin{equation}
		\frac{\partial \lambda_b}{\partial \alpha_2}=\frac{1}{2}
		\frac{\partial \Sp}{\partial \alpha_2}+ \frac12 \sqrt{\left( 
		\frac{\Sp}{2}\right)}^{-1}\left(  \frac{\partial \Sp}{\partial 
		\alpha_2}\Sp-\frac{\partial \Det}{\partial \alpha_2} \right)
		\end{equation}
 		mit
 		\begin{align}
 		\Sp&=2X_1\alpha_1 +3Y_1\alpha_1^2+2Z_1\alpha_1\alpha_2+2X_2\alpha_2 
 				+Y_2\alpha_2^2+2 Z_2 \alpha_1 \alpha_2\\ 
 		\frac{\partial \Sp}{\partial \alpha_2}&=2\alpha_1(Z_1+Z_2)+
 				2X_2+6Y_2\alpha_2\\
 		\Det&=(2X_1\alpha_1+3Y_1\alpha_1^2+2Z_1\alpha_1\alpha_2)
 		(2X_2\alpha_2+3Y_2\alpha_2^2+2Z_2\alpha_1\alpha_2)
 		-Z_1Z_2\alpha_1^2\alpha_2^2\\
 		\frac{\partial \Det}{\partial \alpha_2}&=
 		2Z_1\alpha_1(2X_2\alpha_2+3Y_2 \alpha_2^2+2Z_2 \alpha_1\alpha_2)
 		+(2X_1\alpha_1+3Y_1\alpha_1^2+2Z_1\alpha_1\alpha_2)(2X_2+6Y_2\alpha_2
 		+2Z_2 \alpha_1)-2Z_1Z_2 \alpha_1^2 \alpha_2
 		\end{align}
 		und ausgewertet an $\alpha^*$ ergibt sich
 		\begin{align}
 		\Sp|_*&=\frac{X_1^2}{Y_1} \\
 		\frac{\partial \Sp}{\partial \alpha_2}|_*&=-2\frac{X_1}{Y_1}(Z_1+Z_2)
 		+2X_2 \\
 		\Det|_*&=0\\
 		\frac{\partial \Det}{\partial \alpha_2}|_*&=\frac{X_1^2}{Y_1}(2X_2-2Z_2
 		\frac{X_1}{Y_1}) \quad.
 		\end{align}
 
 		\begin{itemize}
 		\item Für $\Sp|_*>0$ ist $\lambda_b=\lambda_2$ (''$-$''-Lösung) und 	
 		$\lambda_1>0$,
 		\item für $\Sp|_*<0$ ist $\lambda_b=\lambda_1$ (''$+$''-Lösung) und 
 		$\lambda_2<0$.
 		\end{itemize}
 		Für beide Fälle ergibt sich
 		\begin{equation}
 		\frac{\partial \lambda_b}{\partial \alpha_2}|_*=X_2+\frac{X_1}{Y_1}
 		(Z_1-Z_2) \quad .
 		\end{equation}
 		Wir erhalten also die Bedingungen in Tabelle \ref{stab3}.
 		
 		\begin{table}
 		\begin{tabular}{ccc}
 		\toprule
 		$\Sp|_*$ & $\frac{\partial \lambda_b}{\partial \alpha_2}|_*$ & Fixpunkt\\
 		\midrule
 		$>0$ & $>0$ & repulsiv \\
 		$>0$ & $<0$ & Sattelpunkt\\
 		$<0$ & $>0$ & Sattelpunkt\\
 		$<0$ & $<0$ & attraktiv\\
 		\bottomrule 
 		\end{tabular}
 		\caption{Stabilitätsbedingungen für $\alpha^{*3}$.} \label{stab3}
 		\end{table}
 		
 		\subsubsection{$\alpha^{*3}$ als UV-Fixpunkt}
 		Damit $\alpha^{*3}$ ein UV-Fixpunkt ist muss also $\alpha^*_i>0$ und 
 		nach Tabelle \ref{stab3} gelten:
 		\begin{align}
 		-\frac{X_1}{Y_1}&\geq 0 \\
 		\frac{X_1^2}{Y_1} &< 0 \\
 		X_2+\frac{X_1}{Y_1}(Z_1-Z_2)&<0
 		\end{align}
 
\end{document}
